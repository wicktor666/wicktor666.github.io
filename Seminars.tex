\documentclass{article}
\usepackage[utf8]{inputenc}
\usepackage[russian]{babel}
\usepackage{amsmath}
\usepackage{amssymb}
\usepackage{graphicx}
\usepackage{cancel}

\title{Вычисление пределов функций двух переменных \\ методом полярных координат}
\author{Ваше имя}
\date{\today}

\begin{document}

\maketitle

\section{Теоретическая основа}
Для исследования предела функции $f(x,y)$ при $(x,y) \to (0,0)$ используем полярные координаты:
\[
x = r\cos\theta, \quad y = r\sin\theta, \quad r = \sqrt{x^2 + y^2}
\]
Тогда:
\[
\lim_{(x,y)\to(0,0)} f(x,y) = \lim_{r\to 0} f(r\cos\theta, r\sin\theta)
\]

\section{Пример 1: Простая рациональная функция}
Исследуем предел:
\[
\lim_{(x,y)\to(0,0)} \frac{x^2 y}{x^2 + y^2}
\]

\subsection{Решение}
\begin{align*}
\frac{x^2 y}{x^2 + y^2} &= \frac{r^2 \cos^2\theta \cdot r\sin\theta}{r^2 \cos^2\theta + r^2 \sin^2\theta} \\
&= \frac{r^3 \cos^2\theta \sin\theta}{r^2 (\cos^2\theta + \sin^2\theta)} \\
&= r \cos^2\theta \sin\theta
\end{align*}
Поскольку $|\cos^2\theta \sin\theta| \leq 1$ для всех $\theta$, то:
\[
\lim_{r\to 0} r \cos^2\theta \sin\theta = 0
\]
\textbf{Ответ:} $\boxed{0}$

\section{Пример 2: Зависимость от пути}
Исследуем:
\[
\lim_{(x,y)\to(0,0)} \frac{xy}{x^2 + y^2}
\]

\subsection{Решение}
\begin{align*}
\frac{xy}{x^2 + y^2} &= \frac{r\cos\theta \cdot r\sin\theta}{r^2} \\
&= \cos\theta \sin\theta = \frac{1}{2}\sin 2\theta
\end{align*}
Предел зависит от $\theta$:
\begin{itemize}
\item При $\theta = 0$: значение $0$
\item При $\theta = \pi/4$: значение $\frac{1}{2}$
\end{itemize}
\textbf{Вывод:} Предел не существует.

\section{Пример 3: Кубическая функция}
Вычислим:
\[
\lim_{(x,y)\to(0,0)} \frac{x^3 + y^3}{x^2 + y^2}
\]

\subsection{Решение}
\begin{align*}
\frac{x^3 + y^3}{x^2 + y^2} &= \frac{r^3(\cos^3\theta + \sin^3\theta)}{r^2} \\
&= r(\cos^3\theta + \sin^3\theta)
\end{align*}
Так как $|\cos^3\theta + \sin^3\theta| \leq 2$:
\[
\lim_{r\to 0} r(\cos^3\theta + \sin^3\theta) = 0
\]
\textbf{Ответ:} $\boxed{0}$

\section{Критерии существования предела}
\begin{enumerate}
\item Если выражение стремится к 0 при $r\to 0$ \textbf{независимо} от $\theta$ → предел существует
\item Если результат зависит от $\theta$ → предел не существует
\item Если можно выделить $r^k$ с $k>0$ → предел вероятно существует
\end{enumerate}



\newpage

\section*{Принцип сжимающих отображений}

\subsection*{Задача 1 (Чистая математика)}
\textbf{Условие:} Доказать, что уравнение \(x = \cos x\) имеет единственное решение на \([0, 1]\). Найти его с точностью \(0.001\) методом итераций (\(x_0 = 0.5\)).

\textbf{Решение:}
\begin{enumerate}
\item Функция \(g(x) = \cos x\):
\begin{itemize}
\item Отображает \([0, 1]\) в \([\cos 1, 1] \approx [0.54, 1] \subset [0, 1]\)
\item \(|g'(x)| = |-\sin x| = \sin x \leq \sin 1 \approx 0.8415 < 1\)
\end{itemize}
Значит, \(g\) -- сжимающее отображение \(\Rightarrow\) решение существует и единственно.

\item Итерации:
\begin{align*}
x_0 &= 0.5 \\
x_1 &= \cos(0.5) \approx 0.8776 \\
x_2 &= \cos(0.8776) \approx 0.6390 \\
x_3 &= \cos(0.6390) \approx 0.8027 \\
x_4 &= \cos(0.8027) \approx 0.6948 \\
&\cdots \\
x_{15} &\approx 0.7396 \quad (\text{разность с предыдущим} < 0.001)
\end{align*}
Ответ: \(x^* \approx 0.740\)
\end{enumerate}

\subsection*{Задача 2 (Экономика)}
\textbf{Условие:} Спрос и предложение: 
\[
D(p) = \frac{100}{1 + p}, \quad S(p) = \frac{p}{2} - 5.
\]
Равновесная цена: \(D(p^*) = S(p^*)\).
\begin{enumerate}
\item Показать, что \(p = g(p) = 2\left(\frac{100}{1 + p} + 5\right)\)
\item Доказать сжимаемость \(g(p)\) на \([15, 25]\)
\item Найти \(p^*\) (3 итерации, \(p_0 = 15\))
\end{enumerate}

\textbf{Решение:}
\begin{enumerate}
\item Из \(D(p) = S(p)\):
\[
\frac{100}{1 + p} = \frac{p}{2} - 5 \Rightarrow p = 2\left(\frac{100}{1 + p} + 5\right)
\]

\item Производная:
\[
|g'(p)| = \left| -\frac{200}{(1 + p)^2} \right| \leq \frac{200}{16^2} = \frac{200}{256} \approx 0.78 < 1
\]
\(g([15,25]) \approx [16.25, 18.33] \subset [15,25]\)

\item Итерации:
\begin{align*}
p_0 &= 15 \\
p_1 &= 2\left(\frac{100}{16} + 5\right) = 22.5 \\
p_2 &= 2\left(\frac{100}{23.5} + 5\right) \approx 18.52 \\
p_3 &= 2\left(\frac{100}{19.52} + 5\right) \approx 20.24
\end{align*}
Ответ: \(p^* \approx 20.2\)
\end{enumerate}

\section*{Теорема об обратном отображении}

\subsection*{Задача 3}
\textbf{Условие:} Для \(f(x,y) = (x^2 - y^2, 2xy)\):
\begin{enumerate}
\item Матрица Якоби в \((1,1)\)
\item Проверить локальную обратимость
\item Найти \(J_{f^{-1}}\) в \(f(1,1)\)
\end{enumerate}

\textbf{Решение:}
\begin{enumerate}
\item 
\[
J_f = \begin{pmatrix}
2x & -2y \\
2y & 2x
\end{pmatrix} \xrightarrow{(1,1)} \begin{pmatrix}
2 & -2 \\
2 & 2
\end{pmatrix}
\]

\item \(\det J_f = 2\cdot2 - (-2)\cdot2 = 8 \neq 0 \Rightarrow\) обратимо

\item \(f(1,1) = (0,2)\),
\[
J_{f^{-1}}(0,2) = \frac{1}{8}\begin{pmatrix}
2 & 2 \\
-2 & 2
\end{pmatrix} = \begin{pmatrix}
0.25 & 0.25 \\
-0.25 & 0.25
\end{pmatrix}
\]
\end{enumerate}

\section*{Теорема о неявной функции}

\subsection*{Задача 4}
\textbf{Условие:} Для уравнения \(x^3 + y^3 - 3xy = 1\):
\begin{enumerate}
\item Показать, что в окрестности \((1,0)\) существует \(y = y(x)\)
\item Найти \(\frac{dy}{dx}\) в \((1,0)\)
\end{enumerate}

\textbf{Решение:}
\begin{enumerate}
\item \(F(x,y) = x^3 + y^3 - 3xy - 1\)
\begin{itemize}
\item \(F(1,0) = 0\)
\item \(\frac{\partial F}{\partial y} = 3y^2 - 3x \xrightarrow{(1,0)} -3 \neq 0\)
\end{itemize}
Условия теоремы выполнены

\item 
\[
\frac{dy}{dx} = -\frac{F_x}{F_y} = -\frac{3x^2 - 3y}{3y^2 - 3x} \xrightarrow{(1,0)} -\frac{3}{-3} = 1
\]
\end{enumerate}

\section*{Метод множителей Лагранжа}

\subsection*{Задача 5 (Ограничение-равенство)}
\textbf{Условие:} Найти \(\min x^2 + y^2\) при \(x + y = 1\).

\textbf{Решение:}
\begin{enumerate}
\item Функция Лагранжа:
\[
\mathcal{L} = x^2 + y^2 + \lambda(1 - x - y)
\]

\item Система:
\[
\begin{cases}
2x - \lambda = 0 \\
2y - \lambda = 0 \\
x + y = 1
\end{cases} 
\Rightarrow x = y = 0.5, \lambda = 1
\]

\item Значение: \(f = 0.25 + 0.25 = 0.5\) (минимум)
\end{enumerate}

\subsection*{Задача 6 (Ограничения-неравенства)}
\textbf{Условие:} Максимизировать \(U(x,y) = xy\) при \(2x + 3y \leq 100\), \(x \geq 0\), \(y \geq 0\).

\textbf{Решение:}
\begin{enumerate}
\item \textbf{Внутренняя область:} Нет критических точек (\(\nabla U = (y,x) \neq 0\))

\item \textbf{Границы:}
\begin{itemize}
\item Случай \(x=0\): \(U=0\)
\item Случай \(y=0\): \(U=0\)
\item Случай \(2x + 3y = 100\):
\begin{align*}
\mathcal{L} &= xy + \lambda(100 - 2x - 3y) \\
\begin{cases}
y - 2\lambda = 0 \\
x - 3\lambda = 0 \\
2x + 3y = 100
\end{cases} 
&\Rightarrow x = 25,\ y \approx 16.67,\ U \approx 416.67
\end{align*}
\end{itemize}

\item \textbf{Результат:} Максимум в \((25, 16.67)\)
\end{enumerate}

\subsection*{Задача 7 (Самостоятельно)}
\textbf{Условие:} Максимизировать \(f(x,y) = 4x + 3y\) при \(x^2 + y^2 \leq 25\), \(x \geq 0\), \(y \geq 0\).

\textbf{Подсказка:} Решить на границе \(x^2 + y^2 = 25\) методом Лагранжа.


\section*{Задача 1}
Найти экстремумы $f(x, y) = x^2 + y^2$ при $x + 2y = 1$.

\paragraph*{Решение:}
Функция Лагранжа:
\[
\mathcal{L} = x^2 + y^2 - \lambda(x + 2y - 1)
\]

Условия:
\begin{align*}
\frac{\partial \mathcal{L}}{\partial x} &= 2x - \lambda = 0 \\
\frac{\partial \mathcal{L}}{\partial y} &= 2y - 2\lambda = 0 \\
\frac{\partial \mathcal{L}}{\partial \lambda} &= -(x + 2y - 1) = 0
\end{align*}

Решение: $\lambda = 2x$, $y = 2x$, $x + 2(2x) = 1 \Rightarrow x = 0.2$, $y = 0.4$

Матрица Гессе:
$\begin{pmatrix} 2 & 0 \\ 0 & 2 \end{pmatrix}$, 
Касательный вектор $\mathbf{v} = (2, -1)$:
\[
(2,-1)\begin{pmatrix} 2 & 0 \\ 0 & 2 \end{pmatrix}\begin{pmatrix} 2 \\ -1 \end{pmatrix} = 10 > 0
\]
Локальный минимум в $(0.2, 0.4)$

\section*{Задача 2}
Найти экстремумы $f(x,y) = x^2 + 2y^2$ при $x^2 + y^2 = 1$.

\paragraph*{Решение:}
Функция Лагранжа:
\[
\mathcal{L} = x^2 + 2y^2 - \lambda(x^2 + y^2 - 1)
\]

Условия:
\begin{align*}
2x - 2\lambda x &= 0 \\
4y - 2\lambda y &= 0 \\
x^2 + y^2 &= 1
\end{align*}

Критические точки:
\begin{itemize}
\item $x=0, y=\pm1, \lambda=2, f=2$
\item $y=0, x=\pm1, \lambda=1, f=1$
\end{itemize}

Анализ:
\begin{itemize}
\item Для $(1,0)$: $H = \begin{pmatrix} 0 & 0 \\ 0 & 2 \end{pmatrix}$, $Q((0,1)) = 2 > 0$ (минимум)
\item Для $(0,1)$: $H = \begin{pmatrix} -2 & 0 \\ 0 & 0 \end{pmatrix}$, $Q((1,0)) = -2 < 0$ (максимум)
\end{itemize}

\section*{Задача 3}
Найти точку на $y = x^2$ ближайшую к $(1,0)$.

\paragraph*{Решение:}
Минимизировать $f = (x-1)^2 + y^2$ при $y = x^2$.

Функция Лагранжа:
\[
\mathcal{L} = (x-1)^2 + y^2 - \lambda(y - x^2)
\]

Условия:
\begin{align*}
2(x-1) + 2\lambda x &= 0 \\
2y - \lambda &= 0 \\
y - x^2 &= 0
\end{align*}

Решение: $\lambda = 2y$, $y = x^2$, $2(x-1) + 4x^3 = 0$
Корень $x \approx 0.59$, $y \approx 0.348$

Анализ: $H \approx \begin{pmatrix} 3.392 & 0 \\ 0 & 2 \end{pmatrix}$, $Q \approx 6.17 > 0$ (минимум)

\section*{Задача 4}
Для $f(x,y) = x^3 + y^3$ при $x + y = 1$ найти экстремумы.

\paragraph*{Решение:}
Функция Лагранжа:
\[
\mathcal{L} = x^3 + y^3 - \lambda(x + y - 1)
\]

Условия:
\begin{align*}
3x^2 - \lambda &= 0 \\
3y^2 - \lambda &= 0 \\
x + y &= 1
\end{align*}

Решение: $x^2 = y^2$, $x = y = 0.5$, $\lambda = 0.75$

Матрица Гессе: $\begin{pmatrix} 3 & 0 \\ 0 & 3 \end{pmatrix}$, 
Касательный вектор $\mathbf{v} = (1, -1)$:
\[
(1,-1)\begin{pmatrix} 3 & 0 \\ 0 & 3 \end{pmatrix}\begin{pmatrix} 1 \\ -1 \end{pmatrix} = 6 > 0
\]
Локальный минимум в $(0.5, 0.5)$

\end{document}