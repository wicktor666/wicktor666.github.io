\chapter{Начала теории рядов}

\section{Лекция \#7. Числовые ряды}

\subsection{Основные понятия}

\begin{definition}
    Пара последовательностей $(x_n)_{n \ge 1}$, $(\mathsf{S}_n)_{n \ge 1}$ называется \textit{рядом}, если их элементы $x_n$, $\mathsf{S}_n$ при любом $n$ связаны соотношениями
    \[
     \mathsf{S}_n = x_1 + \cdots + x_n,
    \]
    или, что равносильно
    \[
     x_1=\mathsf{S}_1, \qquad x_n = \mathsf{S}_n-\mathsf{S}_{n-1}, \qquad n\ge 1.
    \]
где мы, для удобств, положили, что $\mathsf{S}_0:=0$.

Элемент $x_n$ называется \textit{$n$-ым элементом}, \textit{$\mathsf{S}_n$ -- $n$-й частичной суммой} ряда. 

\begin{mydanger}{\bf !}
 Рассматриваемый ряд часто называется \textit{рядом с общим элементом $x_n$} или просто \textit{рядом} $(x_n)$. Принято также писать, что дан ряд $\sum_{n=1}^\infty x_n$, что следует признать неудачным.\footnote{Так как символ $\sum$ указывает на процесс суммирования (то есть это операция), а ряд мы себе мыслим всё же как последовательность. Впрочем, я не одинок в таком представлении, определение взято из книги Ж. Дьёдонне ``Основы современного анализа''.}
\end{mydanger}
\end{definition}

\begin{example} Приведём некоторые примеры.
    \begin{enumerate}
        \item Ряд
        \[
         \left(\frac{1}{n} \right)_{n\ge 1}, \qquad \left(\sum_{k=1}^n \frac{1}{n} \right)_{n\ge 1}
        \]
        называется \textit{гармоническим рядом}, также пишут $\sum_{n=1}^\infty \frac{1}{n}.$
        \item
        \[
        (q^n)_{n\ge 1},\qquad (1+q+\cdots +q^n)_{n\ge 1}.
        \]
        \item
        \[
         (n)_{n \ge 1}, \qquad \left(1+2+\cdots + n \right)_{n \ge 1}.
        \]
        
        \item Ряд 
        \[
         \left(\frac{(-1)^k}{2k+1}\right)_{k \ge 0}, \qquad \left(1-\frac{1}{3} + \frac{1}{5} + \cdots + \frac{(-1)^k}{2k+1}\right)_{k\ge 0} 
        \]
       называется \textit{рядом Мадхавы -- Лейбница (Madhava--Leibniz series)}. 
        \end{enumerate}    
\end{example}




\begin{definition}
    Ряд $(x_n)$ называется \textit{сходящимся к $s$}, если $\lim_{n \to \infty}\mathsf{S}_n = \mathsf{S}$; в этом случае $\mathsf{S}$ называют \textit{суммой} ряда и пишут $s = x_1 + x_2+\cdots + x_n + \cdots$ или $\mathsf{S} = \sum_{n=1}^\infty x_n$. Величина $r_n: = \mathsf{S}-\mathsf{S}_n$ называется \textit{$n$-ым остатком ряда}.
\end{definition}

\begin{example}\label{geometric_series_and_ML_series} Вернёмся к предыдущим примерам.
    \begin{enumerate}
        \item Рассмотрим ряд $(q^n)$, нетрудно видеть, что последовательность его частичных сумм описывается как
        \[
         \mathsf{S}_n  = \begin{cases}
             \frac{1-q^n}{1-q}, & q \ne 1,\\
             n, & q =1.
         \end{cases}
        \]
Тогда, если $q \ne 1$, имеем
\[
 \lim_{n \to \infty} \mathsf{S}_n = \lim_{n \to \infty} \frac{1-q^n}{1-q} = \frac{1}{1-q} \left( 1- \lim_{n \to \infty} q^n \right)
\]
\textit{т.е.} предел существует, если и только если $q<1$. Таким образом, ряд $(q^n)$ сходится, если и только если $q<1.$

\item Рассмотрим ряд Мадхавы--Лейбница, мы позже докажем, что оказывается, он сходящийся, и более того, мы покажем, что имеет место красивая формула
\[
 \frac{\pi}{4} = \lim_{k \to \infty} \mathsf{S}_k = \sum_{k=1}^\infty \frac{(-1)^k}{2k+1} = 1-\frac{1}{3} + \frac{1}{5} - \frac{1}{7} + \frac{1}{11}-\ldots.
\]
    \end{enumerate}
\end{example}


\begin{lemma}
    Пусть $(x_n)_{n\ge 1}$ -- сходящийся ряд и $\mathsf{S}$ -- его сумма, тогда $\lim_{n \to \infty }r_n = 0.$
\end{lemma}

\begin{proof}
    Имеем $s = x_1+x_2+\cdots + x_n +\cdots$, $\mathsf{S}_n = x_1 + x_2 + \cdots + x_n$, тогда $r_n = \mathsf{S}-\mathsf{S}_n = x_{n+1}+ x_{n+2} + \cdots + x_{n+k} + \cdots$. Так как по условию $\lim_{n \to \infty} \mathsf{S}_n = \mathsf{S}$, то по арифметике предела (Теорема \ref{a+b,ca,ab})
    \begin{eqnarray*}
     \lim_{n \to \infty} r_n &=& \lim_{n \to \infty}(\mathsf{S}- \mathsf{S}_n) \\
     &=&  \lim_{n \to \infty} \mathsf{S} -  \lim_{n \to \infty} \mathsf{S}_n \\
     &=& \mathsf{S}- \mathsf{S} = 0.
    \end{eqnarray*}
\end{proof}


\subsection{Критерий Коши и некоторые признаки сходимости}

\begin{theorem}[Критерий Коши]\label{Coshy2}
    Ряд $(x_n)_{n\ge 1}$ сходится, если и только если для любого $\varepsilon >0$ существует такой номер $N$, что при $n \ge N$, $p\ge 1$ имеет место неравенство
    \[
     |\mathsf{S}_{n+p} -\mathsf{S}_n| = |x_{n+1}+ \cdots + x_{n+p}| < \varepsilon.
    \]
\end{theorem}

\begin{proof}
  Для ряда $(x_n)$ рассмотрим последовательность $(\mathsf{S}_n)$ его частичных сумм, тогда согласно определению $(x_n)$ сходится, если и только если $\lim_{n \to \infty} \mathsf{S}_n =\mathsf{S}$, а тогда по критерию Коши \ref{Coshy} $(\mathsf{S}_n)$ -- фундаментальная (Определение \ref{foundamental_sequence}), \textit{т.е.} мы получаем следующее: $(\mathsf{S}_n)$ сходится, если и только если для любого $\varepsilon >0$ существует такой номер $N$, что при всех $n,m \ge N$, $|\mathsf{S}_m- \mathsf{S}_n| < \varepsilon$.
  
  Без ограничения общности, мы можем положить, что $m>n$, \textit{т.е.} $m= n+p$, где $p \ge 1$, в результате получаем следующее: последовательность $(\mathsf{S}_n)$ сходится, если и только если для любого $\varepsilon >0$ существует такой номер $N$, что для любого $n \ge N$, $p \ge 1$ имеет место
  \[
   |\mathsf{S}_{n+p} - \mathsf{S}_n| = |x_{n+1} + \cdots+  x_{n+p}| < \varepsilon,
  \]
  что и требовалось доказать.
\end{proof}


\begin{corollary}[Необходимое условие сходимости ряда]\label{nessesary_for_series}
    Если ряд $(x_n)$ сходится, то 
    \[
     \lim_{n \to \infty} x_n =0.
    \]
    \end{corollary}
\begin{proof}
  Достаточно воспользоваться критерием Коши \ref{Coshy2} в случае, когда $p=1$, мы тогда получим, что для любого $\varepsilon >0$ существует такой номер $N$, что при всех $m \ge N$, имеет место неравенство $|\mathsf{S}_{m+1} - \mathsf{S}_m| < \varepsilon$, но это означает (см. Определение \ref{limit_of_seqeunce}), что $\lim_{m \to \infty} (\mathsf{S}_{m+1} - \mathsf{S}_m)=0$. С другой стороны, 
  \begin{eqnarray*}
      \mathsf{S}_{m+1} -\mathsf{S}_m &=& x_1 + \cdots x_m + x_{m+1} \\
      && - x_1 - \cdots - x_m \\
      &=& x_{m+1},
  \end{eqnarray*}
  поэтому из сходимости ряда $(x_n)$ следует, что $\lim_{m\to \infty }x_{m+1} = 0$, теперь, полагая $m = n-1$ и принимая во внимание соглашение $\mathsf{S}_0 :=0$, мы завершаем доказательство.
\end{proof}

\begin{mydanger}{\bf{!}}
    Нужно учесть, что это необходимое условие ни в коем случае не является достаточным. Например, гармонический ряд этому условию удовлетворяет, но он не сходится.
\end{mydanger}

\begin{proposition}\label{ariph_for_series}
    Если ряды $(x_n)$ и $(x_n')$ сходятся и имеют суммы $s$ и $s'$, соответственно, то ряд $(x_n+x_n')$ сходится к сумме $s+s'$, а ряд $(\lambda x_n)$ для любого $\lambda \in \mathbb{R}$ -- к сумме $\lambda s.$
\end{proposition}
\begin{proof}~

(1) Последовательность частичных сумм ряда $(x_n + x_n')$ имеет вид 
    \[
     (x_1 + \cdots + x_n + x_1' + \cdots +x_n')
    \]
\textit{т.е.} $(\mathsf{S}_n + \mathsf{S}_n')$, но тогда по арифметике предела для последовательнстей (Теорема \ref{a+b,ca,ab}) получаем $\lim_{n \to \infty} (\mathsf{S}_n + \mathsf{S}_n') = \mathsf{S}+ \mathsf{S}'$, что доказывает первое утверждение.

(2) Последовательность частичных сумм для ряда $(\lambda x_n)$ имеет вид $(\lambda x_1 + \cdots + \lambda x_n)$, \textit{т.е.} $(\lambda \mathsf{S}_n)$, опять воспользовавшись арифметикой предела для последовательностей (Теорема \ref{a+b,ca,ab}), мы завершаем доказательство.
\end{proof}

Напомним (см. \ref{almost_all}), что фраза \textit{почти для всех} означает, что для всех за исключением конечного числа.

\begin{definition}
    Будем говорить, что ряд $(y_n)$ почти такой же (или почти похож) на ряд $(x_n)$ если $y_n = x_n$ почти для всех $n$, \textit{т.е.} существует конечное множество $n_1,\ldots, n_\ell$, таких, что $x_{n_1} \ne y_{n_1},\ldots, x_{n_\ell} \ne y_{n_\ell}$, но $x_n = y_n$ для всех остальных $n.$
\end{definition}

\begin{lemma}\label{almost_for_series}
    Если $(x_n)$ и $(x_n')$ почти похожие ряды, то оба они сходятся или расходятся.
\end{lemma}
\begin{proof}
    Рассмотрим ряд $(x_n''): = (x_n - x_n')$, тогда почти все его элементы равны нулю, а это значит, что он сходится, \textit{т.е.} мы имеем $ \lim_{n \to \infty} \mathsf{S}''_n = \mathsf{S}''$. 

(1) Пусть ряд $(x_n')$ сходится, и пусть $\lim_{n \to \infty}\mathsf{S}_n' =\mathsf{S}'$, тогда согласно Предложению \ref{ariph_for_series} ряд $(x_n'' + x_n')$ тоже сходится к сумме $\mathsf{S}''+\mathsf{S}'$, но $x_n''+x_n' = x_n$, \textit{т.е.} ряд $(x_n)$ сходится.

(2) Пусть теперь ряд $(x_n')$ расходится, а ряд $(x_n)$ сходится. Опять рассмотрим ряд $(x_n''): = (x_n - x_n')$, у которого почти все элементы нулевые, а значит, он сходится, и мы опять положим $\lim_{n \to \infty}\mathsf{S}_n'' = \mathsf{S}''.$ Рассмотрим ряд $(x_n - x_n'')$, по предложению \ref{ariph_for_series} получаем, что этот ряд сходится, но $x_n - x_n'' = x_n'$, и мы тем самым пришли к тому, что ряд $(x_n')$ сходится, что противоречит предположению, следовательно, ряд $(x_n)$ не может быть сходящимся, \textit{т.е.} из расходимости ряда $(x_n')$ следует расходимость ряда $(x_n).$
\end{proof}


\section{Лекция \#8. Положительные ряды}

Перейдём теперь к рядам с положительными элементами. Будем говорить, что ряд $(x_n)$ \textit{положительный}, если все $x_n >0$.

Специфика таких рядов проявляется сразу.

\begin{theorem}[Критерий сходимости положительного ряда]\label{criteria_for_positive_series}
    Положительный ряд $(x_n)$ сходится тогда и только тогда, когда последовательность $(\mathsf{S}_n)$ его частичных сумм ограничена.
\end{theorem}
    \begin{proof}
    Действительно, в таком случае последовательность $(\mathsf{S}_n)$ его частичных сумм строго возрастает, тогда если последовательность $(\mathsf{S}_n)$ ограничена, то по теореме Вейерштрасса \ref{Weierstrass} она имеет предел, \textit{т.е.} ряд сходится. С другой стороны, пусть ряд сходится, тогда $\lim_{n\to \infty} \mathsf{S}_n =\mathsf{S}$, \textit{т.е.} для любого $\varepsilon >0$ найдётся такой номер $N$, что при всех $n \ge N$, $\mathsf{S}-\varepsilon < \mathsf{S}_n < \mathsf{S}+ \varepsilon$, но $(\mathsf{S}_n)$ -- возрастающая, значит, все $\mathsf{S}_n < \mathsf{S} + \varepsilon$, \textit{т.е.} последовательность $(\mathsf{S}_n)$ ограничена.
\end{proof}

\subsection{Признаки сравнения}

Из этого критерия вытекают многочисленные следствия.

\begin{corollary}[Признак сравнения 1]\label{cor1_for_series}
    Пусть $(x_n)$, $(x_n')$ -- два положительных ряда, при этом $x_n \le x_n'$ почти для всех $n$. Если ряд $(x_n')$ сходится, то сходится и ряд $(x_n)$. Если же ряд $(x_n)$ расходится, то расходится и ряд $(x_n').$
\end{corollary}
\begin{proof} 

 Если неравенства $x_n\le x_n'$ не выполнены для каких-то конечных значений $n$, скажем, $n = n_1,\ldots, n_\ell$, то рассмотрим ряды $(y_n)$, $(y_n')$, определённые следующим образом 
\[
 y_n = \begin{cases}
     x_n, & n \ne n_1,\ldots, n_\ell,\\
     0, & n = n_1,\ldots, n_\ell,
 \end{cases} \qquad  y'_n = \begin{cases}
     x'_n, & n \ne n_1,\ldots, n_\ell,\\
     0, & n = n_1,\ldots, n_\ell
 \end{cases} 
\]
которые почти похожи на ряды $(x_n)$ и $(x_n')$, соответственно. Согласно лемме \ref{almost_for_series} ряды $(y_n)$, $(y_n')$ имеют тот же характер сходимости, как и ряды $(x_n)$, $(x_n')$, соответственно. Поэтому исследование характера сходимости рядов $(x_n)$, $(x_n')$ сводится к исследованию характера рядов $(y_n)$, $(y_n')$. Это означает, что мы без ограничения общности можем считать, что неравенства $x_n \le x_n'$ выполняются для всех $n\ge 1.$

(1) Пусть ряд $(x_n')$ сходится, тогда по Теореме \ref{criteria_for_positive_series} последовательность его частичных сумм $(\mathsf{S}_n')$ ограничена, скажем, числом $\alpha$, \textit{т.е.} $\mathsf{S}'_n < \alpha$ для всех $n.$ С другой стороны, по условию $x_n \le x_n'$, тогда в силу положительности рядов
    \[
     \mathsf{S}_n = x_1 + \cdots + x_n \le x_1' + \cdots + x_n' = \mathsf{S}_n' < \alpha
    \]
    для всех $n$, \textit{т.е.} последовательность частичных сумм ряда $(x_n)$ ограничена, тогда по теореме \ref{criteria_for_positive_series} ряд $(x_n)$ сходится.

(2) Пусть ряд $(x_n)$ расходится, тогда по теореме \ref{criteria_for_positive_series} последовательность $(\mathsf{S}_n)$ неограниченна. Так как последовательность $(\mathsf{S}_n)$ возрастает, то неограниченность означает, что для любого числа $M \in \mathbb{N}$ найдётся такой номер $n\in \mathbb{N}$, что все $\mathsf{S}_n, \mathsf{S}_{n+1}, \ldots > M$. С другой стороны, как мы уже видели, $\mathsf{S}_n' \ge \mathsf{S}_n$, таким образом, для всех $m \ge n$ получаем $s'_m \ge \mathsf{S}_m > M$, \textit{т.е.} последовательность $(\mathsf{S}_n')$ неограниченна, а тогда по теореме \ref{criteria_for_positive_series} ряд $(x_n')$ расходится.
\end{proof}

\begin{corollary}[Признак сравнения 2]\label{2_for_series}
    Пусть ряды $(x_n)$, $(y_n)$ положительные, и начиная с некоторого $N$, выполняются неравенства 
    \[
     \frac{y_{n+1}}{y_n} \le \frac{x_{n+1}}{x_n}.
    \]
Тогда если ряд $(x_n)$ сходится, то ряд $(y_n)$ тоже сходится. Если же ряд $(y_n)$ расходится, то расходится и ряд $(x_n).$
\end{corollary}
\begin{proof}
    Пусть $\lambda = \frac{x_N}{y_N}$, тогда согласно Предложению \ref{ariph_for_series}, если ряд $(y_n)$ сходится, то сходится и ряд $(\lambda y_n)$, но $N$-ый элемент ряда $(\lambda y_n)$ есть $\lambda y_N = \frac{x_N}{y_N}y_N = x_N$. Поэтому без ограничения общности мы можем считать, что $x_N = y_N.$ Тогда из неравенства $\frac{y_{N+1}}{y_N} \le \frac{x_{N+1}}{x_N}$ следует, что $y_{N+1} \le x_{N+1}$.

    Далее, из неравенства $\frac{y_{N+2}}{y_{N+1}} \le \frac{x_{N+2}}{x_{N+1}}$ получаем $x_{N+1}y_{N+2} \le y_{N+1} x_{N+2}$, но $y_{N+1} \le x_{N+1}$ и в силу положительности всех элементов получаем 
    \[
     x_{N+1}y_{N+2} \le y_{N+1} x_{N+2} \le x_{N+1}x_{N+2}
    \]
    тогда $y_{N+2} \le x_{N+2}$, и продолжая, мы по индукции получаем, что $y_n \le x_n$ для всех $n \le N$. Теперь, воспользовавшись признаком сравнения 1 (Следствие \ref{cor1_for_series}), мы завершаем доказательство.
\end{proof}

\begin{corollary}\label{cor_for_similar}
  Если $\lambda >0$, то ряды $(x_n)$, $(\lambda x_n)$ имеют одинаковый характер сходимости.
\end{corollary}
\begin{proof}
    Пусть $y_n : = \lambda x_n$, тогда, по условию $y_n \ne 0$, и более того
    \[
     \frac{y_{n+1}}{y_n} = \frac{\lambda x_{n+1}}{\lambda x_n} = \frac{x_{n+1}}{x_n}
    \]
    тогда воспользовавшись Следствием \ref{2_for_series} мы завершаем доказательство. 
\end{proof}

\begin{corollary}[Предельный признак сравнения]\label{critical_for_series}
    Пусть $(x_n)$, $(x_n')$ два положительных ряда, тогда если существует конечный предел,
    \[
     \lim_{n \to \infty} \frac{x_n}{x_n'} = q, 
    \]
    то при $0 \le q < \infty$ из сходимости ряда $(x_n')$ следует сходимость ряда $(x_n)$, а при $0 < q \le \infty$ из расходимости ряда $(x_n')$ следует расходимость ряда $(x_n).$
\end{corollary}

\begin{mydangerr}{\bf !}
    Таким образом, при $0<q < +\infty$ оба ряда сходятся или расходятся одновременно.
\end{mydangerr}

\begin{proof}
Пусть $\lim_{n \to \infty} \frac{x_n}{x_n'} = q$, согласно определению предела последовательности \ref{limit_of_seqeunce}, для любого $\varepsilon >0$ найдётся такой номер $N$, что верны неравенства
    \[
     q-\varepsilon < \frac{x_n}{x_n'} < q+ \varepsilon, \qquad n \ge N.
    \]

Таким образом, почти для всех $n\ge 1$ имеем
\[
(q-\varepsilon)x_n' < x_n < (q+\varepsilon)x_n'.
\]

(1) Если ряд $(x_n')$ сходится, то согласно Предложению \ref{ariph_for_series}, ряд $(q+\varepsilon)x_n'$ тоже сходится, а так как 
$x_n < (q+\varepsilon)x_n'$, тогда по признаку сравнения 1 (Следствие \ref{cor1_for_series}) ряд $(x_n)$ тоже сходится.

(2) Пусть теперь $q>0$ и ряд $(x_n')$ расходится. Тогда обратное отношение $\frac{x_n'}{x_n}$, согласно арифметике пределов, имеет конечный предел. Это значит, что по доказанному выше, ряд $(x_n)$ должен расходится, ибо в противном случае будет сходится ряд $(x_n')$ что противоречт предположению.


\end{proof}

\subsection{Варианты Даламбера и Коши}


\begin{corollary}[Признаки сравнения Даламбера]\label{Dalamber_criteria_for_series}
    Пусть дан положительный ряд $(x_n)$, 
    \begin{enumerate}
        \item Если для почти всех $n \ge 0$
        \[
         \frac{x_{n+1}}{x_n} \le q <1,
        \]
        то ряд $(x_n)$ сходится; если же для почти всех $n \ge 0$
        \[
         \frac{x_{n+1}}{x_n} \ge 1,
        \]
        то ряд $(x_n)$ расходится.

        \item Если 
        \[
         \lim_{n \to \infty} \frac{x_{n+1}}{x_n} = q
        \]
        то ряд $(x_n)$ при $q<1$ сходится, а при $1 < q \le \infty$ расходится.
    \end{enumerate}
\end{corollary}

\begin{proof} Мы воспользуемся леммой \ref{almost_for_series} в случае необходимости и тогда можем считать, что неравенства выполнены для всех $n \ge 0.$

    (1) Имеем для каждого $n \ge 2$
    \[
     x_n = x_1 \cdot \frac{x_2}{x_1}\cdot \frac{x_3}{x_2} \cdots \frac{x_{n-1}}{x_{n-2}}\cdot \frac{x_n}{x_{n-1}},
    \]
    по условию
    \[
     \frac{x_2}{x_1}, \frac{x_3}{x_2}, \ldots, \frac{x_n}{x_{n-1}} \le q < 1,
    \]
    тогда
    \[
     x_n = x_1 \cdot \frac{x_2}{x_1}\cdot \frac{x_3}{x_2} \cdots \frac{x_{n-1}}{x_{n-2}}\cdot \frac{x_n}{x_{n-1}} \le x_1 q^{n-1}.
    \]

С другой стороны, (см. пример \ref{geometric_series_and_ML_series}) ряд $(q^n)$ сходится при $q<1$, а тогда по предложению \ref{ariph_for_series} ряд $(x_1q^n)$ тоже сходится. Наконец, по признаку 1 (Следствие \ref{cor1_for_series}) ряд $(x_n)$ сходится.

(2) Если же 
\[
     \frac{x_2}{x_1}, \frac{x_3}{x_2}, \ldots, \frac{x_n}{x_{n-1}} > 1,
    \]
    то
    \[
     x_n \ge x_1, \qquad n \ge 2.
    \]

Ряд $(y_n)$, где все $y_n = x_1$, очевидно, расходится, тогда по признаку 1 (Следствие \ref{cor1_for_series}), ряд $(x_n)$ тоже расходится.

(3) Пусть $\lim_{n \to \infty} \frac{x_{n+1}}{x_n}= q$, тогда по определению \ref{limit_of_seqeunce} для любого $\varepsilon >0$ существует такой $N$, что для всех $n \ge N$ имеют место неравенства
\[
 q - \varepsilon < \frac{x_{n+1}}{x_n} < q + \varepsilon.
\]

Если $q<1$, то пусть $q+\varepsilon <1$, тогда для всех $n \ge N$, $\frac{x_{n+1}}{x_n} \le 1$, тогда по доказанному признаку (1) ряд $(x_n)$ сходится.

Если $q>1$, то возьмём $\varepsilon >0$ такое, что $q-\varepsilon >1$. Но
\[
 \frac{x_{n+1}}{x_n} > q - \varepsilon
\]
при каких-то $n \ge N$, поэтому для $N\le n_0 < n$ получаем
\[
 x_n=  \frac{x_n}{x_{n-1}} \cdot \frac{x_{n-1}}{x_{n-2}} \cdots \frac{x_{n_0 +1}}{x_{n_0}} x_{n_0} > (q-\varepsilon)^{n-n_0} x_{n_0},
\]
а так как $q - \varepsilon >1$ и (см. Пример \ref{geometric_series_and_ML_series}) ряд $(q-\varepsilon)^{n}$ расходится, то по теореме \ref{ariph_for_series} получаем, что ряд $(x_n)$ расходится.

Тем самым признаки Даламбера полностью доказаны.
\end{proof}


\begin{theorem}[Радикальный признак Коши]
    Пусть $(x_n)$ --положительный ряд. 
    \begin{enumerate}
        \item Если для почти всех $n \ge 1$
        \[
         \sqrt[n]{x_n} < q < 1,
        \]
        то ряд $(x_n)$ сходится.
        
        \item Если для почти всех $n \ge 1$
        \[
         \sqrt[n]{x_n} \ge 1
        \]
        то ряд $(x_n)$ расходится.

        \item Если
        \[
         \lim_{n\to \infty } \sqrt[n]{x_n} = q,
        \]
        то при $q<1$ ряд $(x_n)$ сходится, а при $q>1$ расходится, и при этом $\lim_{n \to \infty} x_n = \infty.$
    \end{enumerate}
\end{theorem}

\begin{proof}
    Пользуясь леммой \ref{almost_for_series}, мы можем считать, что неравенства выполнены для всех $n \ge 1.$

    (1) Если $\sqrt[n]{x_n}<q$, то $x_n <q^n$, а так как $q<1$, то согласно примеру \ref{geometric_series_and_ML_series} и признаку сравнения (см. Следствие \ref{cor1_for_series}) получаем, что ряд $(x_n)$ сходится. 

    (2) Если $\sqrt[n]{x_n}\ge 1$, то $x_n \ge 1$, но ряд $(n)$ расходится, а тогда согласно признаку сравнения (см. Следствие \ref{cor1_for_series}) ряд $(x_n)$ расходится.

    (3) Пусть $q<1$. Возьмём такой $\varepsilon>0$, чтобы $q<q+\varepsilon <1$. Тогда по определению предела \ref{limit_of_seqeunce} найдётся такой $N$, что при всех $n \ge N$ мы получаем
    \[
     \sqrt[n]{x_n} < q+\varepsilon <1,
    \]
    тогда $x_n < (q+\varepsilon)^n$ почти для всех $n$, а так как (см. Пример \ref{geometric_series_and_ML_series}) ряд $((q+\varepsilon)^n)$ сходится при $q+\varepsilon<1$, то согласно признаку сравнения (см. Следствие \ref{cor1_for_series}) ряд $(x_n)$ сходится.

    Пусть теперь $q>1$, то выберем такое $\varepsilon>0$, чтобы $q-\varepsilon >1$, тогда получаем, что $\sqrt[n]{x_n} > q-\varepsilon >1$, начиная с какого-то $n$, \textit{т.е.} $x_n > (q-\varepsilon)^n$ почти для всех $n\ge 1$, но (см. Пример \ref{geometric_series_and_ML_series}) ряд $((q-\varepsilon)^n)$ расходится при $q-\varepsilon>1$, то согласно признаку сравнения (см. Следствие \ref{cor1_for_series}) ряд $(x_n)$ расходится.

    Тем самым радикальный признак Коши полностью доказан.
\end{proof}

\begin{remark}
    Для положительного ряда $(x_n)$ выражения 
    \[
     \mathscr{D}_n: = \frac{x_{n+1}}{x_n}, \qquad \mathscr{C}_n: = \sqrt[n]{x_n},
    \]
    называют \textit{вариантой Даламбера} и \textit{вариантой Коши}, соответственно.
\end{remark}

\begin{mydanger}{\bf{!}}
 В предыдущих двух признаках (Даламбера и Коши) не рассматривался случай, когда обе варианты равны $1$, оказывается, в этом случае эти признаки не работают! Сейчас мы приведём пример.
\end{mydanger}

\begin{example}
    \textit{Рядом Дирихле} называют ряд вида $\left( \frac{1}{n^\alpha} \right)$, где $\alpha \in \mathbb{R}_+$. Позже мы докажем, что этот ряд сходится при $\alpha >1$ и расходится при $\alpha \le 1$. При этом, в обоих случаях (когда $\alpha >1$ или когда $\alpha <1$) имеем
    \[
     \lim_{n\to \infty} \mathscr{D}_n = \lim_{n \to \infty} \frac{x_{n+1}}{x_n} = \lim_{n\to \infty} \left(\frac{n}{n+1} \right)^\alpha  = \left( \lim_{n \to \infty} \frac{n}{n+1} \right)^\alpha =1,
    \]
    так же, как (см. Пример \ref{sqrt[n]{n}->1})
    \[
    \lim_{n\to \infty} \mathscr{C}_n =  \lim_{n\to \infty} \sqrt[n]{x_n} = \lim_{n \to \infty}\sqrt[n]{\frac{1}{n^\alpha}} = \left( \frac{1}{\lim\limits_{n\to \infty}\sqrt[n]{n}}\right)^\alpha =1.
    \]

    Таким образом, существуют как сходящиеся, так и расходящиеся ряды, для которых верны равенства $\lim_{n\to \infty}\mathscr{D}_n= 1$, $\lim_{n \to \infty}\mathscr{C}_n = 1.$
\end{example}

\subsection{Инвариантность суммы}

Докажем инвариантность суммы сходящегося положительного ряда при произвольной перестановки его элементов.

\begin{theorem}[ ]\label{comm_for_positive_series}
    Пусть $(x_n)$ -- сходящийся положительный ряд с суммой $\mathsf{S}$. Тогда полученный в результате произвольной перестановки его элементов новый (заново перенумерованный) ряд также сходится и имеет ту же сумму $\mathsf{S}.$
\end{theorem}

\begin{proof}
 Пусть $x_1' = x_{n_1}, \ldots, x_k' = x_{n_k}, \ldots,$ и пусть $n: = \max \{n_1,\ldots, n_k\}$, рассмотрим тогда частичные суммы
  \[
  \mathsf{S}_n: = x_1 + \cdots + x_n, \qquad \mathsf{S}'_k: = x_1' + \cdots + x_k',
 \]
так как $1\le n_1, \ldots, n_k \le n$ и $(x_n)$ -- положительный ряд, то
\[
 \mathsf{S}_k' \le \mathsf{S}_n.
\]

Но, положительный ряд $(x_n)$ сходится, а тогда по критерию сходимости положительного ряда (см. Теорема \ref{criteria_for_positive_series}), последовательность $(\mathsf{S}_n)$ ограничена, и более того $\mathsf{S}_n \le \mathsf{S}$ для всех $n$. Таким образом, для всех $k$ получаем
\[
 \mathsf{S}_k' \le \mathsf{S}_n \le \mathsf{S},
\]
\textit{т.е.,} последовательность $(\mathsf{S}_k')$ частичных сумм ряда $(x_k')$ ограничена, а тогда по критерию сходимости положительного ряда (cм. Теорема \ref{criteria_for_positive_series}), ряд $(x_k')$ -- сходится, \textit{т.е.,} существует предел $\lim_{k \to \infty} \mathsf{S}_k' = \mathsf{S}'$. Тогда по Лемме \ref{a<b}, $\mathsf{S}' \le \mathsf{S}.$

Рассмотрим теперь ряд $(x_k')$, тогда на ряд $(x_n)$ можно посмотреть как на ряд который получился из ряда $(x_k')$ в результате какой-то перестановки элементов $x_k'$. Тогда, рассуждая аналогичным образом, мы приходим к выводу, что 
$\mathsf{S}_m \le \mathsf{S}_k'$, и по лемме \ref{a<b}, получаем $\mathsf{S} \le \mathsf{S}'$.

Наконец, из полученных неравенств $\mathsf{S}' \le \mathsf{S}$, $\mathsf{S} \le \mathsf{S}'$ вытекает, что $\mathsf{S} = \mathsf{S}'$. Это завершает доказательство теоремы.
\end{proof}

\section{Лекция \#9. Знакочередующиеся ряды и условно сходящиеся ряды}

Когда мы познакомимся с дифференциальным исчислением, то мы узнаем как приближать известные функции полиномами и даже рядами. Например, если мы хотим вычислить, например, значение $\cos(1)$ с определённой точностью, то мы можем воспользоваться полиномом Тейлора для функции $\cos(x)$ в точке $x=1$. Мы получаем следующий ряд:
\[
 \left(1, - \frac{1}{2!}, \frac{1}{4!}, - \frac{1}{6!}, \ldots, \frac{(-1)^n}{(2n)!}, \ldots, \right) 
\]
и его частичная сумма, скажем, $\mathsf{S}_n$ и есть значение полинома Тейлора функции $\cos(x)$ при $x=1$;
\[
 1 - \frac{x^2}{2!} + \frac{x^4}{4!} - \frac{x^6}{6!} + \ldots + \frac{(-1)^n}{(2n)!} 
\]

\subsection{Знакочередующиеся ряды и признак Лейбница}

\begin{definition}
    Знакочередующийся ряд -- это ряд $(a_n)$, элементы которого попеременно принимают значения противоположных знаков, \textit{т.е.} если $a_n>0$ (\textit{соответственно}, $a_n <0$), то $a_{n+1}<0$ (\textit{соответственно}, $a_{n+1}>0$). Элементы таких рядов можно записать либо как $a_n  = (-1)^n |a_n|$, либо как $a_n = (-1)^{n+1}|a_n|$.
\end{definition}



\begin{theorem}[Признак Лейбница]\label{Leibnitz_for_series}
   Пусть $(a_n)$ -- знакочередующийся ряд, для которого выполняются следующие условия:
   \begin{enumerate}
       \item $|a_n| \ge |a_{n+1}|$ почти для всех $n$,
       \item $\lim\limits_{n \to \infty} |a_n| = 0$.
   \end{enumerate}
   Тогда ряд $(a_n)$ сходится.
\end{theorem}

\begin{proof}
Воспользовавшись леммой \ref{almost_for_series}, мы можем считать, что $|a_n| \ge |a_{n+1}|$ для всех $n$. Для удобства положим, что первый элемент ряда -- это $a_0$, \textit{т.е.} $n \ge 0$. Рассмотрим частичную сумму $\mathsf{S}_{2n+1}$, имеем
\begin{eqnarray*}
    \mathsf{S}_{2n+1} &=& |a_0| - |a_1| + |a_2| - |a_3| + |a_4| + \cdots  + |a_{2n}| - |a_{2n+1}| \\
    &=& |a_0| - \bigl(|a_1| - |a_2|\bigr) - \bigl(|a_3| - |a_4|\bigr) - \cdots - \bigl(|a_{2n-1}|-  |a_{2n}|\bigr) - |a_{2n+1}|,
\end{eqnarray*}
так как $|a_n| \ge |a_{n+1}|$, то каждая скобка положительна, это значит, что $\mathsf{S}_{2n+1} \le |a_0|$, \textit{т.е.} последовательность $(\mathsf{S}_{2n+1})$ ограничена сверху.

С другой стороны, мы можем записать
\begin{eqnarray*}
    \mathsf{S}_{2n+1} &=& |a_0| - |a_1| + |a_2| - |a_3| + \cdots  + |a_{2n-2}| - |a_{2n-1}| + |a_{2n}| - |a_{2n+1}| \\
    &=& \bigl(|a_0| - |a_1| \bigr) + \bigl(|a_2|-|a_3| \bigr) + \cdots + \bigl( |a_{2n-2}| - |a_{2n-1}|\bigr)+ \bigl(|a_{2n}| - |a_{2n+1}|\bigr) \\
    &=& \mathsf{S}_{2n-1} + \bigl(|a_{2n}| - |a_{2n+1}|\bigr),
\end{eqnarray*}
и так как $|a_{2n}| \ge |a_{2n+1}|$, то $\mathsf{S}_{2n+1} \ge \mathsf{S}_{2n-1}$, \textit{т.е.} она не убывает.

Итак, последовательность $(\mathsf{S}_{2n+1})$ ограничена сверху и не убывает, тогда по теореме Вейерштрасса \ref{Weierstrass} у неё есть предел $\lim\limits_{n \to \infty}\mathsf{S}_{2n+1} = \mathsf{S} \le |a_0|.$

Наконец, мы также можем записать
\begin{eqnarray*}
    \mathsf{S}_{2n+1} &=& |a_0| - |a_1| + |a_2| - |a_3| +  |a_{2n}| - |a_{2n+1}| \\
    &=& \mathsf{S}_{2n}  - |a_{2n+1}|,
\end{eqnarray*}
так как $\lim\limits_{n \to \infty}\mathsf{S}_{2n+1} = \mathsf{S}$ и по условию $\lim\limits_{n\to \infty} |a_{2n+1}| = 0$, то по теореме \ref{a+b,ca,ab}
\[
 \lim_{n\to \infty}\mathsf{S}_{2n}  = \lim_{n\to \infty} \left( \mathsf{S}_{2n+1} + |a_{2n+1}| \right) = \mathsf{S} + 0 = \mathsf{S}.
\]

Итак, мы показали, что $\lim\limits_{n \to \infty}\mathsf{S}_n = \mathsf{S}$, что и означает сходимость ряда.
\end{proof}

\begin{example}
    Ряд $(x_n)$, где $x_n = \frac{(-1)^n}{n}$, очевидно удовлетворяет признаку Лейбинца (Теорема \ref{Leibnitz_for_series}) и значит, сходится, более того, мы покажем, что 
    \[
     \lim_{n \to \infty} \sum_{k=1}^n \frac{(-1)^n}{n} = \ln(2).
    \]
\end{example}


\subsection{Абсолютно, условно и безусловно сходящиеся ряды}

Мы видели, что в отношении положительных рядов (а значит, и для отрицательных тоже), характер сходимости, по большей части, устанавливается нетрудно, благодаря наличию ряда удобных признаков.

\begin{mydanger}{\bf{!}}
 С другой же стороны, если ряд почти положительный, то у него в силу Леммы \ref{almost_for_series} такой же характер сходимости, что и у ряда, полученного из данного обнулением этих неположительных чисел.    
\end{mydanger}

Поэтому существенно новым случаем будет тот, когда среди элементов ряда есть бесконечное число отрицательных элементов. Мы уже познакомились со случаем, когда знаки элементов чередуются. Тут же мы займёмся более общим вопросом. 

Начнём с простого, но очень важного наблюдения.

\begin{theorem}\label{abs=ok}
    Пусть дан ряд $(x_n)$. Если сходится ряд $(|x_n|)$, то ряд $(x_n)$ тоже сходится.
\end{theorem}

\begin{proof}
Последовательность частичных сумм ряда $(x_n)$ мы, как обычно, обозначим через $(\mathsf{S}_n)$, а последовательность частичных суммы для ряда $(|x_n|)$ мы обозначим через $(\mathsf{A}_n)$.

Так как ряд $(|x_n|)$ сходится, то по критерию Коши \ref{Coshy2} мы для каждого $\varepsilon >0$ можем найти такое $N$, что для всех $n \ge N$ и $p \ge 1$,  
\[
 | \mathsf{A}_{n+p} - \mathsf{A}_{n} | = |x_{n+1} | + \cdots +|x_{n+p}| < \varepsilon.
\]

Далее, имеем 

\begin{eqnarray*}
    \left|\mathsf{S}_{n+p} - \mathsf{S}_n \right| &=& | x_{n+1} + \cdots + x_{n+p} | \\
    &\le & |x_{n+1}| + \cdots + |x_{n+p}| \\
    &<& \varepsilon,
\end{eqnarray*}
что согласно критерию Коши \ref{Coshy2} означает сходимость ряда $(x_n).$ 
\end{proof}

\begin{remark}
    Итак, из предыдущей теоремы (теорема \ref{abs=ok})) вытекает, что для исследования ряда $(x_n)$ на сходимость достаточно исследовать на сходимость положительный ряд $(|x_n|)$. Таким образом, \textbf{что все те признаки сравнения, которые мы знаем для положительных рядов, применимы и в этом случае} (для ряда из абсолютных величин его элементов). 
\end{remark}

\begin{mydanger}{\bf{!}}
    Нужно быть осторожным с признаками \textbf{расходимости}; если ряд $(|x_n|)$ окажется расходящимся, то ряд $(x_n)$ может быть сходящимся.
\end{mydanger}

\begin{example}
    Как мы уже говорили, ряд $(x_n)$, где $x_n = \frac{(-1)^n}{n}$, сходится, так как выполнятся признак Лейбница (Теорема \ref{Leibnitz_for_series}), однако ряд $(|x_n|) = (\frac{1}{n})$ -- гармонический ряд, который расходится, поэтому ряд $(x_n)$ сходится условно.
\end{example}


Таким образом, оправдано введение следующего понятия
\begin{definition}
    Ряд $(x_n)$ называется \textit{абсолютно сходящимся}, если ряд $(|x_n|)$ сходится. Если ряд $(x_n)$ сходится, а ряд $(|x_n|)$ расходится, то говорят, что \textit{ряд $(x_n)$ сходится условно.}
\end{definition}

\begin{construction}\label{x_n^+}
 Пусть дан ряд $(x_n)$, положим
\[
 x_n^+: = \begin{cases}
     x_n, & \mbox{если $x_n \ge 0$},\\
     0, & \mbox{если $x_n <0$},
 \end{cases}
 \qquad
x_n^-: = \begin{cases}
     -x_n, & \mbox{если $x_n \le 0$},\\
     0, & \mbox{если $x_n >0$},
 \end{cases} 
\]
и будем рассматривать два ряда $(x_n^+)$, $(x_n^{-})$ которые, очевидно, положительны.     
\end{construction}


\begin{proposition}
 Пусть ряд $(x_n)$ сходится абсолютно, тогда ряды $(x_n^+)$, $(x_n^{-})$ сходятся и более того, если $\mathsf{S}, \mathsf{S}^+, \mathsf{S}^-$ -- суммы рядов $(x_n)$, $(x_n^+)$, $(x_n^-)$ соответственно, то
 \[
  \mathsf{S} = \mathsf{S}^+ - \mathsf{S}^-.
 \]
\end{proposition}

\begin{proof}\label{S=S^+-S^-}
 Во-первых, из Леммы \ref{abs=ok} следует корректность утверждения, ибо ряд $(x_n)$ сходится, а тогда последовательность его частичных сумм $(\mathsf{S}_n)$ имеет предел $\mathsf{S}.$

Во-вторых, так как $x_n^+ \le |x_n|$ и $x_n^- \le |x_n|$, то по признаку сравнения \ref{cor1_for_series} ряды $(x_n^+)$, $(x_n^-)$ сходятся, а значит, сходятся последовательности их частичных сумм $(\mathsf{S}_n^+)$, $(\mathsf{S}_n^-)$ к $\mathsf{S}^+$, $\mathsf{S}^-$ соответственно.

Далее, так как из конструкции \ref{x_n^+} следует, что $x_n = x_n^+- x_n^-$, но тогда для каждого $n$ получаем
\begin{eqnarray*}
     \mathsf{S}_n  &=& x_1 + \cdots + x_n \\
     &=& x_1^+ - x_1^- + \cdots + x_n^+ - x_n^- \\
     &=& \left( x_1^+ + \cdots + x_n^+ \right) - \left(x_1^- + \cdots + x_n^- \right) \\
     &=& \mathsf{S}_n^+ - \mathsf{S}_n^-.
\end{eqnarray*}

Тогда по теореме \ref{a+b,ca,ab}, 
\[
 \mathsf{S} = \lim_{n \to \infty} \mathsf{S}_n = \lim_{n\to \infty}(\mathsf{S}_n^+ - \mathsf{S}_n^-) = \lim_{n\to \infty} \mathsf{S}_n^+ - \lim_{n\to \infty}\mathsf{S}_n^- = \mathsf{S}^+ - \mathsf{S}^-,
\]
что и требовалось доказать. 
\end{proof}

\begin{proposition}
    Если ряд абсолютно сходится, то при любой перестановке его элементов абсолютная сходимость полученного нового ряда не нарушается и более того, его сумма остаётся прежней.
\end{proposition}

\begin{proof}
    Пусть ряд $(x_n)$ сходится абсолютно, рассмотрим ряды $(x_n^+)$, $(x_n^-)$ (конструкция \ref{x_n^+}), очевидно, что $x_n = x_n^+ - x_n^-$ для всех $n.$ Так как ряд $(x_n)$ сходится абсолютно, то ввиду $x_n^+ \le |x_n|$, $x_n^- \le |x_n^-|$ и признака сравнения (теорема \ref{cor1_for_series}) ряды $(x_n^+)$, $(x_n^-)$ тоже сходятся.

    Пусть ряд, полученный после перестановки исходного ряда, имеет вид $(y_n)$, рассмотрим также ряды $(y_n^+)$, $(y_n^-)$ (Конструкция \ref{x_n^+}), тогда $y_n = y_n^+ - y_n^-$, и мы получаем

\begin{align*}
    & \mathsf{S}  = \mathsf{S}^+ - \mathsf{S}^- & \mbox{(по предложению \ref{S=S^+-S^-})} \\
    & \phantom{\mathsf{S}} = \sum_{n=1}^\infty x_n^+ - \sum_{n=1}^\infty x_n^- \\
    & \phantom{\mathsf{S}} = \sum_{n=1}^\infty y_n^+ - \sum_{n=1}^\infty y_n^- & \mbox{(по теореме \ref{comm_for_positive_series})} \\
    & \phantom{\mathsf{S}} =  \sum_{n=1}^\infty (y_n^+ - y_n^-) & \mbox{(по предложению \ref{ariph_for_series})} \\
    & \phantom{\mathsf{S}} = \sum_{n=1}^\infty y_n.
\end{align*}

Что и требовалось доказать.
\end{proof}

Итак, мы показали, что в абсолютно сходящемся ряду совершенно не важен порядок его элементов, \textit{т.е.} любая перестановка его элементов не нарушает его сходимость и более того, его сумма остаётся прежней. Но будет ли такой же эффект, если ряд не сходится абсолютно? Как показал Б. Риман, это неверно. Поэтому оправданна следующая терминология.

\begin{definition}
    Говорят, что ряд сходится \textit{безусловно}, если он сходится и любая перестановка его элементов не нарушает его сходимости. 
\end{definition}

\begin{mydanger}{\bf{!}}
    В связи с этим возникает желание назвать ряд не безусловно сходящимся, если он сходится, но можно так поменять местами его элементы, что в результате получится уже не сходящийся ряд. Но теорема Римана (см. ниже) как раз и покажет, что это в точности условно сходящиеся ряды, поэтому необходимости в новом термине нет.
\end{mydanger}

Прежде чем перейти к теореме Римана, докажем следующее утверждение.

\begin{proposition}
    Для того, чтобы ряд $(x_n)$ был абсолютно сходящимся, необходимо и достаточно, чтобы ряды $(x_n^+)$, $(x_n^-)$ были сходящимся.
\end{proposition}

\begin{proof}
(1) Действительно, если ряд $(x_n)$ сходится абсолютно, то, так как $x_n^+ \le |x_n|$ и $x_n^- \le |x_n|$, и по признаку сравнения \ref{cor1_for_series} ряды $(x_n^+)$, $(x_n^-)$ сходятся.

(2) Пусть теперь для ряда $(x_n)$ ряды $(x_n^+)$, $(x_n^-)$ сходятся. Но, так как (см. конструкция \ref{x_n^+}), $|x_n| = x_n^+ + x_n^-$, то для любого $n$, $\mathsf{A}_n = \mathsf{S}_n^+ + \mathsf{S}_n^-$, где $\mathsf{A}_n$, $\mathsf{S}_n^+$, $\mathsf{S}_n^-$ -- частичные суммы рядов $(|x_n|)$, $(x_n^+)$, $(x_n^-)$, соответственно. Откуда следует сходимость последовательности $(\mathsf{A}_n)$. Тем самым мы завершаем доказательство.
\end{proof}

\begin{proposition}\label{condtional_convergance_and_x_n^+}
    Если ряд $(x_n)$ сходится условно, (\textit{т.е.} ряд $(|x_n|)$ расходится), то оба ряда $(x_n^+)$, $(x_n^-)$ расходятся, при этом $\lim_{n\to \infty }x_n^+ = \lim_{n \to \infty}x_n^- = 0.$
\end{proposition}

\begin{proof}
    Пусть ряд $(x_n)$ сходится, но не абсолютно, \textit{т.е.} ряд $(|x_n|)$ расходится. Так как $|x_n| = x_n^+ + x_n^-$, то $\mathsf{A}_n = \mathsf{S}_n^+ + \mathsf{S}_n^-$, где $\mathsf{A}_n$, $\mathsf{S}_n^+$, $\mathsf{S}_n^-$ -- частичные суммы рядов $(|x_n|)$, $(x_n^+)$, $(x_n^-)$, соответственно. Тогда из расходимости последовательности $(\mathsf{A}_n)$ вытекает, что хотя бы одна из последовательностей $(\mathsf{S}_n^+)$, $(\mathsf{S}_n^-)$ расходится. Если они обе расходится, то предложение доказано.

    Пусть расходится последовательность $(\mathsf{S}_n^+)$. Так как $x_n = x_n^+ - x_n^-$ (см. конструкцию \ref{x_n^+}), то $\mathsf{S}_n = \mathsf{S}_n^+ - \mathsf{S}_n^-$.
    
    Тогда
    \[
     \mathsf{S}_n^- = \mathsf{S}_n^+ - \mathsf{S}_n,
    \]
но так как $\lim_{n \to \infty} \mathsf{S}_n = \mathsf{S}$, то согласно лемме \ref{lim-->bounded_for_sequence}, последовательность $(\mathsf{S}_n)$ -- ограничена, скажем, $L \le \mathsf{S}_n \le R$. С другой же стороны, так как $(x_n^+)$ -- положительный расходящийся ряд, то согласно Теореме \ref{criteria_for_positive_series}, последовательность $(\mathsf{S}_n^+)$ -- неограниченна. Это значит, что для любого числа $N$ можно найти такой номер $n$, что $\mathsf{S}_n^+ >N+L$, а тогда $\mathsf{S}_n^- > N+L-L =N$, \textit{т.е.} для любого $N$ мы нашли номер $n$ такой, что $\mathsf{S}_n^- >N$, это означает, что последовательность $(\mathsf{S}_n^-)$ неограниченна, а тогда по теореме \ref{criteria_for_positive_series}, ряд $(x_n^-)$ -- расходится. 

Аналогично рассматривается случай, когда расходится ряд $(x_n^-)$. 

Наконец, так как ряд $(x_n)$ сходится, то по необходимому признаку (см. следствие \ref{nessesary_for_series}), $\lim_{n \to \infty} x_n = 0$, а из того, что $(x_n^+)$, $(x_n^-)$ подпоследовательности в последовательности $(x_n)$, то из теоремы \ref{lim(sub)=lim} получаем, что $\lim_{n \to \infty} x_n^+ = \lim_{n \to \infty} x_n^- = 0.$ Это завершает доказательство.
\end{proof}

\subsection{Теорема Римана}

Теперь мы докажем замечательную теорему, которая показывает как сильно отличается условно сходящийся ряд от абсолютно сходящегося.

\begin{theorem}[Риман]
    Пусть ряд $(x_n)$ сходится условно, тогда для любого числа $\alpha \in \mathbb{R}$, а также если $\alpha = \pm \infty$ можно так переставить элементы этого ряда, что сумма полученного таким образом ряда будет равна $\alpha.$
\end{theorem}

\begin{proof}
Для ряда $(x_n)$ мы рассмотрим ряды $(x_n^+)$, $(x_n^-)$ (см. конструкцию \ref{x_n^+}). Согласно предложению \ref{condtional_convergance_and_x_n^+}, ряды $(x_n^+)$, $(x_n^-)$ расходятся. Это значит, что последовательности их частичных сумм неограничены (см. теорема \ref{criteria_for_positive_series}), \textit{т.е.} их значения могут быть больше любого числа. 
    
Пусть $p_1$ -- наименьшее натуральное число (=номер последовательности $(x_n^+)$) такое, что
    \[
     \alpha < x_1^+ + \cdots + x_{p_1}^+  = \sum_{i=1}^{p_1}x_i^+,
    \]
далее, пусть $q_1$ -- наименьшее натуральное число (=номер последовательности $(x_n^-)$) такое, что
\[
 \alpha > \sum_{i=1}^{p_1} x_i^+ - \sum_{j=1}^{q_1} x_j^-.
\]

Пусть теперь $p_2$ есть наименьшее натуральное число (=номер последовательности $(x_n^+)$), большее, чем $p_1$, такое, что
\begin{eqnarray*}
  \alpha &<& \sum_{i=1}^{p_1} x_i^+ - \sum_{j=1}^{q_1} x_j^- + \sum_{i={p_1}+1}^{p_2}x_i^+\\
  &=& \sum_{i=1}^{p_2} x_i^+ - \sum_{j=1}^{q_1} x_j^- .    
\end{eqnarray*}

Потом мы выбираем такое наименьшее натуральное $q_2$ (=номер последовательности $(x_n^-)$) большее, чем $q_1$, чтобы было верно неравенство

\begin{eqnarray*}
  \alpha &>& \sum_{i=1}^{p_2} x_i^+ - \sum_{j=1}^{q_1} x_j^- - \sum_{j={q_1}+1}^{q_2}x_j^-\\
  &=&     \sum_{i=1}^{p_2} x_i^+ -  \sum_{j={q_1}+1}^{q_2}x_j^-.
\end{eqnarray*}



Продолжая таким образом, мы получаем последовательность номеров $p_1, q_1,\ldots, p_k,q_k,\ldots,$ и новую последовательность 
\[
 (x_n') = x_1^+, \ldots, x_{p_1}^+, x_1^-, \ldots, x_{q_1}^-, x_{p_1+1}^+, \ldots, x_{p_2}^+, x_{q_1+1}^-, \ldots, x_{q_2}^-, \ldots,
\]
при этом, если числа $p_1,q_1, \ldots, p_k, q_k$ выбраны, то мы имеем
\[
 \alpha > \sum_{i=1}^{p_k} x_i^+ - \sum_{j=1}^{q_k} x_j^-
\]
и тогда мы подбираем $p_{k+1}$ как наименьшее натуральное число, большее, чем $p_k$ так, чтобы
\[
 \alpha < \sum_{i=1}^{p_k}x_i^+ - \sum_{j=1}^{q_k} x_j^- + \sum_{i=p_k+1}^{p_{k+1}} x_i^+ = \sum_{i=1}^{p_{k+1}}x_i^+ - \sum_{j=1}^{q_k}x_j^-,
\]
но тогда (в силу условия минимальности на выбор числа $p_{k+1}$) имеем
\[
 \alpha \ge \sum_{i=1}^{p_k}x_i^+ - \sum_{j=1}^{q_k} x_j^- + \sum_{i=p_k+1}^{p_{k+1}-1} x_i^+ = \sum_{i=1}^{p_{k+1}-1}x_i^+ - \sum_{j=1}^{q_k}x_j^-.
\]

Итак, мы получаем
\[
 \sum_{i=1}^{p_{k+1}-1}x_i^+ - \sum_{j=1}^{q_k}x_j^- \le \alpha < \sum_{i=1}^{p_{k+1}}x_i^+ - \sum_{j=1}^{q_k}x_j^-.
\]

Из полученных неравенств вычтем сумму $\sum_{i=1}^{p_{k+1}}x_i^+ - \sum_{j=1}^{q_k}x_j^-$, тогда получаем
\[
 \left(\sum_{i=1}^{p_{k+1}-1}x_i^+ - \sum_{j=1}^{q_k}x_j^-\right) - \left( \sum_{i=1}^{p_{k+1}}x_i^+ - \sum_{j=1}^{q_k}x_j^- \right) \le \alpha - \left( \sum_{i=1}^{p_{k+1}}x_i^+ - \sum_{j=1}^{q_k}x_j^- \right) < 0.
\]
откуда получаем
\[
 - x_{p_k+1}^+ \le \alpha - \left( \sum_{i=1}^{p_{k+1}}x_i^+ - \sum_{j=1}^{q_k}x_j^- \right) < 0,
\]
или
\[
 0 < \left( \sum_{i=1}^{p_{k+1}}x_i^+ - \sum_{j=1}^{q_k}x_j^- \right) - \alpha \le x_{p_k+1}^+.
\]

Далее, согласно предложению \ref{condtional_convergance_and_x_n^+}, $\lim_{k \to \infty}x_{p_k+1}^+ = 0$, то по лемме о зажатой последовательности (см. лемма \ref{sqeezy}) и предложению \ref{lim(a_n-a)=0} получаем, что

\begin{equation}\label{S_{k+1,k}=a}
    \boxed{
    \lim_{k\to \infty} \left( \sum_{i=1}^{p_{k+1}}x_i^+ - \sum_{j=1}^{q_k}x_j^- \right) = \alpha.    
    }
\end{equation}

С другой стороны, если числа $p_1,\ldots, p_k,q_k,p_{k+1}$ выбраны, то
\[
 \alpha < \sum_{i=1}^{p_{k+1}} x_i^+-  \sum_{j=1}^{q_k}x_j^-,
\]
и тогда $q_{k+1}$ мы выбираем как наименьшее натуральное число такое, что
\[
 \alpha > \sum_{i=1}^{p_{k+1}} x_i^+-  \sum_{j=1}^{q_k}x_j^- - \sum_{j=q_{k+1}}^{q_{k+1}} x_j^- = \sum_{i=1}^{p_{k+1}} x_i^+-  \sum_{j=1}^{q_{k+1}}x_j^-,
\]
а тогда получаем
\[
 \alpha \le \sum_{i=1}^{p_{k+1}} x_i^+-  \sum_{j=1}^{q_{k+1}-1}x_j^-.
\]

Мы получаем неравенства
\[
 \sum_{i=1}^{p_{k+1}} x_i^+-  \sum_{j=1}^{q_{k+1}}x_j^- < \alpha \le \sum_{i=1}^{p_{k+1}} x_i^+-  \sum_{j=1}^{q_{k+1}-1}x_j^-, 
\]
вычитая сумму $\sum_{i=1}^{p_{k+1}} x_i^+-  \sum_{j=1}^{q_{k+1}}x_j^-$ из каждого неравенства, мы получаем
\[
 0 < \alpha - \left( \sum_{i=1}^{p_{k+1}} x_i^+-  \sum_{j=1}^{q_{k+1}}x_j^- \right) \le \left( \sum_{i=1}^{p_{k+1}} x_i^+-  \sum_{j=1}^{q_{k+1}-1}x_j^-\right) - \left( \sum_{i=1}^{p_{k+1}} x_i^+-  \sum_{j=1}^{q_{k+1}}x_j^-\right),
\]
откуда вытекает
\[
 0 < \alpha - \left( \sum_{i=1}^{p_{k+1}} x_i^+-  \sum_{j=1}^{q_{k+1}}x_j^- \right) \le x_{q_{k+1}}^-.
\]

А тогда, согласно предложению \ref{condtional_convergance_and_x_n^+}, $\lim_{k \to \infty}x_{p_k+1}^+ = 0$, то по лемме о зажатой последовательности (см. лемма \ref{sqeezy}) и предложению \ref{lim(a_n-a)=0} получаем, что

\begin{equation}\label{S_{k,k+1}=a}
    \boxed{
    \lim_{k\to \infty} \left( \sum_{i=1}^{p_{k+1}}x_i^+ - \sum_{j=1}^{q_{k+1}}x_j^- \right) = \alpha.    
    }
\end{equation}

Но по построению, все частичные суммы ряда 
\[
 (x_n') = x_1^+, \ldots, x_{p_1}^+, x_1^-, \ldots, x_{q_1}^-, x_{p_1+1}^+, \ldots, x_{p_2}^+, x_{q_1+1}^-, \ldots, x_{q_2}^-, \ldots,
\]
имеют либо вид $\sum_{i=1}^{p_{k+1}}x_i^+ - \sum_{j=1}^{q_k}x_j^-$ либо $ \sum_{i=1}^{p_{k+1}}x_i^+ - \sum_{j=1}^{q_{k+1}}x_j^-$, а тогда из уравнений (\ref{S_{k+1,k}=a}), (\ref{S_{k,k+1}=a}) вытекает, что сумма ряда $(x_n')$ есть $\alpha.$

\end{proof}
