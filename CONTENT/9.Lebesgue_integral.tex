\chapter{Интеграл Лебега; основные понятия}


\section{Мера Лебега на прямой}

\begin{definition}
   \textit{Мера Лебега} $\mu$ промежутка $I \subseteq \overline{\mathbb{R}}$, это число
    \[
     \mu(I): = b-a,
    \]
    где $I$ принимает одно из следующих значений $(a,b), [a,b), (a,b], [a,b]$. В частности, для одноточечного множества $\{a\}$ мы полагаем $\mu(\{a\}):=0$, так так $\{a\} = [a,a]$.
\end{definition}

\begin{mydangerr}{\bf!}
    Ввиду того что определение дано в случае подмножеств расширенной прямой, то мы допускаем бесконечные значения меры $\mu$.
\end{mydangerr}

\begin{definition}
    Говорят, что множество $S \subseteq \overline{\mathbb{R}}$ имеет меру ноль, если для любого $\varepsilon>0$ существуют такое не более чем счётное множество интервалов $\{I_k\}_{k=1}^\infty$ что $S \subseteq \cup_{i=1}^\infty I_k$ и $\sum_{k=1}^\infty \mu(I_k) \le \varepsilon.$
\end{definition}

\begin{lemma}
    Любое счётное множество на прямой имеет меру ноль.
\end{lemma}
\begin{proof}
 Пусть множество $S$ счётно, тогда его элементы можно представить в виде последовательности
 \[
  S= \{s_1,s_2,\ldots,\}
 \]

Для фиксированного $\varepsilon>0$ и для любого $n\ge 1$ положим
    \[
     I_n:=\left( s_n-\frac{\varepsilon}{2^{n+1}}, s_n + \frac{\varepsilon}{2^{n+1}}\right).
    \]

 Так как каждый $s_n\in \mathbb{N}$ это цент интервала $I_n$, то $S \subseteq \cup_{n=1}^\infty I_n$, при этом имеем
    \begin{eqnarray*}
        \sum_{k=1}^\infty &=& \sum_{k=1}^\infty \left( s_n + \frac{\varepsilon}{2^{n+1}} - s_n +\frac{\varepsilon}{2^{n+1}} \right) \\
        &=& \varepsilon \sum_{k=1}^\infty \frac{1}{2^n} = \varepsilon \frac{1/2}{1-1/2} = \varepsilon.        
    \end{eqnarray*}
Таким образом, мы для любого $\varepsilon>0$ предъявили такое покрытие $\{I_k\}_{k=1}^\infty$ множества $S$, что $\sum_{k=1}^\infty \mu(I_k) \le \varepsilon$, что и показывает, что множество $S$ имеет меру ноль.
\end{proof}


\begin{definition}
 Пусть даны две функции $f,g: X \to \mathbb{R}$, \textit{множество несовпадений} этих функций, назовём множество
 \[
  N(f,g): = \{x\in X\, :\, f(x) \ne g(x)\} \subseteq X.
 \]

 Говорят, что функции \textit{почти всюду равны} на $X$, если множество $N(f,g)$ имеет меру ноль. В случае, когда $X = \mathbb{R}$, то вместо фразы ``почти всюду равны на $\mathbb{R}$'' говорят просто ``почти всюду равны''.
\end{definition}



\section{Ступенчатые функции}


\begin{definition}
    Пусть $I$ -- промежуток, под \textit{разбиением промежутка $I$ на промежутки} понимается \textbf{конечное} множество $\mathscr{P}(I)$ промежутков содержащихся в $I$, при этом любой $x \in I$ принадлежит одному и только одному промежутку из $\mathscr{P}(I).$
\end{definition}

\begin{mydanger}{\bf !}
    Основное отличите этого определиния от определения разбиения данного для определения интегарала Римана, заключается в том, что нам уже нужны не только точки, но также и промежутки. 
\end{mydanger}

\begin{example}
  Пусть $I = [1,8]$, тогда следующее множество
  \[
   \mathscr{P}(I) : = \{ \varnothing, \{1\}, (1,3), [3,5), \{5\}, (5,8]  \}
  \]
есть разбиение промежутка $I = [1,8]$, потому что любой $x \in I$ лежит в одном и только в одном из перечисленных подмножеств множества $\mathscr{P}$. А если же мы положим, что
\[
 \mathscr{P}'(I): = \{ \varnothing, \{1\}, [1,3), (2,7), [3,5), \{5\}, (5,8] \}
\]
то мы уже получаем не разбиение, поскольку, например, точка $2.5 \in [1,3)$ и $2.5 \in (2,7)$.

Множество $\mathscr{P}''(I) := \{[1,4), (4,8]\}$ тоже не является разбиением промежутка $I = [1,8]$ так как $4$ не принадлежит ни одному из подмножеств множества $\mathscr{P}''(I)$. Наконец, множество $\mathscr{P}'''(I) := \{(0,5], (5,8]\}$ также не является разбиением промежутка $I = [1,8]$, потому что $(0,5]$ не содержится в $I.$
\end{example}

\begin{mydanger}{\bf !}
    Заметим, что пустое множество может не входить в разбиение промежутка. 
\end{mydanger}





Сейчас мы опишем класс функций, которые ``очень просты'' для интегрирования\footnote{Отметим, что эти функции также ещё называются \textit{кусочно-постоянными}, в англоязычной литературе они так и называются, \textit{piecewise constant functions.}}, а потом с помощью их мы уже определим интеграл в общем виде.  

\begin{definition}
  Пусть $A \subseteq \mathbb{R}$, и пусть дана функция $f: A \to \mathbb{R}$. Говорят, что $f$ \textit{постоянная функция}, если существует такое $\alpha \in \mathbb{R}$, что $f(x) = \alpha$ для всех $x \in A$. Если $B \subseteq A$, то говорят, что $f$ \textit{постоянная на $B$}, если существует такое $\beta \in \mathbb{R}$, что $f(y) = \beta$ для всех $y \in B.$
\end{definition}


\begin{mydanger}{\bf !}
 Из этого определения следует, что если функция $f$ постоянна на \textbf{непустом} множестве $A$, то она не может принимать два или более разных значения. Однако из определения пустого множества следует, что постоянная функция на пустом множестве может принимать \textbf{любое значение!}
\end{mydanger}

\begin{definition}
    Пусть дан промежуток $I \subsetneq \mathbb{R}$, и пусть дана функция $f: I \to \mathbb{R}$, и пусть $\mathscr{P}(I)$ -- какое-то разбиение промежутка $I$. Говорят, что \textit{функция $f$ есть ступенчатая функция на $I$ относительно $\mathscr{P}(I)$}, если для каждого $J \in \mathscr{P}(I)$, $f$ является постоянной на $J$. 
\end{definition}

\begin{example}\label{int_[1,6]=10}
    Пусть $I = [1,6]$ и определим функцию $f: [1,6]: \to \mathbb{R}$ следующим образом
    \[
     f(x) = \begin{cases}
         \,7, & 1 \le x < 3 \\
         \,4, & x = 3 \\
         \,5, & 3 < x <6 \\
         \,2, & x = 6.
     \end{cases}
    \]
Тогда, если мы рассмотрим разбиение
\[
 \mathscr{P}(I) := \Bigl\{[1,3), \{3\}, (3,6), \{6\} \Bigr\}
\]
промежутка $I$, то получаем, что $f$ -- ступенчатая функция относительно этого разбиения. Рассмотрим теперь другое разбиение этого же промежутка
\[
 \mathscr{P}'(I) : = \Bigl\{\varnothing, [1,2), \{2\}, (2,3), \{3\}, (3,5), [5,6),\{6\} \Bigr\}.
\]
Тогда несложно видеть, что $f$ будет тоже ступенчатой относительно этого разбиения. 
\end{example}

Этот пример показывает, что понятие ступенчатой функции можно определить без привлечения разбиения промежутка.

\begin{definition}\label{fiber}
Пусть $I$ -- промежуток, и пусть $\mathscr{P}(I)$, $\mathscr{P}'(I)$ -- два его разбиения. Говорят, что разбиение $\mathscr{P}'(I)$ \textit{тоньше}, чем $\mathscr{P}(I)$, если для каждого $J' \in \mathscr{P}'(I)$ найдётся такой $J \in \mathscr{P}(I)$, что $J' \subseteq J$.
\end{definition}

\begin{example}
 Вернёмся к предыдущему примеру с промежутком $I = [1,6]$ и разбиениями   
 \begin{eqnarray*}
     \mathscr{P}(I) &:=& \Bigl\{[1,3), \{3\}, (3,6), \{6\} \Bigr\},\\
     \mathscr{P}'(I) &: =& \Bigl\{\varnothing, [1,2), \{2\}, (2,3), \{3\}, (3,5), [5,6),\{6\} \Bigr\}.
 \end{eqnarray*}

Тогда видно, что $\mathscr{P}'(I)$ тоньше, чем $\mathscr{P}(I)$.
\end{example}


\begin{definition}
    Пусть дан промежуток $I \subsetneq \mathbb{R}$ и пусть дана функция $f: I \to \mathbb{R}$. Говорят, что функция $f$ \textit{ступенчатая на $I$}, если существует такое разбиение $\mathscr{P}(I)$, что $f$ -- постоянная на $I$ относительно $\mathscr{P}(I).$
\end{definition}


\begin{lemma}\label{fiber_for_functions}
  Пусть $I \subsetneq \mathbb{R}$ -- промежуток и пусть $f:I \to \mathbb{R}$ -- ступенчатая функция относительно разбиения $\mathscr{P}(I)$, тогда если имеем разбиение $\mathscr{P}'(I)$, которое тоньше, чем $\mathscr{P}(I)$, то $f$ -- ступенчатая относительно $\mathscr{P}'(I).$ 
\end{lemma}

\begin{proof}
    Действительно, пусть $A' \in \mathscr{P}'(I)$ -- произвольный элемент разбиения, тогда найдётся такой $A \in \mathscr{P}(I)$, что $A' \subseteq A$. Тогда если $f(x) = \alpha$ для всех $x \in A$, то и $f(x') = \alpha$ для всех $A'.$
\end{proof}

\begin{mydanger}{\bf !}
    Таким образом, мы будем рассматривать просто ступенчатые функции на промежутке, не определяя какое-то конкретное разбиение.
\end{mydanger}


В связи с этим уместно ввести следующее важное для дальнейшего определение.

\begin{definition}
    \textit{Характеристической функцией} некоторого множества $A \subseteq X$ называется функция $\chi_A: X\to \{0,1\}$ определённая следующим образом
    \[
     \chi_A(x): = \begin{cases}
         1 & x \in A, \\
         0 & x \notin A.
     \end{cases}
    \]
\end{definition}

\begin{remark}
 Таким образом, если $f: I \to \mathbb{R}$ -- ступенчатая функция на промежутке $I$ и пусть $\mathscr{P}(I)$ -- соответствующее разбиение промежутка $I$, тогда мы можем записать
 \[
  f =\sum_{A \in \mathscr{P}(I)} f(A) \cdot \chi_A.
 \]
\end{remark}

\begin{mydanger}{\bf !}
    Из определения следует, что $\chi_{\varnothing} = 0$, так как не существует такого $x$ чтобы $x \in \varnothing.$
\end{mydanger}



\begin{example}

Вернёмся к примеру \ref{int_[1,6]=10}, имеем $I = [1,6]$ и функцию $f: [1,6]: \to \mathbb{R}$;
    \[
     f(x) = \begin{cases}
         \,7, & 1 \le x < 3 \\
         \,4, & x = 3 \\
         \,5, & 3 < x <6 \\
         \,2, & x = 6.
     \end{cases}
    \]
Как мы уже видели, $f$ -- ступенчатая относительно разбиений
\begin{align*}
    &  \mathscr{P}(I) := \Bigl\{[1,3), \{3\}, (3,6), \{6\} \Bigr\},\\
    &  \mathscr{P}'(I) : = \Bigl\{\varnothing, [1,2), \{2\}, (2,3), \{3\}, (3,5), [5,6),\{6\} \Bigr\}.
\end{align*}

Тогда получаем, что
\[
 f = 7 \cdot \chi_{[1,3)} + 4 \cdot \chi_{\{3\}} + 5 \cdot \chi_{(3,6)} + 2 \cdot \chi_{\{6\}},  
\]
а также
\[
f = \alpha \cdot \chi_{\varnothing} + 7 \cdot \chi_{[1,2)} + 7 \cdot \chi_{\{2\}} + 7 \cdot \chi_{(2,3)} + 4 \cdot \chi_{\{3\}} + 5 \cdot \chi_{(3,5)} + 5 \cdot \chi_{[5,6)} + 2 \cdot \chi_{\{6\}},
\]
где $\alpha \in \mathbb{R}$ -- произвольное число, но так как $\chi_\varnothing = 0$, то всё корректно.
\end{example}

\section{Интеграл Лебега от ступенчатой функции}

Итак, у нас всё готово, чтобы ввести следующее важное определение.

\begin{definition}\label{int_of_p.c_on_I}
    Пусть $I \subseteq \overline{\mathbb{R}}$ -- промежуток, $\mathscr{P}(I)$ -- разбиение промежутка $I$ и пусть $f:I \to \mathbb{R}$ -- ступенчатая функция относительно этого разбиения, \textit{т.е.} $f = \sum_{A \in \mathscr{P}(I)}f(A) \cdot \chi_A$. Определим \textit{интеграл на промежутке $I$} ступенчатой функции $f:I \to \mathbb{R}$ относительно разбиения $\mathscr{P}(I)$ следующим образом
    \[
     \int_{\mathscr{P}(I)}f: =  \sum_{A \in \mathscr{P}(I)} f(A)\cdot \mu(A)
    \]
\end{definition}

\begin{mydanger}{\bf !}
     Во-первых, что стоит слева от равно нужно понимать как символ и не более того! Во-вторых, это определение может показаться некорректным если $\mathscr{P}(I)$ содержит пустое множество, но так как $|\varnothing| = 0$, то мы на самом деле получаем корректное определение.     
\end{mydanger}

\begin{remark}\label{int_via_chi}
    Если $f = \chi_I$, и взяв разбиение $\mathscr{P}(I) = \{I\}$ то мы получаем следующее
    \[
     \int_{\mathscr{P}(I)}\chi_I = \mu(I).
    \]
И тогда мы можем записать, что если $f = \sum_{A \in \mathscr{P}(I)}f(A) \cdot \chi_A$, то
\[
\boxed{
 \int_{\mathscr{P}(I)}f = \sum_{A \in \mathscr{P}(I)} f(A) \cdot \int_{\mathscr{P}(I)}\chi_A
 }
\]
\end{remark}


\begin{example}\label{int_[1,4]=10}
    Пусть $f: [1,4] \to \mathbb{R}$ определена следующим образом
    \[
     f(x)  = \begin{cases}
          \, 2 & 1 \le x <3 \\
          \, 4 & x = 3 \\
          \, 6 & 3< x \le 4
     \end{cases}
    \]
    и пусть $\mathscr{P}(I) = \{ [1,3), \{3\}, (3,4] \}$, тогда
 \begin{eqnarray*}
  \int_{\mathscr{P}(I)} f  &=& \alpha_{[1,3)}\cdot | [1,3) | + \alpha_{\{3\}}\cdot |\{3\}| + \alpha_{(3,4]} \cdot | (3,4] | \\
  &=& 2 \cdot 2 + 4 \cdot 0 + 6 \cdot 1 \\
  &=& 10.
 \end{eqnarray*}

 С другой стороны, рассмотрим такое разбиение $\mathscr{P}'(I) = \{ \varnothing, [1,2), [2,3), \{3\}, (3,4] \}$, нетрудно видеть, что оно тоньше разбиения $\mathscr{P}(I)$. Находим
 \begin{eqnarray*}
     \int_{\mathscr{P}'(I)}f &=&\alpha_\varnothing \cdot |\varnothing| + \alpha_{[1,2)} \cdot | [1,2) | + \alpha_{[2,3)} \cdot |[2,3)| + \alpha_{\{3\}}\cdot |\{3\}| + \alpha_{(3,4]} \cdot | (3,4] | \\
     &=& \alpha_\varnothing \cdot 0 + 2 \cdot 1 + 2 \cdot 1 + 4 \cdot 0 + 6 \cdot 1 \\
     &=& 10.
 \end{eqnarray*}
 
\end{example}

Итак, мы увидели, что, взяв разбиение тоньше, значение интеграла не изменилось, очевидно, что это верно и в общем случае.

\begin{lemma}
  Пусть $I \subsetneq \mathbb{R}$ -- промежуток и пусть $f:I \to \mathbb{R}$ -- ступенчатая функция относительно разбиения $\mathscr{P}(I)$, тогда если имеем разбиение $\mathscr{P}'(I)$, которое тоньше, чем $\mathscr{P}(I)$, то
  \[
   \int_{\mathscr{P}(I)}f = \int_{\mathscr{P}'(I)}f.
  \]
\end{lemma}

\begin{proof}
    Пусть $\mathscr{P}(I) = \{ A_1,\ldots, A_n \}$ и пусть 
    \[
     \mathscr{P}'(I) : = \Bigl\{A_{11}', \ldots, A'_{1\ell_1},\ldots, A'_{n1},\ldots, A'_{n\ell_n} \Bigr\},
    \]
    где $A_i$ содержит только $A'_{i1},\ldots, A'_{i\ell_i}$, $1\le i \le n$. Из определения \ref{fiber} тогда следует, что $A_i = A'_{i1} \cup \cdots \cup A'_{i\ell_i}$ и $|A_i| = |A'_{i1}| + \cdots + |A'_{i\ell_i}|$.
    
    Наконец, используя лемму \ref{fiber_for_functions}, получаем, что
    \[
     f(A'_{i1}) = \cdots = f(A'_{i\ell_i}) = f(A_i), \qquad 1 \le i \le \ell.
    \]

    Таким образом, имеем
    \begin{eqnarray*}
        \int_{\mathscr{P}'(I)}f &=& \Bigl(f(A_{11}') \cdot \left| A'_{11} \right| + \cdots + f(A_{1\ell_1})\cdot \left|A'_{i\ell_1}\right|\Bigr) + \cdots + \Bigl(f(A_{11}') \left|A_{11}'\right| + \cdots + f(A_{1\ell_1})\cdot \left|A'_{i\ell_1}\right| \Bigr) \\
        &=& f(A_1) \cdot \left( \left| A'_{11}  \right| + \cdots + \left| A'_{1\ell_1} \right| \right) + \cdots + f(A_n) \cdot \left( \left| A'_{n1}  \right| + \cdots + \left| A'_{n\ell_n} \right| \right) \\
        &=& f(A_1) |A_1| + \cdots + f(A_n)\cdot |A_n| \\
        &=& \int_{\mathscr{P}(I)}f.
    \end{eqnarray*}
\end{proof}


Таким образом, мы можем ввести следующее определение, которое будем использовать в дальнейшем.

\begin{definition}\label{int_of_p.c}
    Пусть $I \subseteq \overline{\mathbb{R}}$ -- промежуток, $f:I \to \mathbb{R}$ -- ступенчатая функция на нём. Определим \textit{интеграл на промежутке $I$} ступенчатой функции $f:I \to \mathbb{R}$ следующим образом
    \[
     \int_If: =  \int_{\mathscr{P}(I)}f,
    \]
    где $\mathscr{P}(I)$ такое разбиение промежутка $I$, что $f$ является ступенчатой относительно $\mathscr{P}(I).$
\end{definition}


\begin{example}
Вернёмся к примеру \ref{int_[1,4]=10}. Пусть $f: [1,4] \to \mathbb{R}$ определена следующим образом
    \[
     f(x)  = \begin{cases}
          \, 2 & 1 \le x <3 \\
          \, 4 & x = 3 \\
          \, 6 & 3< x \le 4
     \end{cases}
    \]
    и пусть $\mathscr{P}(I) = \{ [1,3), \{3\}, (3,4] \}$, тогда
 \begin{eqnarray*}
  \int_{\mathscr{P}(I)} f  &=& \alpha_{[1,3)}\cdot | [1,3) | + \alpha_{\{3\}}\cdot |\{3\}| + \alpha_{(3,4]} \cdot | (3,4] | \\
  &=& 2 \cdot 2 + 4 \cdot 0 + 6 \cdot 1 \\
  &=& 10.
   \end{eqnarray*}

Тогда
\[
 \int_{[1,4]}f   =10.
\]
\end{example}

\section{Поточечная и равномерная сходимость}

Наш интерес к ступнчатым функциям связан с тем, что любую функую можно приблизить последовательностью ступенчатых. Чтобы дать более строгое утверждение, нам нужно определение

\begin{definition}
    Говорят, что полседовательность функций $(f_n)$ поточечно сходится к функции $f$, где $f, f_n : X \to \mathbb{R}$, $n\ge 1$, если для каждого $x \in X$, последовательность $(f_n(x))$ сходится к $f(x)$.
\end{definition}

Тогда возникает вопрос, допустим теперь, что все функции $f_n, f$ интегрируемы на $X$, будет ли тогда из равенств $\lim_{n\to \infty} f_n(x) = f(x)$, вытекать равенство $\lim_{n\to \infty} \int f_n \to \int f$?

Но следующий пример показывает, что это не так.

\begin{example}
    Пусть $X = [0,1]$, и пусть $\mathbb{Q} \cap [0,1] = \{r_1,r_2,\ldots,\}$, рассмотрим последовательность функций 
    \[
     d_n(x): = \begin{cases}
         1, & x \in \{r_1,\ldots, r_n\},\\
         0, & x \notin\{r_1,\ldots, r_n\},
     \end{cases}
    \]
    тогда $\int_X d_n = 0$, при каждом $n \ge 1$ и более того $\lim_{n\to \infty} d_n =d$, где
    \[
     d(x) = \begin{cases}
         1, & x \in \mathbb{Q}\cap [0,1] \\
         0, & x \notin \mathbb{Q} \cap [0,1]
     \end{cases}
    \]
    это так называемая \textit{функция Дирихле}, нетрудно видеть что $\overline{\int}_X d = 1$, $\underline{\int}_I d  =0$, \textit{т.е.,} функция $d$ не интегрируема.
\end{example}

Итак из сходимости функций не следует, вообще говоря, сходимость интегралов, тем не менее при определённых условиях последовательность интегририруемых функций всё же сходится к интегрируемой функции и более того последовательность интегралов сходится к интегралу.

\begin{definition}
    Говрят, что последовательность функций $(f_n)$, $f_n:X \to \mathbb{R}$ \textit{сходится равномерно} на $X$ к функции $f:X \to \mathbb{R}$, если для любого $\varepsilon>0$ существует такое $N \ge 1$, что при всех $n \ge N$ имеем
    \[
     |f_n(x) -f(x)| <\varepsilon
    \]
    для всех $x \in X.$
\end{definition}

\begin{remark}
    Ясно, что каждая равномерно сходящаяся последовательность сходится и поточечно. Разница между этими двмя понятиями заключается в следующем. Если оследовательность $(f_n)$ сходится поточечно на $X$, то существует функция $f$, такая, что для любого $\varepsilon>0$ и \textbf{для каждого} $x \in X$ существует число $N\ge 1$, которое зависит как от $\varepsilon$ так и от $x$, такое что $|f_n(x) -f(x)| <\varepsilon$ при всех $n \ge N$. А в случае, когда $(f_n)$ сходится равномерно к $f$, то можно при каждом $\varepsilon>0$ найти \textbf{одно число} $N\ge 1$, которое будет годится \textbf{для всех} $x \in X$.
\end{remark}

\begin{theorem}
    Если последовательность функций $(f_n)$, интегрируемых на отрезке $[a,b]$, равномерно сходится на этом отрезке к функции $f$, то функция $f$ интегрируема и более того
    \[
     \lim_{n\to \infty}\int_a^b f_n = \int_a^b f. 
    \]

    Более того, последовательность функций $F_n$ равномерно сходится к функции $F$, где
    \[
     F_n(x) : = \int_a^x f_n, \qquad F(x) : = \int_a^x f.
    \]
\end{theorem}


\begin{proof}~ Доказательство разобьём на несколько шагов.

(1) Покажем ограниченность функции $f$. 

Так как каждая функция $f_n$ интегрируема на $[a,b]$, она ограничена. Поскольку $f_n$ равномерно сходится к $f$, для $\varepsilon = 1$ найдется номер $N$ такой, что для всех $n \geq N$ и всех $x \in [a,b]$ выполняется $|f_n(x) - f(x)| < 1$. В частности, для $n = N$ имеем $|f_N(x) - f(x)| < 1$. Функция $f_N$ ограничена, поэтому существует константа $C_N > 0$ такая, что $|f_N(x)| \leq C_N$ для всех $x \in [a,b]$. Тогда 
\[
|f(x)| \leq |f(x) - f_N(x)| + |f_N(x)| < 1 + C_N
\]
для всех $x \in [a,b]$. Следовательно, $f$ ограничена.

(2) Покажем интегрируемость функции $f$

Зафиксируем $\varepsilon > 0$. Поскольку $f_n$ равномерно сходится к $f$, найдется номер $N$ такой, что для всех $n \geq N$ и всех $x \in [a,b]$ выполняется 
\[
|f_n(x) - f(x)| < \frac{\varepsilon}{3(b-a)}.
\]
Зафиксируем $n = N$. Так как $f_N$ интегрируема, то, согласно критерию интегрируемости (см. Теорема \ref{criteria_for_Rieman}), существует разбиение $\mathsf{P}'$ отрезка $[a,b]$, для которого 
\[
\overline{\mathcal{R}}(\mathsf{P}',f_N) - \underline{\mathcal{R}}(\mathsf{P}',f_N) < \frac{\varepsilon}{3}.
\]

Рассмотрим разбиение $\mathsf{P}'$. Для любого отрезка $[x_{i-1}, x_i]$ разбиения $\mathsf{P}'$ и любых точек $x', x''$ из этого отрезка имеем
\begin{align*}
|f(x') - f(x'')| &\leq |f(x') - f_N(x')| + |f_N(x') - f_N(x'')| + |f_N(x'') - f(x'')| \\
&< \frac{\varepsilon}{3(b-a)} + |f_N(x') - f_N(x'')| + \frac{\varepsilon}{3(b-a)}.
\end{align*}
Следовательно, 
\[
\sup_{[x_{i-1},x_i]} f - \inf_{[x_{i-1},x_i]} f \leq \frac{2\varepsilon}{3(b-a)} + \left( \sup_{[x_{i-1},x_i]} f_N - \inf_{[x_{i-1},x_i]} f_N \right).
\]

Умножая на длину отрезка $\Delta x_i:=x_i - x_{i-1}$ и суммируя по всем отрезкам разбиения, получаем
\begin{align*}
\overline{\mathcal{R}}(\mathsf{P}',f) - \underline{\mathcal{R}}(\mathsf{P}',f) 
&\leq \sum_i \frac{2\varepsilon}{3(b-a)} \Delta x_i + \sum_i \left( \sup f_N - \inf f_N \right) \Delta x_i \\
&= \frac{2\varepsilon}{3} + \left( \overline{\mathcal{R}}(\mathsf{P}',f_N) - \underline{\mathcal{R}}(\mathsf{P}',f_N) \right) \\
&< \frac{2\varepsilon}{3} + \frac{\varepsilon}{3} = \varepsilon.
\end{align*}
Таким образом, $f$ интегрируема по Риману на $[a,b]$.

(3) Найдем теперь предел интегралов.

Для произвольного $\varepsilon > 0$ выберем номер $N$ такой, что для всех $n \geq N$ и всех $x \in [a,b]$ выполняется
\[
|f_n(x) - f(x)| < \frac{\varepsilon}{b-a}.
\]
Тогда для всех $n \geq N$ получаем
\[
\left| \int_a^b f_n - \int_a^b f \right| \leq \int_a^b |f_n - f|  dx < \int_a^b \frac{\varepsilon}{b-a}  dx = \varepsilon.
\]
Следовательно, 
\[
\lim_{n\to \infty} \int_a^b f_n = \int_a^b f.
\]

(4) Нам осталось доказать равномерную сходимость $F_n$ к $F$. 
Для произвольного $\varepsilon > 0$ выберем номер $N$ такой, что для всех $n \geq N$ и всех $t \in [a,b]$ выполняется
\[
|f_n(t) - f(t)| < \frac{\varepsilon}{b-a}.
\]
Тогда для всех $n \geq N$, всех $x \in [a,b]$ и всех $t \in [a,x]$ имеем $|f_n(t) - f(t)| < \frac{\varepsilon}{b-a}$. Следовательно,
\[
|F_n(x) - F(x)| = \left| \int_a^x (f_n - f) \right| \leq \int_a^x |f_n - f|  dt < \int_a^x \frac{\varepsilon}{b-a}  dt = \frac{\varepsilon}{b-a} (x - a) \leq \varepsilon.
\]
Таким образом, $F_n$ равномерно сходится к $F$ на $[a,b]$. Это завершает доказательство.

\end{proof}


\section{Интеграл Лебега}

Мы объясним что такое интеграл Лебега от функции используя подход Даниелля -- Риса. Этот подход заключается в следующем. Если мы хотим интегрировать функцию $f$, то мы должны найти последовательсность ступенчатых функций $\{\phi_n\}$, которые поточечно сходятся к функии $f$ и для которых последовательность $(\int_I \phi_n )$ сходится, тогда предел последовательности этих интегралов и есть интеграл Лебега функции $f.$

Будем считать, что $f\ge 0$


\begin{definition}
    Пусть $(\phi_n(x))$ последовательность неубывающих ступенчатых функций которые поточечно сходятся к фукнкции $f:X \to \mathbb{R}$ при каждом $x \in X$. Тогда интеграл Лебега от этой функции это предел
    \[
     \int_X f \mathrm{d}x: = \lim_{n \to \infty} \int_X\phi_n(x) \mathrm{d}\mu.
    \]
\end{definition}

Это определение эквивлентно следующему определению.

\begin{definition}
    Пусть $\mathcal{S}^+$ -- множество всех ступечатых функций $\phi(x)$ таких что $\phi(x) \le f(x)$ для всех $x \in X$, тогда интеграл Лебега от функции $f$ определяется так 
    \[
     \int_Xf\mathrm{d}\mu : = \sup \left\{ \int_X \phi_n(x) \mathrm{d}\mu,\,\, \phi \in \mathcal{S}^+ \right\}.
    \]
\end{definition}


\begin{lemma}
    Пусть $f,g$ интегрируемы по Лебегу на $X$, тогда
    \begin{itemize}
        \item[(1)] если $f =g$ почти всюду на $X$, то $\int_Xf \mathrm{d}\mu = \int_X g\mathrm{d}\mu$,
        \item [(2)] если $f \le g$ почти всюду на $X$, то $\int_Xf \mathrm{d}\mu \le \int_X g\mathrm{d}\mu$.
    \end{itemize}
\end{lemma}

\begin{proof}
Любую ступенчаю функцию $\phi:X \to [0,+\infty]$ определённую следующим образом
    \[
     \phi(x) : = \sum_{i=1}^n c_i \cdot \chi_{A_i}(x),
    \]
    можно записать так
    \[
     \phi(x) = \sum_{\alpha \in \phi(X)} \alpha\cdot \chi_{\{ x \in X\, :\, \phi(x) = \alpha \}}.
    \]

    Тогда
    \[
     \int_X \phi \mathrm{d}\mu = \sum_{\alpha \in \phi(X)} \alpha\cdot \mu(\{ x \in X\, :\, \phi(x) = \alpha \}).
    \]

    Тогда если $X = X'\sqcup X''$, где $\mu(X'') = 0$, то пусть 
    \[
     \widetilde{\phi}(x): = \begin{cases}
         \phi(x), & x \in X'\\
         0, & x \in X'',
     \end{cases}
    \]
    \textit{т.е.,} имеем равенство $\phi = \widetilde{\phi}$ почти всюду на $X$. 
    
(1) Так как $\mu(X) = \mu(X') + \mu(X'')$, то получаем
\begin{eqnarray*}
 \int_X \widetilde{\phi} \mathrm{d}\mu &:=& \sum_{\alpha \in \widetilde{\phi}(X)} \alpha\cdot \mu(\{ x \in X\, :\, \widetilde{\phi}(x) = \alpha \}) \\
 &=& \sum_{\alpha \in \phi(X)} \alpha\cdot \mu(\{ x \in X'\, :\, \phi(x) = \alpha \}) + 0\cdot \mu(X'')\\
 &=&\sum_{\alpha \in \phi(X)} \alpha\cdot \mu(\{ x \in X\, :\, \phi(x) = \alpha \})\\
 &=:& \int_X \phi \mathrm{d}\mu.
\end{eqnarray*}
    
Это означает, что если ступенчатые функции почти всюду совпадают, то совпадают их интегралы Лебега. Но тогда и интегралы $\int_X f \mathrm{d}\mu$, $\int_X g \mathrm{d}\mu$ совпадают ибо они определяются через интегралы ступенчатых функций.

(2) Пусть $X = X_1 \sqcup X_2$, где $X_1: = \{x\in X\, :\, f(x) \le g(x)\}$, тогда, по условию $\mu(X_2) = 0.$ В силу определения интеграла Лебега 
\begin{eqnarray*}
    \int_X f \mathrm{d}\mu &:=& \sup \left\{ \int_X \phi \mathrm{d}\mu \, :\, \phi \in \mathcal{S}^+,\, \phi \le f \right\} \\
    &=& \sup \left\{ \int_X \widetilde{\phi} \mathrm{d}\mu \, :\, \widetilde{\phi} \in \mathcal{S}^+,\, \widetilde{\phi} \le f \right\}.
\end{eqnarray*}

Так как $f\le g$ при любом $x \in X_1$, то $\widetilde{\phi}(x) \le f \le g$, \textit{т.е.,} $\widetilde{\phi}(x) \le g(x)$.

Тогда
\[
\left\{ \int_X \widetilde{\phi} \mathrm{d}\mu \, :\, \widetilde{\phi} \in \mathcal{S}^+,\, \widetilde{\phi} \le f \right\} \subseteq \left\{ \int_X \widetilde{\phi} \mathrm{d}\mu \, :\, \widetilde{\phi} \in \mathcal{S}^+,\, \widetilde{\phi} \le g \right\},
\]
поэтому 
\[
 \int_Xf \mathrm{d}\mu : = \sup \left\{ \int_X \widetilde{\phi} \mathrm{d}\mu \, :\, \widetilde{\phi} \in \mathcal{S}^+,\, \widetilde{\phi} \le f \right\} \le \sup \left\{ \int_X \widetilde{\phi} \mathrm{d}\mu \, :\, \widetilde{\phi} \in \mathcal{S}^+,\, \widetilde{\phi} \le g \right\}=: \int_X g \mathrm{d}\mu,
\]
что завершает доказательство.    
\end{proof}


\begin{theorem}
    Функция $f:I \to \mathbb{R}$ интегрируема по Риману тогда и только тогда, когда она почти всюду непрерывна на нём, и более того интеграл Римана от этой функции совпадает с интегралом Лебега.
\end{theorem}