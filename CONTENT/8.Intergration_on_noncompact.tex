\chapter{Интегрирование на некомпактном промежутке}

\section{Несобственный интеграл}


В этой главе $I$ может иметь как конечную, так и бесконечную длину, \textit{т.е,} мы считаем, что $I \subseteq \overline{\mathbb{R}}:=\mathbb{R}\cup \{\pm \infty\}$, мы также рассматриваем случай когда $I$ имеет конечную длину, но, вообще говоря, не является компактным.


\subsection{Основные понятия}



\begin{definition}\label{inproper_int}
    Пусть $I \subseteq \overline{\mathbb{R}}$ -- не компактный интервал\footnote{Проще говоря, мы рассматриваем случай, когда $I$ это один из промежутков вида $(a,b)$, $[a,b)$, $(a,b]$.} расширенной прямой $\overline{\mathbb{R}}$, у которого\footnote{Так как мы рассматриваем расширенную прямую, то теперь любое множество $I \subseteq \overline{\mathbb{R}}$ ограниченно, хотя бы элементами $\pm \infty$.} $\inf (I) = a$, $\sup (I)=  b$, где $a \ne - \infty$, допустим что для любого $a\le \beta <b$ существует интеграл Римана $\int_a^\beta f$ от функции $f:[a,b) \to \mathbb{R}$. Тогда если существует предел $\lim_{\beta \to b-}\int_a^\beta f$, то этот предел называют \textit{несобственным интегралом от $f$} на $[a,b]$ и записывают в виде
    \[
     \int_a^b f := \lim_{\beta \to b -} \int_a^\beta f.
    \]
    Аналогично, в случае, когда $b \ne +\infty$ и для каждого $a<\alpha \le b$ существует интеграл Римана $\int_\alpha^b f$ функции $f:(a,b] \to \mathbb{R}$, тогда если существует предел $\lim_{\alpha \to a+} \int_\alpha^b f$, то этот предел называют \textit{несобственным интегралом от $f$} на $[a,b]$ и записывают в виде
    \[
     \int_a^b f := \lim_{\alpha \to a+} \int_\alpha^b f.
    \]

    Наконец, в случае произвольных $a,b \in \overline{\mathbb{R}}$, если оба предела $\lim_{\alpha \to a}\int_\alpha^c f$, $\lim_{\beta \to b}\int_c^\beta f$ существуют, где $a < c <b$, то их сумму называют \textit{несобственным интегралом от $f$} на $[a,b]$ и записывают в виде
    \[
     \int_a^b f := \lim_{\alpha \to a+} \int_\alpha^cf + \lim_{\beta \to b-} \int_c^\beta f.
    \]
\end{definition}

\begin{mydanger}{\bf !}
    В таком случае также принято говорить, что интеграл $\int_a^b f$ \textit{сходится.} В противном случае, говорят, что он \textit{расходится} или \textit{не существует как несобственный интеграл.}
\end{mydanger}

\begin{mydangerr}{\bf !}
    В силу аддитивности интеграла Римана (=Теорема \ref{additive_of_int}), последний абзац в определении корректен, ибо $\int_\alpha^c f + \int_c^\beta f = \int_\alpha^\beta f$, поэтому, можно также написать $\int_a^b f := \lim_{\alpha \to a+} \lim_{\beta \to b-} \int_\alpha^\beta f.$
\end{mydangerr}

Для дальнейшего нам понадобится следящее очень удобное понятие.

\begin{definition}
    Интеграл $\int_a^b f$ будем называть \textit{интегралом с особенностью в точке $b$ (соотв. в точке $a$)}, если $b \ne + \infty$ (\textit{соотв.} $a \ne - \infty$) и $f$ интегрируема на $[a,x]$ при любом $a\le x <b$ (\textit{соотв.} $f$ интегрируема на $[y,b]$ при любом $a < y \le b$) и неограниченна в окрестности точки $b$ (\textit{соотв.} в окрестности $a$). Если же $b = + \infty$ (\textit{соотв.} $a = - \infty$), то про $f$ предполагается лишь то, что она интегрируема на $[a,b']$ при любом $b' < b$, $b' \in \mathbb{R}$ (\textit{соотв.} $f$ интегрируема на $[a',b]$ при любом $a < a'$, $a' \in \mathbb{R}$).
\end{definition}

\begin{mydangerr}{\bf !}
    Мы будем рассматривать интегралы $\int_a^bf$ с особенностью в точке $b$, конечной или бесконечной. Все выводы по аналогии переносятся на случай с особенностью в точке $a.$ 
\end{mydangerr}

Другими словами, в этом и дальнейших пунктах при рассмотрении свойств интегралов будем остававливаться более подробно лишь на интегралах от функций, ппределённых на конечных или бесконечных промежутках вида $[a,b)$ и интегрируемых по Риману всех отрезках $[a,\beta]$, где $a \le \beta < b \le +\infty.$ Любые другие предположения будут спецально оговариваться.




\subsection{Критерий Коши}

\begin{theorem}[Критерий Коши сходимости интеграла]\label{Coushy_for_int}
    Пусть интеграл $\int_a^b f$ имеет особенность в $b$, тогда он существует тогда и только тогда, когда для любого $\varepsilon >0$ существует такое $\eta \in [a,b)$, что для любых $a \le \eta < x' <x'' <b$ вытекает $\left| \int_{x'}^{x''} f \right| < \varepsilon.$
\end{theorem}
Фактически это пересказ существования предела $\lim_{x \in b, x \in [a,b)} \int_a^x f$ (см. Определение \ref{lim_via_neighberhoods}).

\begin{proof}
Действительно, рассмотрим функцию $F(x): = \int_a^x f$, где $a\le x <b$, тогда предел $\lim_{x \to b-}F(x)$ существует тогда и только тогда, когда для любого $\varepsilon>0$ существует такое $\delta>0$, что $F((b-\delta,b)) \subseteq (F(b-)-\varepsilon, F(b-)+\varepsilon)$, где $b - \delta > a.$ Но это значит, что для всех $x', x'' \in (b-\delta,b)$ будет верно неравенство $|F(x'') - F(x')| < \varepsilon$. 

Но в силу аддитивности интеграла
\[
 |F(x'') - F(x')| = \left| \int_a^{x'}f - \int_a^{x''}f   \right| = \left| \int_a^{x'}f - \int_a^{x'} f - \int_{x'}^{x''}f \right| = \left| \int_{x'}^{x''}f \right| < \varepsilon.
\]

Осталось положить, что $b-\delta < \eta < b$ и мы завершаем доказательство теоремы.
\end{proof}


\begin{lemma}\label{good_lemma_for_nonproper_int}
    Если интеграл $\int_a^b f$ существует, то для любого $a \le \alpha <b$, существует интеграл $\int_\alpha^b f.$
\end{lemma}

\begin{proof}
    Действительно, в силу аддитивности интеграла (см. Теорема \ref{additive_of_int}), имеем
    \[
     \int_a^\beta f = \int_a^\alpha f + \int_\alpha^\beta f,
    \]
    тогда
\[
\lim_{\beta \to b-}  \int_a^\beta f = \lim_{\beta \to b-} \left( \int_a^\alpha f + \int_\alpha^\beta f \right) = \int_a^\alpha f + \lim_{\beta \to b-}  \int_\alpha^\beta f ,
\]
откуда и следует утверждение леммы.
\end{proof}


\subsection{Линейность и абсолютная сходимость}

\begin{theorem}
    \[
     \int_a^b (\alpha \cdot f + \beta \cdot g) = \alpha \cdot \int_a^b f + \beta \cdot \int_a^b g
    \]
\end{theorem}
\begin{mydangerr}{\bf !}
    Это равенство надо понимать в том смысле, что если существуют интегралы в правой части, то существует также интеграл в левой части равенства.
\end{mydangerr}

\begin{proof}
    Действительно, так как несобственный интеграл определяется как предел от интеграла Римана, то согласно линейности интеграла Римана (см. Теоремы \ref{int(f+g)=int(f)+int(g)}, \ref{int(af)=a*int(f)}) и арифметике предела (см. Теорема \ref{lim(f+g)}), мы получаем требуемое.
\end{proof}


\begin{definition}
    Говорят, что интеграл $\int_a^bf$, имеющий особенность в точке $b$, \textit{сходится абсолютно}, если сходится интеграл $\int_a^b |f|.$
\end{definition}


\begin{proposition}
    Если интеграл сходится абсолютно, то он сходится.
\end{proposition}

\begin{proof}
    Пусть сходится интеграл $\int_a^b |f|$, тогда, используя критерий Коши (=Теорема \ref{Coushy_for_int}), для любого $\varepsilon>0$ на интервале $(a,b)$ существует такая точка $\eta$, что если $\eta < x' < x'' < b$, то $\int_{x'}^{x''} |f| < \varepsilon$. Далее, так как $\int_{x'}^{x''} |f|$ это интеграл Римана, то, согласно интегрируемости $|f|$ (см. Следствие \ref{int|f|}), 
    \[
     \left| \int_{x'}^{x''} f \right| \le \int_{x'}^{x''} |f| < \varepsilon,
    \]
    \textit{т.е.,} для интеграла $\int_a^b f$ выполняется условие Коши (=Теорема \ref{Coushy_for_int}), поэтому интеграл $\int_a^b f$ сходится.
\end{proof}

\begin{theorem}[Формула Ньютона--Лейбница для несобственного интеграла]
    Пусть интеграл $\int_a^b f$ имеет особенность хотя бы одной из точек $a,b$, тогда, если существует непрерывная функция на отрезке $[a,b]$, такая, что $F'(x) = f(x)$ при всех $x \in [a,b)$, то
    \[
     \int_a^b f = \lim_{\beta \to b-}F(\beta)- \lim_{\alpha \to a+} F(\alpha).
    \]
\end{theorem}
\begin{proof}
    Действительно, согласно определению несобственного интеграла (=Определение \ref{inproper_int}) и теореме Ньютона--Лейбница (=Теорема \ref{the_general_theorem_of_int}), имеем
    \begin{eqnarray*}
        \int_a^b f &:=& \lim_{\alpha \to a+} \int_\alpha^cf + \lim_{\beta \to b-} \int_c^\beta f \\
        &=& F(c) - \lim_{\alpha \to a+} F(\alpha) + \lim_{\beta \to b-}F(\beta) - F(c) \\
        &=& \lim_{\beta \to b-}F(\beta)- \lim_{\alpha \to a+} F(\alpha),
    \end{eqnarray*}
    что и требовалось доказать.    
\end{proof}


\section{Интегрирование неотрицательных функций}

В этой секции, мы будем предполагать, что функция $f$ является неотрицательной. Всё что будет сказано далее переносится на случай когда функция $f$ является неположительной, для этого достаточно изменить все знаки в рассуждениях.

\begin{definition}
 Функция $f:I \to \mathbb{R}$ называется \textit{неубывающей} (соотв. \textit{невозрастающей)}, если при $x<y$, следует $f(x) \le f(y)$ (соотв. $f(x) \ge f(y)$), для всех $x,y \in I.$ Функцию которая является либо невозрастающей или неубывающей называют \textit{монотонной}. 
\end{definition}


\begin{claim}\label{int=monotonic}
 Пусть $f \ge 0$ на, вообще говоря, некомпактном промежутке $I = [a,b] \subseteq \overline{\mathbb{R}}$, тогда функция $F(x):=\int_a^x f$ -- является монотонной, при условии, что интеграл существует.    
\end{claim}

\begin{proof}
 Действительно, пусть $x<y$, тогда, в силу аддитивности интеграла (см. Теорема \ref{additive_of_int}), 
 \[
  F(y) = \int_a^y f = \int_a^x f + \int_x^y f = F(x) + \int_x^y f , 
 \]
 но, так как $f \ge 0$, то в силу монотонности интеграла (см. Следствие \ref{int>0}), $\int_x^y f \ge 0$, откуда $F(y) \ge F(x)$, что и доказывает утверждение.
\end{proof}


\subsection{Некоторые свойства монотонных функций}

Мы ограничимся рассмотрением неубывающих функций, потому что, если $f(x)$ -- неубывающая, то функция $-f(x)$ будет тогда невозрастающей.


\begin{theorem}\label{about_monotonic}
 Пусть функция $f:(a,b) \to \mathbb{R}$ неубывающая, тогда, для любого $\xi \in (a,b)$
 \[
  \lim_{x \to \xi+}f(x) = \inf_{x \in (\xi, b)} f(x), \qquad \lim_{x\to \xi - }f(x) = \sup_{x \in (a,\xi)} f(x).
 \]
\end{theorem}
\begin{mydangerr}{\bf !}
    Так как у нас всё происходит в расширенной прямой, то, конечно же, мы допускаем возможность $a = - \infty$ и $b = + \infty.$
\end{mydangerr}

\begin{proof}
 Мы докажем лишь второе равенство, так как первое доказывается совершенно аналогично.

 В силу того, что функция $f$ не убывающая, то при любом $a < x < \xi$, $f(x) \le f(\xi)$, поэтому множество $\{f(x),\, a < x < \xi\}$ -- ограниченно сверху, а тогда, согласно принципу полноты Вейерштрасса (см. Теореме \ref{W=complete}) у этого множества существует 
    \[
    M= \sup \{f(x),\, a < x < \xi\}
    \]
Тогда $M \le f(\xi)$, покажем теперь, что $M = f(\xi-): = \lim_{x\to \xi - }f(x).$ Из определения $\sup$ следует, что для любого $\varepsilon>0$ можно найти такое $x_0\in (a ,\xi)$, что 
\[
 M-\varepsilon < f(x_0) \le M,
\]
мы можем положить, что $x_0 = \xi - \delta$, для некоторого $\delta>0$, и тогда мы можем сказать, что для любого $\varepsilon>0$ существует такое $\delta >0$, что что если $a <\xi-\delta <\xi$, то
\[
 M-\varepsilon < f(\xi  - \delta) \le M.
\]

В силу не убывания функции $f$, для любого $\xi - \delta <x <\xi$ имеем
\[
 f(\xi - \delta ) \le f(x) \le M,
\]
тогда, учитывая предыдущее неравенство,  для любого $\xi - \delta <x <\xi$ получаем
\[
 M - \varepsilon < f(x) < M,
\]
это можно переписать так
\[
 |f(x) - M| < \varepsilon.
\]

Итак, мы для любого $\varepsilon>0$ нашли $\delta >0$, такое, что всякий раз когда $\xi-\delta < x < \xi$, получаем $|f(x) - M| < \varepsilon$, что и означает $\lim_{x \to \xi-}f(x) = M$.
\end{proof}

\begin{corollary}\label{cor_about_monotonic}
Если неубыващая функция $f:(a,b) \to \mathbb{R}$ ограничена сверху, то существует предел $\lim_{x\to b-}f(x)$ и $\lim_{x \to b-}f(x) = \sup_{x\in (a,b)}f(x)$, а если $f$ ограничена снизу, то существует предел $\lim_{x\to a+}f(x)$ и $\lim_{x\to a+}f(x) = \inf_{x\in (a,b)} f(x).$
\end{corollary}

\begin{proof}
    Действительно, если $f$ ограничена сверху, то это значит, что множество $\{f(x),\, x\in (a,b)\}$ ограничено сверху, а тогда, согласно принципу полноты Вейерштрасса (=Теорема \ref{W=complete}), существует $M=\sup \{f(x),\, x\in (a,b)\}$. Осталось повторить доказательство предыдущей теоремы, где нужно положить, что $\xi:= b$ и мы получам утверждение. Аналогично доказывается в случае, когда $f$ ограничена снизу.
\end{proof}

\begin{mydanger}{\bf !}
    Следует отметить, то, что в доказательстве теоремы мы использовали то, что множество значений ограничено числом $f(\xi)$, в доказательстве следствия, по условию множество значений ограничено.
\end{mydanger}


\subsection{Признаки сравнения}

\begin{theorem}[Критерий интегрируемости неотрицательной функции]\label{criteria_for_non-negative_on_non-compact}
    Пусть функция $f:[a,b) \to \mathbb{R}$ является неотрицательной на этом промежутке, тогда интеграл $\int_a^b f$ существует тогда и только тогда, когда существует такое число $C>0$, что для всех $ \beta \in [a ,b)$ интегралы $\int_a^\beta f$ существуют и $\int_a^\beta f \le C$.
    
    При выполнении этого условия
    \[
     \int_a^b f = \sup_{\beta \in [a,b)} \int_a^\beta f.
    \]
\end{theorem}

\begin{proof}
    Согласно Утверждению \ref{int=monotonic}, при любом $x\in [a,b)$, функция $F(x) : = \int_a^x f$ является монотонной на промежутке $[a,b)$. По условию функция $F(x)$ ограничена сверху, а тогда, согласно следствию \ref{cor_about_monotonic}, существует предел $\lim_{x \to b-}F(x)$, что и означает существование интегерала $\int_a^b f.$ Далее, согласно теореме \ref{about_monotonic}, $\lim_{x \to b-}F(x) = \sup_{x\in [a,b)} F(x)$, что и доказывает предложение.
\end{proof}

Напомним (см. Определение \ref{O-big})), запись 
$$f=O(g),\qquad x \to x_0$$
означает, что $x_0$ -- точка прикосновения множества $X\subseteq \mathbb{R}$ (см. Определине \ref{limit_point}), $f,g: X \to \mathbb{R}$, для любой окрестности $\mathscr{U}$ точки $x_0$ существует такое число $C>0$, что для всех $x \in \mathscr{U}$ верно неравенство $|f(x)| \le C \cdot |g(x)|$.

\begin{mydanger}{\bf !}
В частности, если $f(x) \le g(x)$ при всех $x \in [a,b)$, и $f,g:[a,b) \to \mathbb{R}$ -- неотрицательны, то $f = O(g)$ при $x \to b.$    
\end{mydanger}




\begin{theorem}[Признак сравнения]
    Пусть $f,g:[a,b) \to \mathbb{R}$ неотрицательные функции и
    \[
     f(x) = O(g(x)), \qquad x \to b.
    \]
Пусть теперь для любого $a\le \beta <b$, они интегрируемы на отрезках $[a,\beta]$, тогда
    \begin{enumerate}
        \item если интеграл $\int_a^b g$ сходится, то сходится и интеграл $\int_a^b f$;
        \item если интеграл $\int_a^b f$ расходится, то расходится и интеграл $\int_a^b g.$
    \end{enumerate}
\end{theorem}

\begin{proof}~\\
(1) Согласно условию, для любого $\delta>0$, существует такое число $C>0$, что для всех $x \in (b- \delta, b)$, $|f(x)| \le C \cdot |g(x)|$, а так как функции не отрицательны, то $f(x) \le C \cdot g(x).$

Далее, так как интеграл $\int_a^b g$ существует, то согласно Лемме \ref{good_lemma_for_nonproper_int}, существуют интегралы $\int_{b-\delta}^\beta g$, а тогда по критерию интегрируемости неотрицательной функции (см. Теорема \ref{criteria_for_non-negative_on_non-compact}), существует такое число $M>0$, что $\int_{b-\delta}^\beta g \le M.$

Так как, по условию, интегралы $\int_a^\beta f$ существуют, то согласно лемме \ref{restriction_of_int}, сущесвтуют интегрылы $\int_{b-\delta}^\beta f$. В силу монотонности и линейности интеграла (см. Теорема \ref{int(f+g)=int(f)+int(g)} и Следствие \ref{int>0}, ), 
\[
 \int_{b-\delta}^\beta f \le C \cdot \int_{b-\delta}^\beta g \le C\cdot M,
\]
таким образом, согласно критерию интегрируемости (=Теорема \ref{criteria_for_non-negative_on_non-compact}), существует интеграл $\int_{b-\delta}^b f$. Наконец, в силу аддитвиности интеграла (см. Теорему \ref{additive_of_int}), 
\[
 \int_a^\beta f = \int_a^{b-\delta} f + \int_{b-\delta}^\beta f,
\]
откуда $\int_{b-\delta}f^\beta f = \int_a^\beta f - \int_a^{b-\delta}f$, и тогда, так как существует предел $\lim_{\beta \to b-} \int_{b-\delta}^\beta f$, то существует и предел $\lim_{\beta \to b-}\int_{b-\delta}^\beta f$, что завершает доказательство пункта 1.

(2) Если интеграл $\int_a^b g$ сходится, то по только что доказанному утверждению должен сходится интеграл $\int_a^b f$, таким образом, если расходится интеграл $\int_a^b f$, то интеграл $\int_a^b g$ тоже расходится.
\end{proof}


\begin{corollary}
    Пусть $f,g:[a,b) \to \mathbb{R}$ положительны, и пусть существует (не обязательно конечный) предел
    \[
     \lim_{x\to b-}\frac{f(x)}{g(x)} = q, 
    \]
    тогда:
    \begin{itemize}
        \item[(1)] если интеграл $\int_a^b g$ сходится и $0\le q < +\infty$, то интеграл $\int_a^b f$ также сходится;
        \item[(2)] если интеграл $\int_a^b g$ расходится и $0< q \le +\infty$, то интеграл $\int_a^b f$ также расходится.
    \end{itemize}
\end{corollary}
\begin{mydangerr}{\bf !}
    В частности, если $f = o(g)$ при $x \to b-$, или $0<q<+\infty$, то характер сходимости у интегралов $\int_a^b f$, $\int_a^b g$ одинковый.
\end{mydangerr}
\begin{proof}~\\
(1) Пусть $0 \le q < +\infty$, тогда в виду того, что $\lim_{x\to b-}\frac{f(x)}{g(x)} = q$, то для любого $\varepsilon>0$ существует такое $\delta >0$, что для всех $x \in (b-\delta, b)$, следует неравенство 
\[
 \left| \frac{f(x)}{g(x)} - q \right|<\varepsilon.
\]

Откуда, $f(x) \le (k+\varepsilon)\cdot g(x)$, что и означает $f = O(g)$ при $x \to b-$, и тогда по теорему выше получаем утверждние следствия.

(2) Пусть $0<q \le +\infty$, тогда обратное соотношение $\frac{f(x)}{g(x)}$, согласно арифметике пределов, имеет конечный предел. Это значит, что по доказанному выше, интеграл $\int_a^b f$ должен расходится, ибо в противном случае будет сходится интеграл $\int_a^b g$ что противоречт предположению. Это завершает доказательство.    
\end{proof}


\section{Признаки Дирихле и Абеля}

\subsection{Признак Дирихле}

\begin{theorem}[Признак Дирихле]
    Пусть:
    \begin{itemize}
         \item[(1)] функция $f:[a,b) \to \mathbb{R}$ непрерывна и имеет ограниченную первоообразную $F$ на $[a,b)$;
        \item[(2)] функци $g:[a,b) \to \mathbb{R}$ непрерывно дифференцируема и монотонна;
        \item[(3)] $\lim_{x \to b-} g(x) = 0$.
    \end{itemize}
    Тогда интеграл $\int_a^b fg$ сходится.
\end{theorem}

\begin{proof}

Будем считать, что функция $g$ убывает, иначе заменим её на $-g$, и в силу линейности интеграла, сведём задачу к нашему случаю.

Согласно условиям, функция $fg$ является непрерывной на каждом $[a,\beta)$, $a \le \beta <b$, поэтому она интегрируема. 

Интегрируем по частям
\[
 \int_a^\beta f(x)g(x) \mathrm{d}x = \int_a^b g(x) \mathrm{d}F(x) = \Bigl. g(x) F(x) \Bigr|_a^\beta - \int_a^\beta F(x) g'(x) \mathrm{d}x.
\]

Найдём предел $\lim_{\beta \to b-}g(x) F(x)$. По условию $F(x)$ ограничена на $[a,b)$, поэтому положим $M: = \sup_{a \le x >b}\{|F(x)|\}$. Следовательно $g(\beta)F(\beta) \le M g(\beta)$, но тогда при $\beta \to b-$, получаем
\[
 \lim_{\beta \to b-} g(\beta) F(\beta) = 0, 
\]
поэтому
\[
\Bigl. g(x) F(x) \Bigr|_a^\beta  = - g(a)F(a),
\]
тогда
\[
 \int_a^\beta f(x)g(x) \mathrm{d}x  =- g(a) F(a) - \int_a^\beta F(x) g'(x) \mathrm{d}x.
\]

Рассмотрим интеграл $\int_a^b |F(x)g(x)| \mathrm{d}x$. Так как $g$ убывает, то $g'\le 0$, поэтому

\begin{eqnarray*}
 \int_a^\beta| F(x) g'(x) |\mathrm{d}x &\le &  M \int_a^\beta |g'(x)| \mathrm{d}x = - M \int_a^\beta |g'(x)| \mathrm{d}x\\
 &=& M(g(a)-g(\beta)).
\end{eqnarray*}

Так как $\lim_{x \to b-} g(x) = 0$, и функция $g(x)$ убывает, то $g(x) \ge 0$. Действительно, если существует такой $x_0$, что $g(x_0) <0$, то в силу убывания функции, для любого $x>x_0$, $g(x) <g(x_0) <0$, но тогда $\lim_{x \to \infty, x\ge x_0} g(x) <0$, что противоречит равенству $\lim_{x \to b-} g(x) = 0$ (см. Теорему \ref{limit_for_any_subset}). Итак, получаем, что $g(x)\ge 0$.

Тогда, в частности, $g(\beta) \ge 0$, и тогда
\[
 \int_a^\beta| F(x) g'(x) |\mathrm{d}x \le M g(a).
\]

Наконец, согласно критерию интегририуемости неотрицательной функций (=теорема \ref{criteria_for_non-negative_on_non-compact}), интеграл сходится, то есть интеграл $\int_a^\beta F(x) g'(x) \mathrm{d}x$ сходится абсолютно, а тогда, он сходится.
\end{proof}


\subsection{Признак Абеля}


\begin{theorem}[Признак Абеля]
    Пусть даны две функции $f,g:[a,b) \to \mathbb{R}$, при этом
    \begin{itemize}
        \item[(1)] функция $f$ непрерывна и сходится интеграл $\int_a^b f$,
        \item[(2)] функция $g$ непрерывно дифференцируема, ограничена и монотона,
    \end{itemize}
    тогда интеграл $\int_a^b fg$ сходится.
\end{theorem}

\begin{mydanger}{\bf!}
    Признак Абеля вытекает из признака Дирихле.
\end{mydanger}

\begin{proof}
    Так как функция $f$ имеет первообразную на $[a,b)$, $F(x) = \int_a^b f$, тогда, согласно первой фундаментальной теореме (см. Теорема \ref{the_first_fundamental_theorem}), $F(x)$ непрерывна, а так как существует предел $\lim_{x \to b-}F(x)=:\int_a^b f$, то, согласно определению предела (см. Определение \ref{the_main_def_of_limit_on_R})), функция $F(x)$ непрерывна на отрезке $[a,b]$,
    \[
     F(x) : = \begin{cases}
         \int_a^xf, & x\in [a,b)\\
         \int_a^b f, & x = b.
     \end{cases}
     \]

     Так как $[a,b]$ компактен\footnote{Обратите внинмание, что у нас по прежнему $[a,b] \subseteq \overline{\mathbb{R}}$, \textit{т.е.,} $b$ может принимать значение $b = + \infty$, но так как на $\overline{\mathbb{R}}$ вводится топология из отрезка $[-1,1]$ (см. секцию "Окрестность бесконечности" \ref{neighborhood_of_infinity}), то все рассуждения про компактность для отрезков из $\mathbb{R}$ становяться такими же как и для отрезков в $\overline{\mathbb{R}}$.} то, функция $F(x)$ -- ограничена на нём (см. Теорему \ref{continous_on_interval_on_R}).

     Далее, так как $g$ монотонна и ограничена, то согласно следствию \ref{cor_about_monotonic}, существует предел $\lim_{x \to b-} g=: B$. В таком случае, функция $\widetilde{g}: = g - B$ непрервыно дифференцируема и монотонна на $[a,b)$ и $\lim_{x \to b-}\widetilde{g}=0 $, поэтому, согласно признаку Дирихле, интеграл $\int_a^b f\widetilde{g}$ сходится, а так
     \[
      \int_a^b fg = \int_a^b f\widetilde{g} + B\int_a^bf, 
     \]
     то интеграл $\int_a^b fg$ сходится. Это завершает доказательство.
\end{proof}