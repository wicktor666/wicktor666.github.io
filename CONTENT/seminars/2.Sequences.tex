\section{Последовательности и их пределы}

\begin{figure}[h!]
    \centering
\begin{tikzpicture}[
    dot/.style={circle,fill=black,inner sep=1.5pt},
    label distance=1mm,
    brace/.style={decoration={brace,amplitude=5pt,raise=2pt},decorate}
]

% Числовая прямая
\draw[->,thick] (-5,0) -- (5,0) node[right] {$\mathbb{R}$};
\foreach \x in {-4,-3,-2, ,2,3,4} {
    \draw (\x,0.1) -- (\x,-0.1);
}

% Точка предела и ε-окрестность
\node[dot,label=above:{$a$}] (a) at (0,0) {};
\draw[red!50,line width=10pt,opacity=0.2] (-1,0) -- (1,0);
\draw[<->,red] (-1,1.5) -- node[above,midway] {$\varepsilon$-окрестность} (1,1.5);
\draw[dashed,red] (-1,-0.5) -- (-1,1) node[above] {$a-\varepsilon$};
\draw[dashed,red] (1,-0.5) -- (1,1) node[above] {$a+\varepsilon$};

% Элементы вне окрестности (над осью)
\node[dot,label=above:{$x_1$}] (x1) at (-4,0.5) {};
\draw[dotted] (x1) -- (-4,0);
\node[dot,label=above:{$x_2$}] (x2) at (3.5,0.5) {};
\draw[dotted] (x2) -- (3.5,0);
\node[dot,label=above :{$x_3$}] (x3) at (-2.7,0.5) {};
\draw[dotted] (x3) -- (-2.7,0);
\node[dot,label=above:{$x_4$}] (x4) at (1.8,0.5) {};
\draw[dotted] (x4) -- (1.8,0);

% Элементы внутри окрестности (под осью)
\foreach \x/\i in {-0.8/5,-0.4/6,0.3/7,0.7/8,0.1/9,-0.6/10,0.5/11} {
    \node[dot] (x\i) at (\x,-0.3) {};
     \draw[dotted] (x\i) -- (\x,0);
}

% Стрелка и пояснение для элементов внутри
\draw[->,green!50!black,thick] (0,-1) -- (0,-0.4) node[midway,right] {Все элементы $x_n$, $n \geqslant 5$};
\end{tikzpicture}
\caption{Если $a$ -- предел последовательности, то для любой её $\varepsilon$-окрестности, можно всегда найти такое число $N$, что \textbf{все} элементы последовательности $x_N, x_{N+1},\ldots, $ попадут в эту окрестность. Если же хотя бы один среди $x_N, x_{N+1},\ldots,$ не будет входит в эту окрестноть, то $a$ тогда не будет пределом.}
\end{figure}


\begin{problem}
 Покажем, что последовательность $x_n = (-1)^n$ не имеет предела.
\end{problem}

\begin{proof}[Решение]


::::{figure}
     :::{tikzpicture}[dot/.style={circle,fill=black,inner sep=1.5pt},
     every label/.style={black}, brace/.style={decoration={brace,amplitude=5pt,raise=2pt},decorate}
]

% Числовая прямая
\draw[->,thick] (-3,0) -- (3,0) node[right] {$\mathbb{R}$};
\foreach \x in {-2,-1,0,1,2} {
    \draw (\x,0.1) -- (\x,-0.1) node[below] {$\x$};
}

% Элементы последовательности (разнесены по вертикали для наглядности)
% Нечетные n = 1,3,5...
\node[dot,label=above left:{\color{blue}$x_1$}] (x1) at (-1,0.8) {};
\node[dot,label=above left:{\color{blue}$x_3$}] (x3) at (-1,1.3) {};
\node[dot,label=above left:{\color{blue}$x_5$}] (x5) at (-1,1.8) {};
\draw[dashed,blue] (-1,0) -- (-1,2);

% Четные n = 2,4,6...
\node[dot,label=above right:{\color{red}$x_2$}] (x2) at (1,0.8) {};
\node[dot,label=above right:{\color{red}$x_4$}] (x4) at (1,1.3) {};
\node[dot,label=above right:{\color{red}$x_6$}] (x6) at (1,1.8) {};
\draw[dashed,red] (1,0) -- (1,2);

% Подписи значений
\node[blue,left] at (-1,2.2) {$-1$};
\node[red,right] at (1,2.2) {$1$};

% Демонстрация проблемы для произвольного a
% Случай 1: a = 0
\draw[green!50,line width=8pt,opacity=0.3] (-0.5,0) -- (0.5,0);
\draw[<->,green!70!black] (-0.5,-1) -- node[below,midway] {$\varepsilon$-окрестность $a=0$} (0.5,-1);
\draw[dashed,green!70!black] (-0.5,0) -- (-0.5,-0.8);
\draw[dashed,green!70!black] (0.5,0) -- (0.5,-0.8);

% Случай 2: a = 0.5
\draw[orange!50,line width=8pt,opacity=0.3] (0,0) -- (1,0);
\draw[<->,orange!70!black] (0,-1.5) -- node[below,midway] {$\varepsilon$-окрестность $a=0.5$} (1,-1.5);
\draw[dashed,orange!70!black] (0,0) -- (0,-1.3);
\draw[dashed,orange!70!black] (1,0) -- (1,-1.3);

% Поясняющие стрелки
\draw[red,->,thick] (0.5,-0.8) -- (1,0.5);
\draw[blue,->,thick] (-0.5,-0.8) -- (-1,0.5);
\draw[orange,->,thick] (0,-1.3) -- (-1,0.8);

% Поясняющий текст
%\node[text width=8cm, align=center] at (0,-2.5) {
 %   Последовательность $x_n = (-1)^n$ \textbf{не имеет предела}, так как: \\
  %  $\bullet$ Для любого кандидата $a$ выберем $\varepsilon < 1$ \\
   % $\bullet$ Всегда существуют элементы \textcolor{blue}{далеко от $a$} (бесконечно много!) \\
    %$\bullet$ Пример 1: При $a=0$ все \textcolor{blue}{нечетные} члены вне $\varepsilon$-окрестности \\
    %$\bullet$ Пример 2: При $a=0.5$ все \textcolor{blue}{нечетные} члены вне $\varepsilon$-%окрестности
%};

:::
  ::::
\end{proof}
