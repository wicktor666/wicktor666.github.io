\section{Занятие \#1}

\subsection{Рациональные числа}

\begin{problem}
 Докажите, что число \(\sqrt{2} + \sqrt{3}\) иррационально.
\end{problem}

\begin{proof}[Решение]~

(1) Первый способ (от противного). 
Предположим, что \(\sqrt{2} + \sqrt{3}\) рационально. Тогда:
\[
\sqrt{2} + \sqrt{3} = r, \quad r \in \mathbb{Q}
\]
Возведем в квадрат:
\[
(\sqrt{2} + \sqrt{3})^2 = r^2 \implies 2 + 2\sqrt{6} + 3 = r^2 \implies 5 + 2\sqrt{6} = r^2
\]
Выразим \(\sqrt{6}\):
\[
\sqrt{6} = \frac{r^2 - 5}{2}
\]
Так как \(r\) рационально, то правая часть рациональна, но \(\sqrt{6}\) иррационально. Противоречие.

(2) Второй способ (через обратную величину).

Заметим:
\[
(\sqrt{2} + \sqrt{3})(\sqrt{3} - \sqrt{2}) = 3 - 2 = 1
\]
Предположим \(\sqrt{2} + \sqrt{3} = r \in \mathbb{Q}\). Тогда:
\[
\sqrt{3} - \sqrt{2} = \frac{1}{r} \in \mathbb{Q}
\]
Сложим два равенства:
\[
(\sqrt{2} + \sqrt{3}) + (\sqrt{3} - \sqrt{2}) = r + \frac{1}{r} \implies 2\sqrt{3} = r + \frac{1}{r}
\]
Отсюда \(\sqrt{3} = \frac{r^2 + 1}{2r} \in \mathbb{Q}\), но \(\sqrt{3}\) иррационально. Противоречие.

\end{proof}

\subsection{Бином Ньютона}


Напомним формулы бинома Ньютона.
\[
(a + b)^n = \sum_{k=0}^{n} C_n^k a^{n-k} b^k = \sum_{k=0}^{n} \binom{n}{k} a^{n-k} b^k
\]
где \(\binom{n}{k} = \frac{n!}{k!(n-k)!}\) -- биномиальный коэффициент.


\begin{figure}[h!]
    \centering
     \begin{tikzpicture}[scale=0.7]
\foreach \n in {0,...,5} {
  \foreach \k in {0,...,\n} {
    \node at (2*\k - \n, -\n) {\(\binom{\n}{\k}\)};
  }
}
\end{tikzpicture}
    \caption{Биномиальные коэффициенты удобно располагать в треугольнике Паскаля}
\end{figure}

\begin{mydanger}{\bf !}
При закрашивании нечётных чисел в треугольнике Паскаля возникает фрактальная структура, аналогичная треугольнику Серпинского.    
\end{mydanger}

\begin{problem}
    Раскройте скобки в выражении \((a + b)^7\).
\end{problem}

\begin{proof}[Решение]
\[
(a + b)^7 = \sum_{k=0}^{7} \binom{7}{k} a^{7-k} b^k
\]
Вычислим коэффициенты:
\begin{align*}
\binom{7}{0} &= 1, &\binom{7}{1} &= 7, &\binom{7}{2} &= 21, \\
\binom{7}{3} &= 35, &\binom{7}{4} &= 35, &\binom{7}{5} &= 21, \\
\binom{7}{6} &= 7, &\binom{7}{7} &= 1.
\end{align*}
Результат:
\[
(a + b)^7 = a^7 + 7a^6b + 21a^5b^2 + 35a^4b^3 + 35a^3b^4 + 21a^2b^5 + 7ab^6 + b^7
\]    
\end{proof}


\begin{problem}
    Найдите коэффициент при \(x^3\) в выражении \(\left( \sqrt{x} + \frac{1}{\sqrt[3]{x}} \right)^{16}\). 
\end{problem}
\begin{proof}[Решение]
Преобразуем:
\[
\sqrt{x} = x^{1/2}, \quad \frac{1}{\sqrt[3]{x}} = x^{-1/3}
\]
Общий элемент разложения:
\[
T_k = \binom{16}{k} (x^{1/2})^{16-k} (x^{-1/3})^k = \binom{16}{k} x^{\frac{16-k}{2} - \frac{k}{3}}
\]
Упростим показатель:
\[
\frac{16-k}{2} - \frac{k}{3} = \frac{3(16-k) - 2k}{6} = \frac{48 - 5k}{6}
\]
Решим уравнение:
\[
\frac{48 - 5k}{6} = 3 \implies 48 - 5k = 18 \implies 5k = 30 \implies k = 6
\]
Тогда, искомый коэффициент:
\[
\binom{16}{6} = \frac{16!}{6!10!} = \frac{16×15×14×13×12×11}{6×5×4×3×2×1} = \frac{5765760}{720} = 8008
\]    
\end{proof}


\subsection{Вычисление некоторых сумм}

\begin{problem}
    Вычислить сумму \(1 + 2 + \cdots + 100\).
\end{problem}

\begin{proof}[Решение]
Сгруппируем слагаемые:
\[
(1 + 100) + (2 + 99) + \cdots + (50 + 51)
\]
Каждая пара дает сумму 101, количество пар равно 50, и тогда искомая сумма находится так
\[
S = 50 \times 101 = 5050.
\]    
\end{proof}


\begin{problem}
    Вычислить сумму \(1 + q + q^2 + \cdots + q^n\), где \(q \neq 1\)
\end{problem}
\begin{proof}[Решение]
Обозначим \(S_n = \sum_{k=0}^{n} q^k\). Умножим на \(q\):
\[
qS_n = q + q^2 + \cdots + q^{n+1}
\]
Вычтем из исходного:
\[
S_n - qS_n = 1 - q^{n+1} \implies S_n(1 - q) = 1 - q^{n+1}
\]
\[
S_n = \frac{1 - q^{n+1}}{1 - q}
\]    
\end{proof}


\begin{problem}
 Вычислить сумму \(\sum_{k=1}^{n} \frac{1}{k(k+1)}\).
\end{problem}
\begin{proof}[Решение]
 Разложим на простейшие дроби:
\[
\frac{1}{k(k+1)} = \frac{A}{k} + \frac{B}{k+1} \implies 1 = A(k+1) + Bk
\]
При \(k=0\): \(A = 1\), при \(k=-1\): \(B = -1\). Значит:
\[
\frac{1}{k(k+1)} = \frac{1}{k} - \frac{1}{k+1}
\]
Сумма телескопическая:
\[
\sum_{k=1}^{n} \left( \frac{1}{k} - \frac{1}{k+1} \right) = \left(1 - \frac{1}{2}\right) + \left(\frac{1}{2} - \frac{1}{3}\right) + \cdots + \left(\frac{1}{n} - \frac{1}{n+1}\right) = 1 - \frac{1}{n+1} = \frac{n}{n+1}
\]    
\end{proof}


\begin{problem}
    Оцените сверху сумму \(\sum_{k=1}^{n} \frac{1}{k^2}.\) 
\end{problem}
\begin{proof}[Решение]
Для \(k \geq 2\) верно:
\[
\frac{1}{k^2} < \frac{1}{k(k-1)} = \frac{1}{k-1} - \frac{1}{k}
\]
Тогда:
\[
\sum_{k=1}^{n} \frac{1}{k^2} = 1 + \sum_{k=2}^{n} \frac{1}{k^2} < 1 + \sum_{k=2}^{n} \left( \frac{1}{k-1} - \frac{1}{k} \right)
\]
Телескопическая сумма:
\[
\sum_{k=2}^{n} \left( \frac{1}{k-1} - \frac{1}{k} \right) = \left(1 - \frac{1}{2}\right) + \left(\frac{1}{2} - \frac{1}{3}\right) + \cdots + \left(\frac{1}{n-1} - \frac{1}{n}\right) = 1 - \frac{1}{n}
\]
Итого:
\[
S_n < 1 + \left(1 - \frac{1}{n}\right) = 2 - \frac{1}{n} < 2
\]    
\end{proof}

\subsection{Метод математической индукции}


\begin{problem}
Используя метод математической индукции доказать, что 
\[
1 + 2 + \cdots + n = \frac{n(n+1)}{2}
\]
\end{problem}
\begin{proof}[Решение]~\\
\textbf{База.} Если \(n=1\) то \(1 = \frac{1 \cdot 2}{2} = 1\), что верно. \\
\textbf{Предположение.} Допустим, что для \(n=k \ge 1\) мы уже доказали \(\sum_{i=1}^{k} i = \frac{k(k+1)}{2}\) \\
\textbf{Шаг.} Пусть теперь \(n=k+1\), тогда имеем
\[
\sum_{i=1}^{k+1} i = \sum_{i=1}^{k} i + (k+1) = \frac{k(k+1)}{2} + (k+1) = (k+1) \left( \frac{k}{2} + 1 \right) = \frac{(k+1)(k+2)}{2}.
\]    
\end{proof}


\begin{problem}
 Используя метод математической индукции докажите неравенство Бернулли
 \[
  (1+x)^n \geq 1 + nx, \qquad x > -1.
 \]
\end{problem}
 
\begin{proof}[Решение]~\\
\textbf{База.} Пусть \(n=1\), тогда \(1+x \geq 1+x\), что верно всегда.\\
\textbf{Предположение.} Допустим, что неравенство доказано при \(n=k\ge 1\)\textit{т.е.,} \((1+x)^k \geq 1 + kx\).\\
\textbf{Шаг} Пусть теперь \(n=k+1\), тогда получаем
\[
(1+x)^{k+1} = (1+x)^k (1+x) \geq (1 + kx)(1+x) = 1 + x + kx + kx^2 = 1 + (k+1)x + kx^2
\]
Так как \(kx^2 \geq 0\), то:
\[
(1+x)^{k+1} \geq 1 + (k+1)x
\] 
что и завершает доказательство.
\end{proof}

\begin{problem}
 Методом индукции докажите, что \(a_1 + \cdots + a_n \geq n\), при \(a_1 \cdots a_n = 1\), \(a_i > 0.\)
\end{problem}
\begin{proof}[Решение]~\\
\textbf{База.} Если \(n=1\), то \(a_1 = 1 \geq 1\) что верно всегда. \\
\textbf{Предположение.} Допустим, что утверждение верно при \(n=k \ge 1\). \\
\textbf{Шаг.} Пусть теперь \(n=k+1\). Если все \(a_i = 1\), то неравенство выполнено. Иначе же найдутся \(a_j < 1 < a_m\) (т.к. произведение равно 1). Без ограничения общности пусть \(a_k < 1 < a_{k+1}\). Положим \(b_k := a_k a_{k+1}\), тогда \(b_1 \cdots b_k = 1\) где \(b_i := a_i\) при \(i<k\). По предположению индукции:
\[
\sum_{i=1}^{k-1} b_i + b_k \geq k
\]
Имеем
\[
\sum_{i=1}^{k+1} a_i = \sum_{i=1}^{k-1} b_i + a_k + a_{k+1} \geq k - b_k + a_k + a_{k+1} = k + (a_k + a_{k+1} - a_k a_{k+1})
\]
Докажем \(a_k + a_{k+1} - a_k a_{k+1} \geq 1\). Действительно,
\[
a_k + a_{k+1} - a_k a_{k+1} - 1 = (1 - a_k)(a_{k+1} - 1) \geq 0
\]
так как \(a_k < 1 < a_{k+1}\). Это и доказывает утверждение.  
\end{proof}

\begin{problem}
    Методом математической индукции докажите неравенство о средних 
    \[
\frac{a_1 + \cdots + a_n}{n} \geq \sqrt[n]{a_1 \cdots a_n}, \quad a_i > 0
\]
\end{problem}

\begin{proof}[Решение]~\\
\textbf{База.} Если \(n=2\) то \(\frac{a_1+a_2}{2} \geq \sqrt{a_1 a_2} \iff (\sqrt{a_1} - \sqrt{a_2})^2 \geq 0\) что верно всегда. \\
\textbf{Предположение.} Пусть неравенство доказано, при \(n=k\ge 1\). \\
\textbf{Шаг.} Пусть теперь \(n=k+1\). Положим \(g := \sqrt[k+1]{a_1 \cdots a_{k+1}}\), и рассмотрим \(b_i := a_i / g\), но в таком случае получаем \( b_1 \cdots b_{k+1} = 1\). По предыдущей задаче имеем
\[
\sum_{i=1}^{k+1} b_i \geq k+1 \implies \sum_{i=1}^{k+1} \frac{a_i}{g} \geq k+1 \implies \sum_{i=1}^{k+1} a_i \geq (k+1) g = (k+1) \sqrt[k+1]{a_1 \cdots a_{k+1}}
\]
Делим обе части на \(k+1\)
\[
\frac{1}{k+1} \sum_{i=1}^{k+1} a_i \geq \sqrt[k+1]{a_1 \cdots a_{k+1}}
\]    
что и доказывает неравенство.
\end{proof}

\begin{problem}
 Докажите методом математической индукции неравенство 
 $$2^{n-1} \leq n! \leq \left(\frac{n+1}{2}\right)^n.$$
\end{problem}

\begin{proof}[Решение]~ \\
\textbf{База.} Пусть \(n=1\), тогда получаем \(2^0 = 1 \leq 1! = 1 \leq (1)^1 = 1\) что верно всегда. \\
\textbf{Предположение.} Пусть мы доказали в случае \(n=k\ge 1\), \textit{т.е.,} пусть \(2^{k-1} \leq k! \leq \left(\frac{k+1}{2}\right)^k\) \\
\textbf{Шаг.} Теперь, допустим \(n=k+1\), тогда для левой части
\[
(k+1)! = (k+1)k! \geq (k+1) 2^{k-1} \geq 2 \cdot 2^{k-1} = 2^k \quad (\text{ т.к. } k+1 \geq 2 \text{ при } k \geq 1),
\]
а для правой части
\[
(k+1)! = (k+1)k! \leq (k+1) \left(\frac{k+1}{2}\right)^k = \frac{(k+1)^{k+1}}{2^k}
\]

Нам достаточно показать что
\[
\frac{(k+1)^{k+1}}{2^k} \leq \left(\frac{k+2}{2}\right)^{k+1} \iff (k+1)^{k+1} \cdot 2 \leq (k+2)^{k+1}
\]
\[
\iff \left(1 + \frac{1}{k+1}\right)^{k+1} \geq 2
\]

Но, последнее верно, т.к. последовательность \(\left(1 + \frac{1}{m}\right)^m\) возрастает к \(e > 2\), и при \(m=2\): \((1.5)^2 = 2.25 > 2\).
    
\end{proof}

