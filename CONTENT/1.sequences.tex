\chapter{Действительные числа и последовательности}

\section*{Предварительные знания}

Мы всюду в дальнейшем будем пользоваться Аксиомой выбора, которую сформулируем в таком виде:

Пусть $X$ -- множество каких-то непустых множеств. \textit{Функция выбора} определена на $X$ и при этом для любого $A \in X$, $f(A)$ -- элемент из $A$. 

\begin{mydanger}{\bf{!}}
Другими словами, функция выбора позволяет выбрать какой-то элемент из множества $A.$
\end{mydanger}~

\textbf{Аксиома выбора утверждает},\label{AC}
 \textit{что для любого семейства непустых множеств $X$ существует функция выбора $f$, определённая на $X$}.\\    

\begin{mydanger}{\bf{!}}
 Менее формально это означает следующее: пусть у нас есть множество $X$ непустых множеств $A_\alpha$, где $\alpha$ пробегает какое-то непустое множество индексов $I.$ Тогда аксиома выбора утверждает, что в каждом $A_\alpha$ можно выбрать элемент из $A_\alpha.$
\end{mydanger}

Если дано множество $X$ и есть какое-то семейство $\{Y_\alpha\}_{\alpha \in A}$ каких-то подмножеств множества $X$, где $A$ какое-то множество индексов (необязательно счётное), то имеют место следующие формулы (=формулы де Моргана) 
\begin{eqnarray}
    X \setminus \bigcup_\alpha Y_\alpha &=& \bigcap_\alpha (X \setminus Y_\alpha), \label{dM1}\\
    X \setminus \bigcap_\alpha Y_\alpha &=& \bigcup_\alpha (X \setminus Y_\alpha). \label{dM2}
\end{eqnarray}


\section{Лекция \#1. Действительные числа}

Определение чисел обычно начинается с того, что определяется множество $\mathbb{N}$ \textit{натуральных чисел}. Натуральные числа -- это числа, которые появляются при счёте. Таким образом, $\mathbb{N} = \{1,2,\ldots,\}$. Далее определяется множество \textit{целых чисел} $\mathbb{Z}$ как множество полученное из $\mathbb{N}$ с добавлением $0$ и противоположных по знаку чисел. Таким образом,
\[
\mathbb{Z} = \{\ldots, -2,-1,0,1,2,\ldots,\}.
\]

Наконец, мы можем ввести следующее множество $\mathbb{Q}$ как множество всех дробей вида $\frac{p}{q}$, где $p\in\mathbb{Z}$, $q \in \mathbb{N}$. При этом мы считаем, что $\frac{p}{q} = \frac{a}{b}$ если и только если $aq = bp$. Множество $\mathbb{Q}$ называют \textit{множеством рациональных чисел.}

Множество действительных чисел приходится либо строить (сечения Дедекинда), либо определять аксиоматически. Мы выбираем второй путь, но мы также предъявляем модель для них.

Напомним, что если даны два множества $A, B$, то можно образовать декартово произведение $A\times B: = \{(a,b)\,|\, a\in A, b\in B\}$, то есть множество всех пар.

\subsection{Аксиомы действительных чисел}

Введём следующее определение.

\begin{definition}\label{field}
 Поле действительных чисел есть множество $\mathbb{R}$, для которого определены:
\begin{enumerate}
    \item два отображения $(x,y)\mapsto x+y$ и $(x,y) \mapsto xy$ из $\mathbb{R}\times \mathbb{R}$ в $\mathbb{R}$;
    \item отношение $x \le y$ (записываемое также в виде $y \ge  x$) между элементами множества $\mathbb{R},$ удовлетворяющее следующим группам аксиом:
     \begin{itemize}
         \item[i)] Множество $\mathbb{R}$ есть \textit{поле}, иными словами:
         \begin{itemize}
             \item[i.1] $x+ (y+z) = (x+y)+z;$
             \item[i.2] $x+y = y+x;$
             \item[i.3] существует такой элемент $0 \in \mathbb{R}$, что $0+x= x$ для каждого $x \in \mathbb{R};$
             \item[i.4] для каждого элемента $x\in \mathbb{R}$ существует такой элемент $-x \in \mathbb{R}$, что $x+(-x) = 0$;
             \item[i.5] $x(yz) = (xy)z;$
             \item [i.6] $xy = yx;$
             \item[i.7] в $\mathbb{R}$ существует такой элемент $1 \ne 0$, что $1 x = x$ для каждого $x\in \mathbb{R};$
             \item[i.8] для каждого элемента $x \ne 0$ в $\mathbb{R}$ существует такой элемент $x^{-1} \in \mathbb{R}$, что $xx^{-1} = 1;$
             \item[i.9] $x(y+z) = xy + xz$.
         \end{itemize}
         \item[ii)] Множество $\mathbb{R}$ есть \textit{упорядоченное поле}. Это означает, что выполняются следующие аксиомы:
          \begin{itemize}
              \item [ii.1] из $x\le y$ и $y\le z$ следует $x\le z$;
              \item [ii.2] отношение $x\le y$, $y\le x$ эквивалентно отношению $x = y$;
              \item[ii.3] для любых двух элементов $x,y$ множества $\mathbb{R}$ или $x\le y$, или $y \le x$;
              \item[ii.4] из $x \le y$ следует $x+z \le y+z$;
              \item[ii.5] из $0 \le x$ и $0 \le y$ следует $0 \le xy$.
          \end{itemize}
          \item[iii)] Множество $\mathbb{R}$ удовлетворяет \textit{аксиоме полноты} (=непрерывности):
          если $A,B\subseteq \mathbb{R}$, $A,B \ne \varnothing$, при этом для всех $a\in A$, $b\in B$ выполняется $a \le b$ (мы это запишем как $A\le B$ и будем говорить, что \textit{$A$ лежит левее $B$}), то существует $c \in \mathbb{R}$ такой, что $a\le c \le b$ для всех $a\in A$, $b\in B$ (говорят что $c$ \textit{разделяет} эти два множества).
     \end{itemize}
\end{enumerate}
    
\end{definition}


Говоря неформальным языком, аксиома полноты говорит о том, что во множестве $\mathbb{R}$ нет дыр. 

Покажем, что во множестве $\mathbb{Q}$ аксиома полноты не выполнена (то есть там дыры есть).

\begin{proposition}
    Множество $\mathbb{Q}$ есть упорядоченное поле, в котором не выполняется аксиома полноты.
\end{proposition}
\begin{proof}
    Выполнение аксиом упорядоченного поля элементарно, и доказывать мы этого не будем. Покажем, что аксиома полноты нарушена. Для этого нам нужно предъявить два множества, скажем, $A,B$, таких, что $A \le B$, но при этом не существует $c\in \mathbb{Q}$, что $a\le c \le b$ для всех $a\in A$, $b\in B.$

    Пусть $A: = \{a \in \mathbb{Q}\, | \, a >0, a^2 <2\}$ и $B: = \{b \in \mathbb{Q}\, |\, b>0, b^2 >2\}$. Покажем, что $A\le B$. Действительно, если $a\in A$, $b \in B$, то тогда $a^2 <2$, $b^2 >2$, то есть мы можем записать $a^2<2<b^2$. Это значит, что $a^2<b^2$ или, что равносильно, $b^2-a^2>0$, $(b-a)(b+a)>0$. Так как $a,b >0$, то $a+b \ne 0$, а тогда, разделив на $a+b$ обе части неравенства $(b-a)(b+a)>0$, получаем, что $b-a >0$, то есть $a<b$. Это и показывает, что $A<B$.

    Далее, допустим, что существует $c \in \mathbb{Q}$, который разделяет эти множества, то есть $a\le c \le b$ для всех $a\in A$, $b\in B.$

    Имеем всего три варианта: (1) $c^2 = 2$, (2) $c^2 <2$, (3) $c^2 >2$. Рассмотрим каждый из них.
    
(1) Пусть $c^2 =2$ и пусть $c\in \mathbb{Q}$. Тогда можно записать $c = \frac{p}{q}$, где $p \in \mathbb{Z}$, $q \in \mathbb{N}$, и $(p,q) = 1$. Тогда получаем равенство $p^2 = 2q^2$, которое влечёт чётность числа $p$, \textit{т.е.,} $p = 2k$. Подставляя, получаем, что $q^2 = 2k^2$, что влечёт чётность числа $q$, то есть $q = 2r$, но тогда это значит, что $(p,q)$ делится на $2$, что противоречит предположению $(p,q) =1$. Таким образом, если $c^2 = 2$, то $c\notin \mathbb{Q}$.\\

\centerline{
{\color{red}Это означает, что если $c^2=2$, то $c$ \textbf{не разделяет эти два множества.}}}~
    
(2) Пусть $c^2 <2$. Наша цель -- показать, что такое невозможно. Для этого мы предъявим такое положительное рациональное число $h\in \mathbb{Q}$, что $(c+h)^2 <2$. Если мы найдём такое число $h$, то это будет означать, что, во-первых, $c+h \in A$ и, во-вторых, $c < c+h <b$ для каждого $b \in B$, потому что $c+h \in A$, а мы уже показали, что $A \le B$. Но в таком случае выходит, что $c$ не разделяет множества $A,B$.

Неравенство $(c+h)^2<2$ можно переписать как $2ch + h^2 < 2- c^2$. Пусть $0<h<1$, \textit{т.е.,} $h \in (0,1) \cap \mathbb{Q}$, тогда $2c+h < 2c+1$, и, так как $h >0$, то $h(2c+h)<h(2c+1)$. Поэтому, если мы нашли такое $h \in (0,1) \cap \mathbb{Q}$, что $h(2c+1) < 2-c^2$, то из цепочки неравенств
\[
 2ch +h^2 = h(2c+h) < h(2c+1) < 2-c^2
\]
вытекает, что $h(2с+1) < 2-c^2.$

Итак, нам осталось предъявить такое $h$. Например, можно положить, что
\[
 h : = \frac{2-c^2}{2c+1}\cdot \frac{1}{10^N},
\]
где $N$ сколь угодно большое натуральное число (большое $N$ нужно, чтобы добиться условия $h<1$ при конкретном $c$).

В таком случае, мы получаем
\[
 h(2c+1) = \frac{2-c^2}{2c+1}\cdot \frac{1}{10^N} \cdot (2c+1) = \frac{2-c^2}{10^N} < 2-c^2.
\]

Очевидно, что $h \in \mathbb{Q}.$

\begin{mydanger}{\bf !}
    Таким образом, мы показали, что какое бы $c$, для которого $c^2<2$, мы бы не взяли, \textbf{всегда} можно найти такое $h\in \mathbb{Q}$, что $(c+h)^2<2$. 
\end{mydanger}~

\centerline{
{\color{red}Это означает, что если $c^2<2$, то $c$ \textbf{не разделяет эти два множества.}}}~

(3) Пусть теперь $c^2>2$. Покажем, что в таком случае можно найти такое $k\in \mathbb{Q}$, $k>0$, что $(c-k)^2 >2$. Если мы такое $k$ найдём, то это будет означать, что $c-k \in B$, а так как $A\le B$, то учитывая $c-k < c$, из цепочки неравенств
\[
 a< c-k < c 
\]
будет вытекать, что $c$ не разделяет множества $A$ и $B.$

Имеем
\begin{eqnarray*}
    (c-k)^2 >2 &\Longleftrightarrow& c^2 - 2ck + k^2 >2 \\
     &\Longleftrightarrow& k^2 - 2ck >2-c^2 \\
     &\Longleftrightarrow& k(k-2c) > 2 -c^2 \\
     &\Longleftrightarrow& k(2c - k) < c^2 - 2.
\end{eqnarray*}

С другой стороны, так как $k^2 >0$, то
\[
 k(2c-k) = 2ck - k^2 < 2ck.
\]

Поэтому, если мы найдём такой $k\in \mathbb{Q}$, что $2ck < c^2 - 2$, то оно подойдёт для наших целей, ведь в таком случае мы получаем цепочку неравенств
\[
 k(2c-k)<2ck < c^2-2. 
\]

Можно положить
\[
 k := \frac{c^2-2}{2c}\cdot \frac{1}{10^M},
\]
где $M$ можно взять сколь угодно большим. Таким образом, мы видим, что $c$ не может разделять элементы из $A$ и $B$, так как $a<c-k<c$, но $c-k \in B$.\\ 

\centerline{
{\color{red}Это означает, что если $c^2>2$, то $c$ \textbf{не разделяет эти два множества.}}}
\end{proof}

Из этого утверждения вытекает следующее:

\begin{corollary}
    В поле $\mathbb{R}$ существует такой $x>0$, что $x^2 = 2$. Такое число обозначается как $\sqrt{2}.$
\end{corollary}
\begin{proof}
 Рассмотрим такие же подмножества $A,B$ как и в предыдущем доказательстве, но уже будем их рассматривать в множестве $\mathbb{R}$. По аксиоме полноты должно существовать такое $c\in \mathbb{R}$, которое их разделяет.  Мы видели, что условия $c^2<2$, $c^2>2$ означают, что такое $c$ разделять их не может. Значит, остаётся только одна возможность, когда $c^2 =2$, что и доказывает требуемое.
\end{proof}

\subsection{Бесконечные десятичные дроби}\label{model_of_R}

Здесь мы предъявим модель для множества действительных чисел, но мы не будем объяснять как они складываются и умножаются, так как для наших нужд в этом вообще нет необходимости.


\begin{definition}
    Бесконечная десятичная дробь -- это последовательность вида 
    \[
     \m{a}:=\alpha_0,\alpha_1, \alpha_2, \alpha_3\ldots,
    \]
где $\alpha_0 \in \mathbb{Z}$, $\alpha_1,\alpha_2,\ldots, \in \{0,1,\ldots, 9\}$. При этом дробь, у которой, начиная с какого-то номера (после запятой), встречаются только $9$, запрещены.

Общепринятое обозначение таких дробей: $\overline{\alpha_0,\alpha_1\alpha_2\alpha_3\ldots}$
\end{definition}



\begin{mydanger}{\bf{!}}
    Причина такого запрета заключается в том, что $0,99999\ldots = 1$. Действительно, пусть $x = 0,9999999\ldots$, тогда $10x = 9,999999999\ldots = 9 + x$, откуда $x=1$.
\end{mydanger}

Для удобства, мы будем такие дроби записывать следующим образом
\[
 \m{a} = (\alpha_0 \,|\, \alpha_1, \alpha_2,\ldots,)
\]

Сравниваются две дроби, скажем, $\m{a} = (\alpha_0 \,|\, \alpha_1, \alpha_2,\ldots,)$ и $\m{b} = (\beta_0 \,|\, \beta_1, \beta_2,\ldots,)$ лексикографически, \ie если $\alpha_0<\beta_0$, то $\m{a}<\m{b}$, если $\alpha_0=\beta_0$, то сравниваем $\alpha_1$ и $\beta_1$. Тогда $\alpha_1<\beta_1$ влечёт $a<b$ и т.д. Мы тут предполагаем, что $\alpha_0,\beta_0 \ge 0$. В противном случае ясно, что нужно делать, если мы уже умеем сравнивать дроби при положительных $\alpha_0,\beta_0.$

\begin{theorem}
    На множестве бесконечных десятичных дробей выполняется аксиома полноты, и более того, на этом множестве можно ввести операцию суммы и умножения так, чтобы выполнялись все аксиомы упорядоченного поля. 
\end{theorem}
\begin{proof}
    Мы не будем доказывать часть про аксиомы упорядоченного поля. Докажем, что выполнена аксиома полноты.

    Пусть $A = \{\m{a}_i\}$, где $i$ пробегает какое-то множество индексов (возможно несчётное), здесь каждое $\m{a}_i$ -- это какая-то бесконечная десятичная дробь. Аналогично, пусть $B = \{\m{b}_j\}$, где каждое $\m{b}_j$ -- это какая-то бесконечная десятичная дробь. Пусть $A\le B$, найдём теперь такую бесконечную десятичную дробь $\m{c}$, которая разделяет эти два множества. 

    Пусть $\m{c}:= c_0,\mathsf{c}_1\mathsf{c}_2\ldots,$ где $c_0$ -- это наименьшая целая часть среди всех дробей множества $B$. Пусть $ B_0 $ -- это подмножество множества $B$, которое состоит из всех таких дробей, целая часть которых начинается только с $c_0$. Тогда $c_1$ будет наименьшая первая цифра после запятой в этих дробях из $B_0$. Далее, мы рассматриваем множество $B_1$, которое состоит уже из дробей вида $c_0,\mathsf{c}_1\ldots$ и определяем $\mathsf{c}_2$ как наименьшее среди всех цифр во второй позиции после запятой. Тем самым, как нетрудно видеть, мы и построили разделитель.
    
    \end{proof}


\section{Лекция \# 2. Предел последовательности}

\subsection{Понятие последовательности}

Понятие последовательности возникает во многих случаях. 

\begin{definition}\label{def_of_seqeunce}
    Если каждому $n \in \mathbb{N}$ поставлено в соответствие некоторое число $x_n \in \mathbb{R}$, то говорят, что \textit{задана последовательность} $\m{x}:=(x_n)$, при этом $x_n$ называется \textit{элементом последовательности, а $n$ -- её номером.}
\end{definition}


\begin{example}~
    \begin{enumerate}
        \item $\m{x} = \{1,1,1,\ldots,\}$. \ie $\m{x} = (x_n)$, где $x_n = 1$ при каждом $n \in \mathbb{N}.$
        \item $\m{x} = \{1,-1,1,-1,1,-1,\ldots, \}$ \ie $\m{x} = (x_n)$, где $x_n = (-1)^n$ при каждом $n \in \mathbb{N}.$
        \item $\m{x} = \{1,2,3,4, \ldots,\}$ \ie $\m{x} = (x_n)$, где $x_n = n$ при каждом $n \in \mathbb{N}.$
    \end{enumerate}
\end{example}

Приведённые выше примеры хороши тем, что понятно, как строится эта последовательность. В том случае, когда у нас есть какое-то правило, которое позволяет сказать, как будет выглядеть её любой элемент, мы говорим, что последовательность задана общим элементом. 

Разумеется, не все последовательности можно задать явным описанием, например, последовательность $(\mathscr{M}_n)$ хоть и является очень важной\footnote{это последовательность меандров с заданным порядком, \url{https://oeis.org/A005315/list}}, но на данный момент известен лишь её $22$-ой элемент 
\begin{center}
 \begin{tabular}{llll}
    $\mathscr{M}_1=  1,$ & $\mathscr{M}_2 = 2,$ &  $\mathscr{M}_3 = 8,$ & $\mathscr{M}_4 = 42,$  \\
     $\mathscr{M}_5=  262,$ & $\mathscr{M}_6 = 1828,$ & $\mathscr{M}_7 = 13820,$ & $\mathscr{M}_8 = 110954,$ \\
     $\mathscr{M}_9=  933458,$ & $\mathscr{M}_{10} = 8152860,$ & $\mathscr{M}_{11} = 73424650,$ & $\mathscr{M}_{12} = 678390116,$ \\
    $\mathscr{M}_{13}=  6405031050,$ & $\mathscr{M}_{14} = 61606881612,$ & $\mathscr{M}_{15} = 602188541928,$ & $\mathscr{M}_{16} = 5969806669034,$ 
\end{tabular}    
\end{center}

\subsection{Определение предела последовательности}

В предыдущем примере мы видели, что элементы последовательности растут, \textit{т.е.} с увеличением номера увеличивается и само значение элемента.



\begin{example}
Рассмотрим следующие примеры.
    \begin{enumerate}
        \item Если вписать в окружность единичного радиуса правильные многоугольники, то их периметры образуют последовательность $(p_n)$, где $p_n$ -- периметр правильного $(n+2)$-угольника
        \begin{center}
            \begin{tabular}{c|c|c|c|c|c|c|c}
                $n$ & $1$ &  $2$ & $3$     & $4$  & $5$ & $6$ & $7$  \\ \hline
                $p_n$ & $5.1961$ & $5.6568$ & $5.8778$ & $5.9999$ & $6.0743$ & $6.1229$ & $6.1563$
            \end{tabular}
        \end{center}
        Из курса элементарной геометрии хорошо известно, что общий элемент этой последовательности задаётся формулой
        \[
         p_n = 2(n+2)\sin\left(\frac{\pi}{n+2}\right),
        \]
        и что эта последовательность -- это десятичное приближение числа $2\pi.$
     \item Десятичное приближение числа $\sqrt{2}$ можно представить в виде последовательности
    \[
    \begin{tabular}{l|c|c|c|c|c|c|c|c|c|c}
      $n$ & 1&2&3&4&5&6&7&8&9&10\\
       \hline
     $x_n$&  1 & 1.4 & 1.41 & 1.414 & 1.4142 & 1.41421 & 1.142& 1.414213 & 1.4142135 & 1.41421356  
         \end{tabular}
    \]
    Предъявить формулу для общего элемента уже требует некоторых дальнейший знаний нашего предмета, и мы позже научимся это делать.
    
    \item Рассмотрим последовательность обратную к последовательности натуральных чисел, \textit{т.е.}
    \[
      \frac{1}{1}, \,\frac{1}{2}, \,\frac{1}{3}, \, \frac{1}{4}, \, \frac{1}{5}, \, \frac{1}{6},\, \frac{1}{7},\, \frac{1}{8},\, \frac{1}{9},\, \frac{1}{10},  \ldots,
    \]
    общий элемент этой последовательности -- это просто $x_n = \frac{1}{n}$, $n \in \mathbb{N}.$ Первые 12 элементов этой последовательности можно выписать как десятичные дроби в следующей таблице:

        \[
         \begin{tabular}{l|c|c|c|c|c|c|c|c|c|c|c|c}
             $n$ & 1&2&3&4&5&6&7&8&9&10&11&12  \\
             \hline
             $x_n$&  1 & 0.5 & 0.333 & 0.25 & 0.2 & 0.166 & 0.142& 0.125 & 0.111 & 0.1 & 0.09 & 0.0833 
         \end{tabular}
        \]
        
    \item Наконец, мы можем придумать любое выражение в котором присутствует натуральное число, например, рассмотрим последовательность $(a_n)$, у которой общий элемент задаётся формулой
    \[
     a_n = \frac{2n}{n+3},
    \]
    её первые 14 элементов выглядят следующим образом
    \[
       \begin{tabular}{l|c|c|c|c|c|c|c|c|c|c|c|c|c|c}
             $n$ & 1&2&3&4&5&6&7&8&9&10&11&12&13&14  \\
             \hline
             $a_n$&  0.5 & 0.8 & 1 & 1.14 & 1.25 & 1.33& 1.4 & 1.45 & 1.5 & 1.53 & 1.57 & 1.6 & 1.62 & 1.647 
         \end{tabular}
        \]
    \end{enumerate}
\end{example}

\begin{mydanger}{\bf {\color{red}!}}
 У всех этих примеров есть одна общая особенность: {\color{red}элементы каждой из этих последовательностей \textit{приближаются} к некоторому значению.}     
\end{mydanger}

\begin{remark}
 Такие последовательности играют очень большую роль в анализе. Мы позже увидим, что множество действительных чисел можно определить с помощью последовательностей из рациональных чисел.    
\end{remark}

Итак, мы подошли к очень важному понятию.

\begin{definition}\label{limit_of_seqeunce}
    Числовая последовательность $(x_n)$ \textit{сходится} к числу $x$, если для любого (=каждого) числа $\varepsilon>0$ можно найти такое число (=номер последовательности) $N$, что $|x_n -x|<\varepsilon$ {\color{red} \bf для всех} $n >N$. При этом пишут $\lim\limits_{n \to \infty} x_n = x$. Число $x$ в таком случае называется \textbf{пределом последовательности.}
\end{definition}

Это же определение в кванторах записывается следующим образом: 
\[
 \boxed{
\lim_{n \to \infty } x_n = x \Longleftrightarrow  \forall \varepsilon >0\, \exists N \in \mathbb{N}\, :\, \forall n >N, \, |x_n - x| <\varepsilon.
}
\]

\begin{mydanger}{\bf{!}}
Неформально это определение как раз и понимается следующим образом: \textit{последовательность $x_n$ сходится к числу $x$, если при возрастании номера $n$ число $x_n$ неограниченно приближается к числу $x.$}
\end{mydanger}

\begin{comments}
  Неравенство $|x_n-x|<\varepsilon$ эквивалентно системе $x-\varepsilon <x_n < x+ \varepsilon$. Тогда определение предела может пониматься так, что начиная с какого-то номера $N$, \textbf{ВСЕ остальные элементы последовательности $x_{N+1}, x_{N+2}, \ldots,$ попадут в интервал $(x-\varepsilon, x+\varepsilon)$.}
  
  Далее число $\varepsilon>0$ здесь нужно, чтобы сказать, что этот интервал может быть сколь угодно малым.

  Напротив, если хотя бы один элемент $x_m$ при $m>N$ не попадает в интервал $(x-\varepsilon, x+\varepsilon),$ то $x$ \textbf{не будет пределом этой последовательности!}
\end{comments}

\begin{example}
    Рассмотрим последовательность $(x_n)$
    \[
      \frac{1}{1}, \,\frac{1}{2}, \,\frac{1}{3}, \, \frac{1}{4}, \, \frac{1}{5}, \, \frac{1}{6},\, \frac{1}{7},\, \frac{1}{8},\, \frac{1}{9},\, \frac{1}{10},  \ldots,
    \]
  и покажем, что $\lim_{n \to \infty} x_n = 0$.

 Используя определение, нам нужно найти для любого $\varepsilon >0$ такой номер $N$, что $|x_n - 0| < \varepsilon$ для всех $n>N$.

Итак, рассмотрим неравенство $|x_n - 0| < \varepsilon$, так как $x_n = \frac{1}{n}$ и $n\ge 1$, то это неравенство, очевидно, эквивалентно неравенству $n > \frac{1}{\varepsilon}$.

Пусть
\[
 N := \left\lfloor \frac{1}{\varepsilon} \right\rfloor \mbox{ = целая часть числа.}
\]

Тогда если $n>N$, то все $x_{N+1}, x_{N+2}, \ldots $ будут меньше чем $\varepsilon.$ Например, пусть $\varepsilon = 0.1$, тогда $N = 10$, и, значит, ВСЕ элементы этой последовательности, начиная с $x_{11}$, будут меньше, чем $\varepsilon = 0.1$, \textit{т.е.} $x_{11}, x_{12}, x_{13}, \ldots < 0.1$.

Действительно, имеем

 \begin{table}[h!]
     \centering
      \begin{tabular}{llll}
    {\color{blue}$x_1= \frac{1}{1}=  1,$} & {\color{blue}$x_2= \frac{1}{2}=  0.5,$} &  {\color{blue}$x_3= \frac{1}{3}=  0.333...,$} & {\color{blue}$x_4= \frac{1}{4}=  0.25,$}  \\
    &&&\\
    $x_5= \frac{1}{5}=  0.2,$ & $x_6= \frac{1}{6}=  0.166...,$ & $x_7= \frac{1}{7}=  0.142,$ & $x_8= \frac{1}{8}=  0.125,$ \\
    &&&\\
    $x_9= \frac{1}{9} = 0.111..., $ & $x_{10} = \frac{1}{10} = 0.1,$ & {\color{red}$x_{11} = \frac{1}{11} = 0.09...,$} & {\color{red}$x_{12} = \frac{1}{12} = 0.0833..,$} \\
    &&&\\
    {\color{red}$x_{13}=  \frac{1}{13} = 0.076...,$} & {\color{red}$x_{14} = \frac{1}{14} =0.0714,$} & {\color{red}$x_{15} = \frac{1}{15} = 0.0666...,$} & {\color{red}$x_{16} = \frac{1}{16} = 0.0625,$} 
\end{tabular}    
     \caption{Красным цветом показаны те элементы последовательности, которые меньше заданного $\varepsilon =0.1$. Синим показаны те элементы, для которых верно неравенство $|x_n - 1| < 0.8.$}
     \label{tab_for_1/n}
 \end{table}

Покажем, например, что $1$ не является пределом этой последовательности. Если бы $1$ был бы пределом этой последовательности, то неравенство $|x_n-1|<\varepsilon$ выполнялось бы при всех $n>N$, начиная с какого-то $N$. Но так как $x_n <1$, то это неравенство эквивалентно неравенству $1-\frac{1}{n}<\varepsilon$ и влечёт $n < \frac{1}{1-\varepsilon}$.

Таким образом, неравенство $|x_n -1| < \varepsilon$ будет выполняться лишь для {\bf КОНЕЧНОГО} числа номеров $n$!

Например, если $\varepsilon =0.8$, то $\frac{1}{1-0.8} = \frac{1}{0.2} = 5$, \ie неравенство
\[
\left |\frac{1}{n}-1 \right|<0.8
\]
выполнено только для $x_1,x_2,x_3,x_4$, что видно из таблицы \ref{tab_for_1/n}. Это означает, что $1$ не является пределом этой последовательности.
\end{example}

\begin{definition}
    Последовательность $(x_n)$ называется \textit{бесконечно малой} если $\lim_{n\to \infty} x_n = 0.$
\end{definition}

Как мы уже показали, последовательность 
 \[
      \frac{1}{1}, \,\frac{1}{2}, \,\frac{1}{3}, \, \frac{1}{4}, \, \frac{1}{5}, \, \frac{1}{6},\, \frac{1}{7},\, \frac{1}{8},\, \frac{1}{9},\, \frac{1}{10},  \ldots,
    \]
является бесконечно малой.

\begin{proposition}\label{lim(a_n-a)=0}
    Пусть дана последовательность $(x_n)$, и пусть $\lim_{n\to \infty}x_n = x$, тогда это равносильно тому, что $\lim_{n\to \infty}x_n' = 0$, где $x_n' := x_n - x$, $n\in \mathbb{N}.$
\end{proposition}
\begin{proof}
    Действительно, имеем следующую цепочку эквивалентностей
    \begin{eqnarray*}
      \lim_{n \to \infty} x_n = x & \Longleftrightarrow &  \forall \varepsilon >0\, \exists N \in \mathbb{N}\, :\, \forall n >N, \, |x_n - x| <\varepsilon \\
      &\Longleftrightarrow& \forall \varepsilon >0\, \exists N \in \mathbb{N}\, :\, \forall n >N, \, |x_n' - 0| <\varepsilon \\
      &\Longleftrightarrow &  \lim_{n \to \infty} x'_n = 0.
    \end{eqnarray*}
\end{proof}

\begin{mydanger}{\color{blue}\bf !}
    Таким образом, изучение сходящихся последовательностей равносильно изучению бесконечно малых последовательностей. 
\end{mydanger}


\begin{example}
    Рассмотрим последовательность $(a_n)$, где $a_n = \frac{2n}{n+3}$ и выпишем её первые 15 элементов:
        \[
         \begin{tabular}{l|c|c|c|c|c|c|c|c|c|c|c|c|c|c|c|}
             $n$ & 1&2&3&4&5&6&7&8&9&10&11&12&13&14&15  \\
             \hline
             $a_n$&  0.5 & 0.8 & 1 & 1.14 & 1.25 & 1.33& 1.4 & 1.45 & 1.5 & 1.53 & 1.57 & 1.6 & 1.62 & 1.647 & 1.666 
         \end{tabular}
        \]
Мы утверждаем, что $\lim_{n\to \infty}a_n =2$, чтобы это доказать, мы рассмотрим последовательность $(a_n')$, где $a_n' := a_n  -2$ и покажем, что эта последовательность бесконечно малая.

Имеем
\[
 a_n'=a_n -2= \frac{ 2n}{n+3} -2 = - \frac{6}{n+3}.
\]

Тогда, чтобы доказать, что $\lim_{n\to \infty } a_n' = 0$, нужно для любого $\varepsilon >0$ найти такой номер $N$, что все остальные $x_{N+1}, x_{N+2}, \ldots$ попадут в промежуток $(-\varepsilon, \varepsilon)$. Другими словами, для всех $n>N$ должно выполняться неравенство
\[
  \left|-\frac{6}{n+3}\right|<\varepsilon
\]

Но, в таком случае, мы легко находим $n> \frac{6}{\varepsilon} - 3$, то есть, начиная с номера $N= \left\lfloor \frac{6}{\varepsilon} - 3 \right\rfloor$, все $x_{N+1}, x_{N+2}, \ldots,$ будут находиться в интервале $(-\varepsilon, \varepsilon).$ 

Пусть, например, $\varepsilon = 0.1$, тогда
\[
 N =\left\lfloor \frac{6}{0.1} - 3 \right\rfloor = \left\lfloor 60 - 3 \right\rfloor = 57,
\]
\ie все $a'_{58},a'_{59}, \ldots$ будут находиться в промежутке $(-0.1, 0.1)$, в чём можно убедится вычислив эти значения последовательности $(a_n')$, $a_n' = a_n - 2 = -\frac{6}{n+3}$

  \[
         \begin{tabular}{l|c|c|c|c|c|c|c|c|c|c|c}
             $n$ & 55& 56 & 57 &58 &59 & 60 &61 &62 &63 &64 &65 \\
             \hline
             $a'_n$&  -0.103 & -0.101 & -0.1 & -0.09 & -0.096 & -0.095& -0.093 & -0.0923 & -0.0909& -0.089 & -0.088 
         \end{tabular}
        \]
\end{example}

\begin{example}
    С другой стороны, вернёмся к последовательности $(x_n)$, $x_n = \frac{1}{n}$
     \[
      \frac{1}{1}, \,\frac{1}{2}, \,\frac{1}{3}, \, \frac{1}{4}, \, \frac{1}{5}, \, \frac{1}{6},\, \frac{1}{7},\, \frac{1}{8},\, \frac{1}{9},\, \frac{1}{10},  \ldots,
    \]
и рассмотрим последовательность $(x_n')$, где $x_n':= x_n - 1$. 

Первые её 12 элементов имеют вид
        \[
         \begin{tabular}{l|c|c|c|c|c|c|c|c|c|c|c|c}
             $n$ & 1&2&3&4&5&6&7&8&9&10&11&12  \\
             \hline
             $x_n'$&  0 & -0.5 & -0.666 & -0.75 & -0.8 & -0.833 & -0.857& -0.875 & -0.888 & -0.9 & -0.909 & -0.916   
         \end{tabular}
        \]
Мы уже показали, что $1$ не является пределом последовательности $x_n$, а тогда $x_n'$ не является бесконечно малой, что и подтверждает эта таблица.
\end{example}




\subsection{Отделимость, существование и единственность предела}



\begin{theorem}[Отделимость]\label{separate}
    Если $\lim_{n\to \infty } x_n = x$, $x>0$, то найдётся такой номер $N_0$, что при всех $n>N_0$, будет иметь место неравенство $x_n > \frac{x}{2}>0.$
\end{theorem}

\begin{proof}
Так как $\lim_{n\to \infty } x_n = x$, то для любого числа $\varepsilon>0$ можно найти такой номер $N$, что при всех $n>N$, $|x_n-x|<\varepsilon$. Так как по условию $x>0$, то мы можем положить $\varepsilon := \frac{x}{2}$, тогда можно найти такой номер $N'$, что
\[
 |x_n - x|<\frac{x}{2}
\]
для всех $n>N'$.

Последнее неравенство равносильно системе неравенств
\[
 \frac{x}{2}< x_n < \frac{3x}{2}
\]
но так как $x>0$, то мы получаем $0<\frac{x}{2}< x_n < \frac{3x}{2}$. Поэтому если положить, что $N_0 :=N'$,то при всех $n>N_0$, имеем $x_n > \frac{x}{2}$, \textit{т.е.,} что и завершает доказательство.
\end{proof}

Обратимся теперь к вопросу о существовании предела у последовательности.

\begin{example}
Рассмотрим последовательность $(x_n):$
\[
 1,\,2,\,3,\, 4,\, 5,\, 6,\, 7,\, \ldots,
\]
и пусть $\lim_{n \to \infty}x_n = a$, тогда по определению для любого $\varepsilon>0$ можно найти такой номер $N$, что при всех $n>N$ будем выполняться неравенство $|x_n - a|<\varepsilon$, а так как $x_n = n$, то $|n-a|<\varepsilon$. Но если $a$ -- фиксированное число, то выражение $|n-a|$ можно сделать сколь угодно большим при $n>a$. Таким образом, эта последовательность не может иметь предела.    
\end{example}

\begin{mydanger}{\bf !}
 Приведённый пример последовательности это так называемые \textit{бесконечно большие} последовательности. В таких случаях принято писать $\lim_{n\to \infty}x_n = \infty$, но мы пока не будем рассматривать такие случаи.
\end{mydanger}

\begin{example}\label{(-1)^n}
Рассмотрим теперь последовательность $(x_n)$, где $x_n = (-1)^n:$
\[
 -1,\, 1,\,-1,\, 1,\,-1,\, 1,\,-1,\, 1,\,-1,\, 1,\,-1,\, 1,\,-1,\, 1,\, \ldots
\]
и покажем, что эта последовательность не имеет предела.

Пусть она имеет предел, скажем, $x$. Рассмотрим варианты. Пусть $0 \le x <1$, тогда, выбрав $\varepsilon$ таким, что $\varepsilon < 1-x$, мы видим, что ни один элемент этой последовательности не лежит в интервале $(x- \varepsilon, x+\varepsilon)$. Аналогично рассматривается случай $-1<x\le 0$.

Пусть $x\ge 1$, тогда, выбрав $\varepsilon < 1$, мы видим, что не все элементы последовательности попадают в интервал $(x-\varepsilon, x+ \varepsilon)$, а именно туда не попадут элементы $-1$. Аналогично рассматривается случай, когда $x\le -1$. Это и показывает, что такая последовательность не имеет предела.
\end{example}

\begin{example}
    Пусть $x_n =\alpha$ последовательность вида $\{\alpha, \alpha, \alpha, \ldots,\}.$ Такая последовательность называется \textit{постоянной.} Тогда $\lim_{n\to \infty }x_n = \alpha$. Действительно, мы имеем $|x_n - \alpha| = |\alpha - \alpha| = 0$ для любого $n \in \mathbb{N}$, тогда и $|x_n - \alpha| < \varepsilon$ для любого $\varepsilon>0.$
\end{example}


\begin{theorem}
    Если предел последовательности существует, то он единственен. 
\end{theorem}
\begin{proof}
    Пусть последовательность $\m{x} = (x_n)$ имеет два предела, скажем $a,b$, при этом $a\ne b$. Тогда по определению предела для любого $\varepsilon>0$ мы можем найти такие $N, M$, что $|x_n - a|<\varepsilon$, когда $n>N$, и $|x_m-b|<\varepsilon$, при $m>M$. Пусть $K:=\max\{N,M\}$, тогда при $k>K$ мы будем иметь:
   \begin{eqnarray*}
      |a-b| &=& |a-x_K + x_K -b| \\
      &\le & |a-x_K| + |x_K-b| \\
      &\le& \varepsilon + \varepsilon \\
      &=& 2 \varepsilon.
   \end{eqnarray*}

    А теперь воспользуемся тем фактом, что это неравенство должно выполняться \textbf{для любого числа $\varepsilon>0$}. Тогда, выбрав, например, $\varepsilon = \frac{1}{3}|a-b|$, мы приходим к противоречию.
\end{proof}

\begin{remark}
Вообще-то, можно придумать пример пространства, на котором одна и та же последовательность будет иметь более, чем один предел. Например, возьмём два экземпляра прямой $\mathbb{R}$ и склеим их по всем точкам кроме нуля.
    
Более строго: рассмотрим фактормножество $\mathbb{R}\times \mathbb{R}/\sim$, где $x \sim y$ всегда, когда $x,y \in \mathbb{R}\setminus \{0\}$. То, что получилось, называется прямой с удвоенной точкой (affine line with double origin). Тогда последовательность будет как раз и иметь два предела, соответствующие этой двойной точке.

Но! Это уже не математический анализ, и такие структуры мы рассматривать не будем.
\end{remark}


\section{Лекция \# 3. Некоторые свойства предела}

\subsection{Арифметика предела}

\begin{theorem}[{Арифметика пределов}]\label{a+b,ca,ab}
    Пусть $(a_n), (b_n )$ -- две последовательности, причём $\lim_{n\to \infty} a_n  =a$ и $\lim_{n\to \infty}b_n =b$. Тогда:
    \begin{enumerate}
        \item $\lim_{n\to \infty}(a_n + c) = a+c$ и $\lim_{n\to \infty}(ca_n) = ca$ для любого числа $c\in \mathbb{R};$
        \item $\lim_{n\to \infty}(a_n +b_n) = a+b;$
        \item $\lim_{n\to \infty}(a_nb_n) = ab;$
        \item если $a\ne 0$ и $a_n \ne 0$ для всех $n\in \mathbb{N}$, то $\lim_{n\to \infty}\frac{a_n}{b_n} = \frac{a_n}{b_n}.$
    \end{enumerate}
\end{theorem}
\begin{proof}~\\
    (1) Пусть $a_n':=a_n+c$, $n\in \mathbb{N}$. По определению получаем, что для любого $\varepsilon > 0$ найдётся такой $N$, что $|a_n - a| < \varepsilon$. Имеем $|a_n -a| = |a_n+c - a - c| = |a_n' - (a+c)|$, то есть для того же $N$ мы получаем, что при $n>N$, $|a_n' - (a+c)| < \varepsilon$, что и доказывает, что $\lim_{n\to \infty}(a_n + c) = a+c$.

    Пусть теперь $a'_n: = ca_n$, $n\in \mathbb{N}$. Ясно, что если $c =0$, то мы получаем постоянную последовательность $\{0,0,...\ldots,\}$ и мы уже знаем, что она сходится к $0.$ Пусть $c \ne 0.$ Так как $\lim_{n\to \infty}a_n = a$, то для любого $\varepsilon>0$ есть такой номер $N$, что $|a_n - a|<\varepsilon$ для всех $n>N$. Значит, если мы рассмотрим $\varepsilon':=\frac{\varepsilon}{|c|}$, то и для такого $\varepsilon'$ мы тоже знаем такой номер $N'$, что $|a_n - a|< \varepsilon'$. Умножив обе части этого неравенства на $|c|$, мы получаем $|c||a_n - a| <|c|\varepsilon'$, что равносильно неравенству $|a_n' - ca| < \varepsilon$, что и доказывает требуемое.

    (2) Возьмём какое-нибудь число $\varepsilon>0$ и положим $\varepsilon':= \frac{\varepsilon}{2}$, тогда, согласно определению предела, у нас есть номера $N,M$ такие, что неравенства $|a_n - a|< \varepsilon'$ и $|b_m - b|<\varepsilon'$ выполнены для всех $n>N$, $m>M$. Пусть $K:=\max\{N,M\}$, $k>K$.
    
    Имеем
    \[
     |(a_k + b_k) - (a+b)| = |(a_k-a) + (b_k-b)|\le |a_k-a| + |b_k-b| < \varepsilon' + \varepsilon' = \varepsilon,
    \]
    что и означает, что для любого $\varepsilon>0$ мы нашли номер $K$ такой, что для всех $k>K$ имеет место неравенство $|(a_k + b_k) - (a+b)|< \varepsilon$, что и доказывает требуемое.

    (3) Заметим, что 
    \[
    a_nb_n - ab = (a_n-a)(b_n-b) + a(b_n -b) + b(a_n -a).
    \]

    Докажем, что $\lim_{n\to \infty}(a_nb_n - ab) = 0$, тогда по Предложению \ref{lim(a_n-a)=0} будет следовать требуемое.

    Согласно только что доказанным пунктам, имеем
    \begin{eqnarray*}
     \lim_{n\to \infty}(a_nb_n - ab) &=&  \lim_{n\to \infty}\Bigl( (a_n-a)(b_n-b) + a(b_n -b) + b(a_n -a) \Bigr)  \\
    &=&  \lim_{n\to \infty} (a_n-a)(b_n-b) + a \lim_{n\to \infty}(b_n -b) + b \lim_{n\to \infty}(a_n -a),
    \end{eqnarray*}
по Предложению \ref{lim(a_n-a)=0}, $\lim_{n\to \infty}(b_n -b) =0$, $\lim_{n\to \infty}(a_n -a)=0$, значит, $\lim_{n\to \infty}(a_nb_n - ab) = \lim_{n \to \infty}(a_n-a)(b_n-b).$ Покажем, что этот предел равен нулю. Так как $\lim_{n\to \infty} a_n  =a$ и $\lim_{n\to \infty}b_n =b$, то для любого $\varepsilon>0$, есть номера $N,M$ такие, что $|a_n - a| <\varepsilon$ и $|b_m -b| < \varepsilon$ для всех $n >N$, $m>M$. Пусть теперь $\varepsilon':=\sqrt{\varepsilon},$ тогда и для такого числа мы тоже знаем номера $N',M'$ такие, что  $|a_n - a| <\varepsilon'$ и $|b_m -b| < \varepsilon'$ для всех $n >N'$, $m>M'$. Пусть $K:= \max\{N',M'\}$, тогда для $k>K$ мы получаем
\[
 |(a_k-a)(b_k-b)| = |a_k -a||b_k - b| < \varepsilon' \cdot \varepsilon' = \varepsilon, 
\]
что и доказывает бесконечную малость последовательности $(a_n-a)(b_n-b)$, \ie $\lim_{n \to \infty}(a_n-a)(b_n-b) = 0$ и значит, $\lim_{n\to \infty}(a_nb_n - ab) = 0 \Longleftrightarrow \lim_{n\to \infty}a_nb_n = ab.$

(4) Достаточно доказать, что $\lim_{n \to \infty} \frac{1}{b_n} = \frac{1}{b}$. Тогда из предыдущего пункта будет следовать требуемое. Далее без ограничения общности будем считать, что $b>0$, так как в случае $b<0$ мы умножим последовательность $\{b_n\}$ на $-1$ и по пункту (1) сведём задачу к той, когда $b>0.$

Так как $\lim_{n\to \infty} b_n =b>0$, то по Теореме \ref{separate} найдётся такой $N_0$, что для всех $n>N_0$ будет иметь место неравенство $b_n > \frac{b}{2}>0$. 

Далее, так как $\lim_{n\to \infty} b_n =b$, то по определению предела для любого $\varepsilon>0$ найдётся такой номер $N$, что для всех $n>N$ выполнено $|b_n - b|<\varepsilon.$

Пусть $M: = \max\{N_0, N\}$, тогда для любого $m>M$, получаем
\[
\left| \frac{1}{b_m} - \frac{1}{b} \right| = \frac{|b_m-b|}{b_mb},
\]
так как $b_n >\frac{b}{2}$ для всех $n>N_0$, то $\frac{1}{b_m} < \frac{2}{b}$ для всех $m >M$, и тогда мы получаем 
\[
\left| \frac{1}{b_m} - \frac{1}{b} \right| = \frac{|b_m-b|}{b_mb} \le \frac{|b_m-b|}{\frac{b^2}{2}} < \frac{2\varepsilon}{b^2},
\]
что и доказывает требуемое.
\end{proof}

\begin{proposition}
 Пусть дана последовательность $(a_n )$ такая что $a_n \ge 0$, $n\ge 1$ и $\lim_{n \to \infty} a_n = a$, $a \ge 0$. Тогда $\lim_{n \to \infty}\sqrt{a_n} = \sqrt{a}.$    
\end{proposition}
\begin{proof}
 (1) Пусть $\lim_{n \to \infty}a_n = 1$, тогда для любого $\varepsilon>0$ можно найти такой $N$, что $1 - \varepsilon <a_n<1 + \varepsilon$ для всех $n >N$. Тогда, если $\varepsilon<1$, то получаем $\sqrt{1-\varepsilon} < \sqrt{a_n} <\sqrt{1+\varepsilon}$. Далее, так как $\sqrt{1+\varepsilon} < 1 + \varepsilon$, и $1-\varepsilon < \sqrt{1-\varepsilon}$ при любом $0 <\varepsilon<1$ то мы получаем
 \[
  1 -\varepsilon < \sqrt{1-\varepsilon} < \sqrt{a_n} < \sqrt{1+\varepsilon} < 1 + \varepsilon
 \]
для всех $n > N$ что влечёт $\lim_{n \to \infty}\sqrt{a_n} =1$. 

 Пусть $\varepsilon >1$, тогда мы получаем что $1 - \varepsilon <0$ и $0 < a_n  < 1 +\varepsilon$ для всех $n >N$, но также мы получаем что и $0 < \sqrt{a_n} < 1 + \varepsilon$, что также можно записать так $1- \varepsilon < \sqrt{a_n} < 1 + \varepsilon$.

 Итак, мы показали, что если $\lim_{n \to \infty}a_n = 1$ то и $\lim_{n \to \infty}\sqrt{a_n} = 1.$

 (2) Пусть $\lim_{n \to \infty}a_n = a \ne 1,0.$ Тогда $\lim_{n \to \infty} \frac{a_n}{a} = 1$, и тогда $\lim_{n \to \infty}\sqrt{\frac{a_n}{a}} = 1$. Умножая на $\sqrt{a}$ обе части и воспользовавшись арифметикой предела, получаем $\sqrt{a} \lim_{n \to \infty}\sqrt{\frac{a_n}{a}} = \lim_{n \to \infty}\sqrt{a_n} = \sqrt{a}$.

 (3) пусть $a = 0$, тогда для любого $\varepsilon>0$ можно найти такой $N$, что $-\varepsilon < a_n <\varepsilon$, для всех $n>N$. Тогда $-\sqrt{\varepsilon} < 0 < \sqrt{a_n} < \sqrt{\varepsilon}$, что и показывает $\lim_{n\to \infty} \sqrt{a_n} = 0.$
\end{proof}


\subsection{Лемма о зажатой последовательности}

\begin{lemma}[{\textbf{Лемма о зажатой последовательности}}]\label{sqeezy}
    Пусть даны такие последовательности $(a_n), (b_n), (c_n)$, что $a_n<b_n<c_n$ для всех $n$, $\lim_{n \to \infty} a_n = \lim_{n \to \infty} c_n = a$, тогда $\lim_{n \to \infty} b_n = a.$
\end{lemma}
\begin{proof}
   По определению предела, для любого $\varepsilon >0$, существуют такие номера $N,M$, что $|a_n - a| < \varepsilon$ и $|c_m - a|< \varepsilon$ для всех $n>N$, $m>M$. Пусть $K:=\max\{N,M\}$, тогда для любого $k>K$, $|a_k - a| < \varepsilon$ и $|c_k - a|< \varepsilon$. Мы получили совокупность неравенств
   \[
   a - \varepsilon < a_k < a+ \varepsilon, \qquad a -\varepsilon < c_k < a+ \varepsilon, \qquad \forall k >K,
   \]
   но тогда мы получаем $a- \varepsilon < a_k < b_k < c_k < a + \varepsilon$, \ie для любого $\varepsilon>0$ мы нашли такое $K$, что для всех $k>K$, $a- \varepsilon < b_k < a+ \varepsilon$ или то же самое, что и $|b_k -a|< \varepsilon$, но это и означает, что $\lim_{n \to \infty}b_n = a$, что и требовалось доказать.
\end{proof}

\begin{example}\label{sqrt[n]{n}->1}
    Покажем, что $\lim_{n\to \infty}\sqrt[n]{n} = 1$. Имеем $n = \left(1+ (\sqrt[n]{n}-1) \right)^n$, тогда по биному Ньютона получаем
    \[
     n = 1 + \binom{n}{2} (\sqrt[n]{n} - 1)^2 + \cdots + (\sqrt[n]{n} - 1)^n,
    \]
    тогда
    \[
     n > \binom{n}{2} (\sqrt[n]{n} - 1)^2 = \frac{n(n-1)}{2} (\sqrt[n]{n} - 1)^2, 
    \]
    откуда получаем 
    \[
     0 < (\sqrt[n]{n} - 1) < \sqrt{\frac{2}{n-1}} ,
    \]
    и так как $\lim_{n\to \infty} 0 = 0$ и $\lim_{n\to \infty} \frac{2}{n-1} = 0$, то по лемме о зажатой последовательности \ref{sqeezy}, получаем $\lim_{n\to \infty} (\sqrt[n]{n} - 1) = 0$, а тогда используя предложение \ref{lim(a_n-a)=0} мы получаем требуемое.   
    
\end{example}


\begin{lemma}[{\textbf{Переход к пределу в неравенствах}}]\label{a<b}
 Пусть даны такие последовательности $ (a_n ),  (b_n )$, что $a_n<b_n$ для всех $n$, $\lim_{n \to \infty} a_n =a$ и $\lim_{n \to \infty} b_n = b$. Тогда $a\le b$. 
\end{lemma}
\begin{proof}
Пусть $\varepsilon_0:=a-b >0$. Согласно определению предела, мы можем для $\frac{\varepsilon_0}{2}$ найти такие $N,M$, что $|a_n - a|<\frac{\varepsilon_0}{2}$, $|b_m-b|<\frac{\varepsilon_0}{2}$ для всех $n>N$, $m>M$.  Пусть $K:=\max\{N,M\}$, тогда для любого $k>K$
\begin{eqnarray*}
    \varepsilon_0 &=& a-b\\
    &=& a-a_k + a_k - b_k+b_k - b \\
    &\le &a-a_k + b_k-b \\
    &<&\varepsilon_0
\end{eqnarray*}
что даёт противоречие.

\end{proof}

\section{Лекция \# 4. Ограниченные последовательности}

\subsection{Ограниченные множества и принцип полноты Вейерштрасса}

Прежде всего нам понадобятся следующие определения:

\begin{definition}\label{sup,inf}
    Пусть $A \subseteq \mathbb{R}$ -- непустое подмножество. Число $a$ называется \textit{верхней гранью} множества $A$, если $x\le a$ для любого $x \in A$. Если есть хотя бы одна верхняя грань, то говорят, что множество $A$ \textit{ограниченно сверху.} Наименьшая из всех верхних граней множества $A$ называется \textit{точной верхней гранью} множества $A$ и обозначается $\mathrm{sup}(A)$ (=супремум).
\end{definition}

\begin{lemma}
 Если $\sup (A) = \alpha$, то для любого $\varepsilon>0$, найдётся хотя бы одно $a \in A$, такое, что $\alpha - \varepsilon < a \le \alpha.$
\end{lemma}
\begin{proof}
    Действительно, если все $a \le \alpha - \varepsilon$, то $\alpha$ не может быть точной верхней гранью потому что $\alpha - \varepsilon < \alpha.$
\end{proof}

\begin{definition}
Число $a'$ называется \textit{нижней гранью} множества $A$, если $x \ge a'$ для каждого $x \in A$. Если есть хотя бы одна нижняя грань, то множество называют \textit{ограниченным снизу}. Наибольшая из нижних граней множества $A$ называется \textit{точной нижней гранью} множества $A$ и обозначается $\mathrm{inf}(A)$ (= инфимум).
    Ограниченное и сверху, и снизу множество называется \textit{ограниченным.}
\end{definition}

\begin{lemma}
    Если $\inf(B) = \beta$, то для любого $\varepsilon>0$ существует хотя бы один $b\in B$, такой, что $\beta \le b < \beta + \varepsilon.$
\end{lemma}
\begin{proof}
 Если все $b \ge \beta +\varepsilon$, то $\beta$ не может быть точной нижней гранью, так как $\beta < \beta + \varepsilon.$
\end{proof}


\begin{mydanger}{\bf{!}}
    Из определения не следует, что супремум или инфимум существуют, ведь, скажем, для подмножества $A:= \{x>0, x^2 < 2\}$ множества $\mathbb{Q}$, число $\sqrt{2} = \sup (A)$, но $\sqrt{2} \notin \mathbb{Q}$, \ie в $\mathbb{Q}$ не у каждого подмножества есть супремум. 
\end{mydanger}

\begin{lemma}\label{simple_lemma}
 Пусть $A \subseteq \mathbb{R}$ -- ограниченное множество, тогда если все $a < \alpha$, $a\in A$, то $\sup A \le \alpha.$ Если все $a > \beta$, то $\inf(A) \ge \beta$. 
\end{lemma}
\begin{proof}
 Действительно, если все $\beta <a< \alpha$, то $\beta$, $\alpha$ -- нижняя и верхняя грани, соответственно. Теперь используя определения $\inf$ и $\sup$ мы завершаем доказательство. 
\end{proof}



\begin{theorem}[\textbf{Принцип полноты Вейерштрасса}]\label{W=complete}
    Если $A\subseteq \mathbb{R}$ -- непустое и ограниченное сверху (снизу) множество, то $\mathrm{sup}(A)$ (\textit{соотв.} $\mathrm{inf}(A)$) существует.
\end{theorem}
\begin{proof}
    Докажем эту теорему только в случае ограниченного сверху множества, так как другой случай доказывается аналогично.

    Пусть $B$ есть множество всех верхних граней множества $A$, тогда из условия об ограниченности сверху множества $A$ следует, что $B \ne \varnothing$. Далее из конструкции $B$ вытекает, что $A\le B$ (\ie $A$ левее чем $B$). Тогда по аксиоме о полноте получаем, что существует $c \in \mathbb{R}$ такой, что $A \le c \le B$. Тогда, во-первых, $c$ -- какая-то верхняя грань для $A$, а, во-вторых, $c$ меньше или равен  любому из элементов множества $B$, \ie $c = \mathrm{sup}(A)$, что и требовалось доказать.
\end{proof}



\begin{example}
    Пусть $A := (0,1]$. Тогда, например, числа $1.0000001, 2, 10, 9823750219581$ -- его верхние грани. А, например, числа $0, -0.1111111, -8347027408751$ -- его нижние грани.
Нетрудно видеть, что $1 = \sup(A)$. Действительно, если $1-\delta = \mathrm{sup}(A)$, где $\delta > 0$,  то рассмотрев $x:=1-\frac{\delta}{2}$ мы видим что $x\in A$ и $1-\delta < x$. Это и означает что ни при каких $\delta >0$, число $1 - \delta$ не может быть супремумом.

Аналогично можно показать, что $0 = \mathrm{inf}(A).$
\end{example}





\subsection{Ограниченные последовательности и теорема Вейерштрасса}

Вернёмся к последовательностям.

\begin{definition}
    Последовательность $(a_n)$ называется \textit{ограниченной}, если существуют такие $A,B \in \mathbb{R}$, что $A\le a_n \le B$ для всех $n\in\mathbb{N}.$
\end{definition}

\begin{lemma}\label{lim-->bounded_for_sequence}
    Если у последовательности $(a_n)$ есть предел $\lim_{n \to \infty} a_n = a$, то она ограничена.
\end{lemma}

\begin{proof}
    Действительно, согласно определению предела (см. определение \ref{limit_of_seqeunce}), для любого $\varepsilon>0$ найдётся такой номер $N$, что для всех $n \ge N$ выполняются неравенства
    \[
     a- \varepsilon < a_n < a+\varepsilon.
    \]

Пусть $\varepsilon = 1$ и пусть $N_1$ это такой номер, что для всех $n\ge N_1$ мы имеем
\[
     a-1 < a_n < a+1.
\]

Тогда пусть $R: = \max \{a_1, \ldots, a_{N_1 -1}, a+1\}$, и $L: = \min \{a_1,\ldots, a_{N_1 -1}, a-1\}$, тогда для всех $n\ge 1$, получаем
\[
 L \le a_n \le R
\]
\textit{т.е.,} последовательность $(a_n)$ -- ограничена.
\end{proof}


\begin{definition}
    Говорят, что последовательность $\{a_n\}$ \textit{не убывает} (соответственно, \textit{не возрастает}), если $a_n \le a_{n+1}$ (соответственно, $a_n \ge a_{n+1}$) для всех $n \ge 1.$ Последовательность, которая либо не убывает, либо не возрастает называется \textit{монотонной.}
\end{definition}


\begin{theorem}[{\bf Вейерштрасс}]\label{Weierstrass}
    Если последовательность не убывает (не возрастает) и ограничена сверху (снизу), то существует предел $\lim_{n \to \infty}a_n$, который равен $\mathrm{sup}\{a_n\}$ (соотв. $\mathrm{inf}(a_n)$).
\end{theorem}

\begin{proof}
    Будем доказывать только для не убывающих последовательностей, потому что для невозрастающих рассуждения аналогичные.

    Так как последовательность $\{a_n\}$ ограничена сверху, то по принципу полноты Вейерштрасса существует $\mathrm{sup}\{a_n\} = a$. Это означает, что для любого $\varepsilon >0$, $a-\varepsilon \ne \mathrm{sup}\{a_n\}$, \ie существует такой $N$, что $a_N >a -\varepsilon$.

    Далее, так как $a_n \le a_{n+1}$ (так как по предположению последовательность не убывающая), то все $a_{N+1}, a_{N+2}, \ldots > a -\varepsilon$. Так как $a = \mathrm{sup}\{a_n\}$, то $a_n \le a$ для всех $n$. Таким образом, для всех $n > N$, мы получаем $a- \varepsilon < a_n \le a$. Очевидно, что $a < a+\varepsilon$. Другими словами, мы для любого $\varepsilon >0$ нашли такое $N$, что для всех $n>N$ $a -\varepsilon < a_n < a + \varepsilon$, \ie $|a_n - a| < \varepsilon$. Но это и означает, что $\lim_{n \to \infty}a_n = a$, что и требовалось доказать.
\end{proof}

\begin{example}
    Рассмотрим последовательность $a_{n+1} = \frac{1}{2}\left(a_n + \frac{2}{a_n} \right)$, $a_1 = 2$. 

    Покажем, что она ограничена снизу: для этого воспользуемся неравенством $\frac{x+y}{2} \ge \sqrt{xy}$. Имеем
    \[
     a_{n+1} = c \ge \frac{1}{2}\cdot 2 \cdot \sqrt{a_n \cdot \frac{2}{a_n}} = \sqrt{2},
    \]
\textit{т.е.} $a_n \ge \sqrt{2}$ для всех $n \ge 1$.

Покажем, что она невозрастающая, \ie для всех $n\ge 1$, $a_{n} \ge a_{n+1}$.

Имеем
 \begin{eqnarray*}
     a_{n+1} - a_n &=& \frac{1}{2}\left(a_n + \frac{2}{a_n} \right) - a_n \\
     &= & \frac{a_n}{2} + \frac{1}{a_n} - a_n \\
     &=& -\frac{a_n}{2} + \frac{1}{a_n} \\
     &=& -\frac{1}{2}\cdot\left( \frac{a_n^2 - 2}{a_n^2}\right),
 \end{eqnarray*}
так как $a_n \ge \sqrt{2}$, то $a_{n+1} - a_n \le 0$, \textit{т.е.} $a_{n} \ge a_{n+1}$, а это значит, что она не возрастает. 

   Таким образом, по теореме Вейерштрасса эта последовательность имеет предел. Пусть $\lim_{n\to \infty}a_n =a$, тогда по теореме \ref{a+b,ca,ab} получаем
    \begin{eqnarray*}
        a &=& \lim_{n \to \infty}a_{n+1} \\
        &=&\lim_{n \to \infty} \frac{1}{2} \left(a_n+ \frac{2}{a_n} \right) \\
        &=&  \frac{1}{2}\left( \lim_{n \to \infty} a_n + \frac{2}{\lim_{n \to \infty} a_n} \right) \\
        &=&\frac{1}{2}\left( a + \frac{2}{a}\right).
    \end{eqnarray*}

Мы получили квадратное уравнение $a = \frac{1}{2}\left(a + \frac{2}{a} \right)$, откуда $a^2 = 2$. С учётом $a_n \ge \sqrt{2}$ получаем, что $a = \sqrt{2}$. Итак, $\lim_{n \to \infty}a_n = \sqrt{2}.$
\end{example}



\begin{example}
    Рассмотрим последовательность $e_n =\left(1 + \frac{1}{n} \right)^n$.
    
    Имеем
    \begin{eqnarray*}
        e_n &=& \left(1 + \frac{1}{n} \right)^n = \sum_{k=0}^n \binom{n}{k} \frac{1}{n^k}\\
        &=& 1 + 1 + \frac{1}{2!} \cdot \frac{n(n-1)}{n^2} + \frac{1}{3!}\cdot \frac{n(n-1)(n-2)}{n^3} + \cdots \\
        && + \frac{1}{(n-1)!} \cdot \frac{n(n-1)\cdots (n-(n-2))}{n^{n-1}} + \frac{1}{n!} \frac{n(n-1)\cdots (n-(n-1))}{n^n} \\
        &=& 2 + \frac{1}{2} \cdot 1\cdot \left(1-\frac{1}{n} \right) + \frac{1}{3!} \cdot 1 \cdot \left(1 - \frac{1}{n} \right) \left(1 - \frac{2}{n} \right)  + \cdots \\
        && + \frac{1}{(n-1)!} \cdot 1 \cdot \left(1 - \frac{1}{n} \right) \left(1 - \frac{2}{n} \right) \cdots \left(1 - \frac{n-2}{n} \right) + \frac{1}{n!} \cdot 1 \cdot \left(1 - \frac{1}{n} \right) \cdots \left(1 - \frac{n-1}{n} \right).
    \end{eqnarray*}

Заметим прежде всего, что так как $\frac{m}{n} >0$, то все $e_n >0$. Далее имеем
\begin{eqnarray*}
     e_{n+1} &=& 2 + \frac{1}{2}\cdot 1 \cdot\left(1 - \frac{1}{n+1} \right) + \frac{1}{3!} \cdot \left(1 - \frac{1}{n+1} \right)\left(1 - \frac{2}{n+1} \right) + \cdots \\
 && + \frac{1}{n!} \cdot 1 \cdot \left(1 - \frac{1}{n+1} \right) \left(1 - \frac{2}{n+1} \right) \cdots \left(1 - \frac{n-1}{n+1} \right) + \frac{1}{(n+1)!} \cdot 1 \cdot \left(1 - \frac{1}{n+1} \right) \cdots \left(1 - \frac{n}{n+1} \right).
\end{eqnarray*}

Так как $1 - \frac{t}{n} < 1 - \frac{t}{n+1}$, то $e_n \le e_{n+1}$. С другой стороны, каждая скобка $1- \frac{t}{n} <1$, тогда
\[
e_n \le \sum_{k=0}^n \frac{1}{k!} \le \sum_{k=0}^n \frac{1}{2^{k-1}} = 2 + 1 + \frac{1}{2} + \frac{1}{4} + \cdots + \frac{1}{2^{n-1}} \le 4.
\]
здесь мы воспользовались неравенством $k! \ge 2^{k-1}$. Таким образом, наша последовательность не убывает и ограничена сверху, а значит, по теореме Вейерштрасса у неё есть предел. Этот предел называется \textit{число Эйлера} и обозначается через $e$, при этом 
$$e \approx 2.7182818284590452353602874713527.$$

\end{example}



\section{Лекция \#5. Критерий Коши}

В результате работы с последовательностями мы приходим к выводу, что не всегда бывает легко установить, к чему сходится та или иная последовательность. Приходится придумывать разные трюки и способы. Но даже и этого мало во многих случаях. 

\begin{example}
    Рассмотрим последовательность
    \[
    a_n = \sum_{k=1}^n \frac{1}{k^2} = 1 + \frac{1}{2^2} + \frac{1}{3^2} + \cdots + \frac{1}{n^2},
    \]
    Нетрудно видеть, что она ограничена. Более того, очевидно, что она монотонна. Тогда по теореме Вейерштрасса она имеет предел $\lim_{n \to \infty}a_n = \sup\{a_n\}$. Но найти его теми способами, которыми мы располагаем, очень трудно. Оказывается, что $\lim_{n \to \infty}a_n = \frac{\pi^2}{6}.$
\end{example}

Однако теорема Вейерштрасса не всегда применима, ведь не все последовательности монотонны. 

\begin{example}
    Рассмотрим последовательность
    \[
    a_n = \sum_{k=1}^n \frac{(-1)^{k+1}}{k} = 1 -\frac{1}{2} + \frac{1}{3} - \frac{1}{4} + \cdots + \frac{(-1)^{n-1}}{n}.
    \]
    Трудно сказать, к чему сходится эта последовательность. Более того, она не монотонна; ведь мы то прибавляем, то отнимаем. Поэтому теорема Вейерштрасса тут не применима.
\end{example}

\subsection{Фундаментальная последовательность}

Таким образом, нам нужно научиться понимать, когда последовательность сходится, не используя предел. Более точно, мы должны научиться понимать, когда последовательность сходится в терминах самой последовательности, не прибегая к дополнительным методам. 

\textbf{Иными словами, нам нужен критерий сходимости любой последовательности.}

\begin{definition}\label{foundamental_sequence}
    Последовательность $(a_n)$ называется \textit{фундаментальной} (=удовлетворяет условию Коши, последовательностью Коши), если для любого $\varepsilon >0$ существует такой $N$, что $|a_n -a_m| < \varepsilon$ для всех $n,m \ge N$.
\end{definition}
\begin{comments}
    Идея этого определения заключается в следующем: если $\lim_{n\to \infty}a_n =a$, то, начиная с какого-то номера $N$, все $a_{N+1}, a_{N+2}, a_{N+3}, \ldots,$ \textbf{мало отличаются от} $a$, но тогда они ВСЕ также мало отличаются друг от друга.
\end{comments}

\begin{example}~
    \begin{enumerate}
        \item Рассмотрим последовательность $(a_n) = \{\frac{1}{n}\}$ и покажем, что она фундаментальна. Это значит, что для любого $\varepsilon >0$ мы должны предъявить такой номер $N$, что для всех $n,m > N$ будет верно неравенство $|\frac{1}{n}  - \frac{1}{m}| < \varepsilon$.

        Воспользуемся неравенством треугольника:
        \[  
          \left|\frac{1}{n}  - \frac{1}{m} \right| \le \left| \frac{1}{n}\right| + \left| \frac{-1}{m} \right| = \frac{1}{n} + \frac{1}{m}, 
        \]
        поэтому если мы найдём хотя бы один $N$ такой, что для всех $n,m >N$ будет верно, что $ \frac{1}{n} + \frac{1}{m} <\varepsilon$, то мы и докажем, что последовательность фундаментальна. 

        Но ведь если мы положим $N:=\lfloor \frac{2}{\varepsilon} \rfloor$, то для $n,m > \frac{2}{\varepsilon}$, мы тогда получаем, что 
        \[
         \frac{1}{n} + \frac{1}{m} < \frac{\varepsilon}{2} + \frac{\varepsilon}{2} = \varepsilon.
        \]

        То есть если $N:=\lfloor \frac{2}{\varepsilon} \rfloor$, то верно неравенство $|\frac{1}{n}  - \frac{1}{m}| < \varepsilon$ для всех $n,m >N$, что и показывает её фундаментальность.


        \item Рассмотрим последовательность $(a_n) = \{(-1)^n\}$. Пусть $\varepsilon = 1$, тогда для любого номера $N$, $|a_n - a_{n+1}| = 2 >1$. Таким образом, эта последовательность не фундаментальна.
    \end{enumerate}
\end{example}


\begin{lemma}\label{cap_of_intervals}
    Для любой последовательности $(I_n)$ бесконечного числа вложенных друг в друга отрезков на числовой прямой, \ie $I_{n+1} = [a_{n+1}, b_{n+1}] \subseteq [a_n, b_n] = I_n \subseteq \mathbb{R}$, длины которых стремятся к нулю, $\cap_{n=1}^\infty I_n \ne \varnothing.$ 
\end{lemma}

Другими словами, для таких отрезков есть общая точка.

\begin{proof}
    По условию имеем
    \[
    a_1 \le a_2 \le \ldots \le a_n \le \ldots \le b_n \le \ldots \le b_2 \le b_1,
    \]
    Это означает, что последовательность $(a_n)$ не убывает и ограничена сверху. Тогда по теореме Вейерштрасса она имеет предел $\lim_{n \to \infty} a_n = \sup{a_n}$. Обозначим его через $a$. 

    С другой стороны последовательность $(b_n)$ не возрастает и ограничена снизу, тогда мы получаем, что $\lim_{n\to \infty} b_n = \inf\{b_n\}$.
    
    Далее, $b_n = a_n + (b_n - a_n)$, тогда по арифметике предела:
    \[
     \lim_{n \to \infty}b_n = \lim_{n \to \infty} a_n + \lim_{n \to \infty} (b_n - a_n) = a + 0 =a,
    \]
    так как по условию длины стремятся к нулю, а $b_n - a_n$ и есть длина отрезка $I_n.$
 
    Таким образом, $\sup\{a_n\} = \inf\{b_n\} =a$, \ie это значит, что для всех $n$, $a_n \le a \le b_n$, что и означает $a \in \cap_{n=1}^\infty I_n.$
\end{proof}


\begin{theorem}[{Критерий Коши}]\label{Coshy}
   Последовательность сходится, если и только если она фундаментальна. 
\end{theorem}

\begin{proof}~\\
(1) Пусть $(a_n)$ -- сходящаяся последовательность, скажем, $\lim_{n \to \infty}a_n = a$. Тогда для любого $\varepsilon>0$ можно найти такой $N$, что для всех $n>N$, $|a_n - a| < \frac{\varepsilon}{2}$. Тогда для любых $n,m \ge N$
    \[
     |a_n - a_m| = |a_n - a + a - a_m| \le |a_n - a| + |a - a_m| < \frac{\varepsilon}{2} + \frac{\varepsilon}{2} =\varepsilon,
    \]
    что и означает её фундаментальность.

(2) Пусть $(a_n)$ -- фундаментальная последовательность. Выберем произвольную бесконечно малую последовательность $\{\varepsilon_k\}$, \ie $\lim_{k \to \infty} \varepsilon_k = 0$, но $\varepsilon_k >0$. Тогда для каждого $k$ найдутся такие $N_k$, что $|a_n - a_m| < \varepsilon_k$ при $n,m \ge N_k$. В частности $|a_n - a_{N_k}|<\varepsilon_k$, то есть все $a_n \in [a_{N_k}-\varepsilon_k, a_{N_k}+\varepsilon_k]$ при $n\ge N_k$.

Введём обозначения: $J_k:=[a_{N_k}-\varepsilon_k, a_{N_k}+\varepsilon_k]$, тогда видно, что $|J_k| = 2\varepsilon_k$, и тогда длины этих отрезков стремятся к нулю, так как $\lim_{k \to \infty} \varepsilon_k = 0$.

Покажем что $J_k \cap J_{k+1} \ne \varnothing$. Действительно, мы знаем что все $a_n \in J_k$ при $n\ge N_k$, а также все $a_p \in J_{k+1}$ при $p \ge N_{k+1}$, положим $N:=\max\{N_k,N_{k+1}\}$, тогда при $n>N$ все $a_n \in J_k \cap J_{k+1}$, \textit{i.e.,}  $J_k \cap J_{k+1} \ne \varnothing$.

Далее, положим
\[
 I_1:=J_1, \qquad I_2:=J_1 \cap J_2, \qquad I_3: = J_1 \cap J_2 \cap J_3,\quad  \ldots,
\]
таким образом, мы получили последовательность вложенных друг в друга отрезков 
\[
I_1 \supseteq I_2 \supseteq I_3 \supseteq \cdots,
\]
длины которых стремятся к нулю. Тогда по лемме \ref{cap_of_intervals} существует хотя бы одна $a \in \cap_{k=1}^\infty I_k$. Но тогда $a \in [a_{N_k}-\varepsilon_k, a_{N_k}+\varepsilon_k]$ для любого $k$, а это значит, что для любого $\varepsilon_k$ мы всегда знаем такой номер $N_k$, что $|a_n - a|<\varepsilon_k<2 \varepsilon_k$. Последнее в силу единственности предела и произвольной последовательности из $\varepsilon_k$ показывает, что $\lim_{n\to \infty}a_n = a$.
\end{proof}

\subsection{Новый взгляд на действительные числа.}\label{barQ=R}

Понятие фундаментальной последовательности естественным образом вводится в произвольном метрическом пространстве. Говорят, что метрическое пространство $X$ \textit{полное}, если любая фундаментальная последовательность сходится к какому-то элементу из этого же $X$. В противном случае, можно ввести операцию \textit{пополнения} $X \mapsto \bar X$, которая добавляет пределы ВСЕХ фундаментальных последовательностей. 

То есть эти последовательности и есть фундамент для построения анализа, исходя только из рациональных чисел. Поясним на примере:

Пространство $\mathbb{R}$, где расстояние между числами -- это модуль их разности является полным пространством. С другой стороны, пространство $\mathbb{Q}$ с той же метрикой полным уже не является. Например, рассмотренная ранее последовательность $a_{n+1} = \frac{1}{2}\left( a_n + \frac{2}{a_n} \right)$, $a_1 =2$, имеет предел $\sqrt{2}$, который не лежит в $\mathbb{Q}$. Тогда, добавив пределы всех фундаментальных последовательностей, мы и получим $\mathbb{R}$.

Другими словами, множество действительных чисел можно определить как множество фундаментальных последовательностей из $\mathbb{Q}$ с очевидными операциями. Если $\mathbf{a} = \{a_n\}$, $\mathbf{b} = \{b_n\}$, то положим
\begin{align*}
    & \mathbf{a} + \mathbf{b}: = \{a_n + b_n\}, \\
    & \mathbf{a} \cdot \mathbf{b}: = \{a_n \cdot b_n\}.
\end{align*}

Итак, действительные числа можно определить как
{\Huge
 \[
  \boxed{
   \boxed{
  \mathbb{R}: =  \overline{\mathbb{Q}}.
  }
  }
\]
}


\section{Лекция \#6. Подпоследовательности и частичные пределы.}

Мы будем часто употреблять фразу \textit{почти все}\label{almost_all}, которая означает все за исключением конечного числа. Например, неравенство в натуральных числах $n\ge 10$ верно для почти всех натуральных $n$. То есть оно неверно, если $n=1,2,\ldots,9$, но \textit{для всех остальных оно верно.}

\subsection{Понятие подпоследовательности}


\begin{definition}
    Пусть задана числовая последовательность $(a_n)$
    и задана последовательность строго возрастающих её номеров $n_1 <n_2< n_3 < \ldots$, тогда последовательность $\{a_{n_k}\}:= \{a_{n_1}, a_{n_2}, a_{n_3},\ldots,\}$ называется \textit{подпоследовательностью} последовательности $(a_n).$
\end{definition}

\begin{example}
    Очень важно понимать, что выбранные номера должны строго возрастать.
     \begin{enumerate}
         \item Рассмотрим последовательность $(a_n)=\{1,2,3,1,2,3,\ldots,\}$, тогда если положить, что $n_k = 2k$ для всех $k\ge 1$, то мы получаем подпоследовательность $a_{n_k} = \{2,1,3,2,1,3, \ldots,\}$. Но если мы выберем номера так, чтобы $n_1= 2, n_2 = 1, n_3=1, \ldots,$ то мы получим последовательность $2,1,1,\ldots,$ которая уже подпоследовательностью не является, так как $n_1 = 2 <n_2 = 1.$

         \item Пусть $(a_n) = \{1,1,1,5,5,5,5,\ldots,\}.$ Тогда $\{5,5,1,\ldots,\}$ не будет подпоследовательностью, потому что $n_1 \ge 4$, в то время как $n_3 \le 3$, \textit{i.e.,} $n_1 >n_3$. 
     \end{enumerate}
\end{example}

\begin{mydanger}{\bf{!}}
    Можно сказать, что любая последовательность есть ``сплетение'' подпоследовательностей; они как множества могут пересекаться, а могут и нет, но в объединении мы получаем всю нашу изначальную последовательность.
\end{mydanger}

\begin{lemma}\label{n_k>=k}
    Пусть $ (n_k )$ -- строго возрастающая последовательность натуральных чисел, тогда $n_k \ge k$ для каждого $k\ge 1.$
\end{lemma}
\begin{proof}
    Будем доказывать по индукции. Так как это последовательность натуральных чисел, то каждое $n_t\ge 1$, в частности $n_1\ge 1$. Пусть теперь верно неравенство $n_t \ge t$ для $t\ge 1$. Так как последовательность является строго возрастающей, то $n_{t+1}>n_t \ge t$, \textit{т.е.,} $n_{t+1} >t$ или что есть то же самое, что и $n_{t+1}\ge t+1$.
\end{proof}

\begin{theorem}\label{lim(sub)=lim}
    Пусть дана последовательность $(a_n)$ такая, что $\lim_{n\to \infty}a_n = a$, тогда $\lim_{n\to \infty}a_{n_k} =a$ для любой подпоследовательности $(a_{n_k} )$.
\end{theorem}
\begin{proof}
    Рассмотрим произвольную подпоследовательность $(a_{n_k} )$. Так как, по условию $\lim_{n \to \infty}a_n =a$, то для любого $\varepsilon >0$ можно найти такой номер $N$, что $|a_n - a| < \varepsilon$ для всех $n >N$. Тогда выберем такой номер $k$, что $k >N$. По Лемме \ref{n_k>=k}, $n_k \ge k$, тогда для всех $n_t \ge n_k >N$, $|a_{n_t}- a| < \varepsilon$, что и доказывает теорему.
\end{proof}

\begin{example}
 Вспомним про последовательность $(a_n) = ((-1)^n ) = \{-1,1,-1,1, \ldots, \}$. Нетрудно видеть, что подпоследовательности $\{a_{n_k}\} = \{-1\}$, $\{a_{n_t}\} = \{1\}$ с номерами $n_k = 2k+1$, $n_t = 2t$, $k,t \ge 1$ являются сходящимися подпоследовательностями. В то же время, сама последовательность предела не имеет (Пример \ref{(-1)^n}), хотя эта последовательность ограничена. Оказывается, имеет место следующий факт:    
\end{example}

\begin{theorem}[Больцано-Вейерштрасса]\label{B-W}
    Если $(x_n)$ есть ограниченная последовательность, то в ней есть сходящаяся подпоследовательность.
\end{theorem}
\begin{proof}~

    (1) Так как наша последовательность ограничена, скажем, $a \le x_n \le b$, то это значит, что все её элементы лежат в отрезке $I_1 = [a,b]$.
    
    (2) Разделим этот отрезок на две равные части и выберем ту половину, назовём её $I_2$, в которой находятся почти все элементы этой последовательности (\textit{i.e.,} бесконечное число её элементов). Если в обеих половинах находятся почти все элементы, то выбираем любую из них.

    (3) Ту же процедуру проведём для выбранной половины $I_2$ и так далее. В результате мы получаем систему отрезков $I_1, I_2, I_3, \ldots$, при этом их длины уменьшаются каждый раз в два раза, то есть $\lim_{n \to \infty}|I_n| = 0$. Более того, $I_1 \supseteq I_2 \supseteq I_3 \supseteq \cdots$, тогда по Лемме \ref{cap_of_intervals} $\cap_{n=1}^\infty I_n \ni \{c\}.$

    (4) Выберем теперь в каждом отрезке $I_k$ элемент нашей последовательности с номером $n_k$, при этом требуя, чтобы номер $n_{k+1}$ элемента, выбранного из следующего отрезка $I_{k+1}$, был строго больше предыдущего, \textit{i.e.,} $n_k<n_{k+1}$. Это можно сделать, потому что в каждом из отрезков $I_k$ имеется бесконечное число элементов нашей последовательности. Но это и означает, что для любого номера можно всегда найти такой элемент, номер которого будет заведомо больше выбранного.

    (5) Тогда по построению мы получили для любого $n_k$, $|x_{n_k}-c| < |I_k| = \frac{I_1}{2^{k-1}}$. Но это и означает, что $\lim_{n_k \to \infty}x_{n_k} = c.$
\end{proof}

Как показывает пример с последовательностью $(a_n) =  ((-1)^n )$, у последовательности могут быть подпоследовательности, у которых имеются разные пределы. Поэтому возникает вопрос, а какие пределы у всех подпоследовательностей могут быть?

\begin{definition}
    Предел какой-то подпоследовательности в последовательности $(a_n)$ называется \textit{частичным пределом} последовательности $(a_n)$.
\end{definition}

Наша цель -- описать все частичные пределы. Полностью эта задача будет решена, когда мы познакомимся с топологией вещественной прямой.

Начнём с примера.

\begin{example}
 Рассмотрим последовательность $ (x_n )$, где $x_n = (-1)^{n-1}\left(2 + \frac{3}{n} \right)$. Её первые десять элементов имеют вид
     \begin{center}
      \begin{tabular}{c|c|c|c|c|c|c|c|c|c|c|}
       $n$  & 1& 2 & 3 & 4 & 5 & 6 & 7 & 8 & 9& 10 \\
       \hline
       $x_n$  & 5& -3.5  & 3 & -2.75 & 2.6 & -2.5 & 2.42 & -2.37 & 2.3 & -2.27 
    \end{tabular}     
     \end{center}

 Мы видим, что $ (x_n ) = (x_{2n-1} ) \sqcup (x_{2n} )$, $x_{2n-1} = 2 + \frac{3}{2n-1}$, $x_{2n} = - 2- \frac{3}{2n}$. Тогда очевидно, что $x_{2p-1} > x_{2q}$ для любых $p,q \ge 1$, \ie графически это означает, что $ (x_{2n-1} )$ расположена выше оси $Ox$, а $ (x_{2n} )$ -- ниже оси $Ox$. Нетрудно видеть, что $\lim_{n \to \infty} x_{2n-1} = 2$, $\lim_{n \to \infty} x_{2n} = -2$.
 
 Далее очевидно, что $x_{2n-1}$ убывает, а последовательность $x_{2n}$ возрастает. Тогда $x_{2n-1} > x_{2n}$, следовательно $\lim \sup x_n = \lim \sup x_{2n-1} = 2$, $\lim \inf = \lim \inf x_{2n} = -2.$ 
   
\end{example}

\begin{example}
    Пусть $(a_n) = \{1,2,3,1,2,3,1,2,3,\ldots,\}$. Найдём все её частичные пределы. Пусть $b \ne 1,2,3$, и $b$ -- это какой-то частичный предел. Это значит, что найдётся бесконечное число номеров $n_k$ (которые и сформируют подпоследовательность) такие, что $|a_{n_k} - b| < \varepsilon$ для любого $\varepsilon >0$. Но взяв $\varepsilon = \frac{1}{2}$, мы видим, что неравенства $|1-b|, |2-b|, |3-b| < \frac{1}{2}$ при $b \ne 1,2,3$ не верны. С другой же стороны, если $b \in \{1,2,3\}$, то хотя бы одно из них верно для любого $\varepsilon>0$. 

    Это значит, что множество всех частичных пределов есть множество $\{1,2,3\}$, при этом $1<2<3.$ 
\end{example}

\begin{example}
    Пусть $(a_n' ) = \{-12, 15, 999, -\pi, \sqrt{2}, 1,2,3,1,2,3,1,2,3,\ldots,\}.$ Нетрудно видеть, что множество частичных пределов у этой последовательности ровно такое же, что и в предыдущем примере. Причина этого ясна: наличие конечного числа первых элементов не влияет на сам предел (так как в окрестности предела должно быть бесконечное число элементов последовательности!).
\end{example}

\subsection{Частичные пределы}


\begin{theorem}\label{from_bounded_sequence}
    Пусть $(a_n)$ -- ограниченная последовательность, тогда
    \begin{enumerate}
        \item Последовательность $M_k:=\sup_{n>k}\{a_n\}$;
        \[
         \begin{matrix}
             M_1 & := & \sup \{a_2, a_3, a_4,a_5 \ldots,\},\\
             M_2 & := & \sup \{a_3, a_4,a_5 \ldots,\},\\
             M_3 & := & \sup \{a_4,a_5 \ldots,\},\\
             \vdots && \vdots
         \end{matrix}
        \]
        не возрастает, ограничена и имеет предел, который называется \textit{верхним пределом последовательности} и обозначается $\lim \sup a_n$.

        \item Последовательность $m_k:=\inf_{n>k}\{a_n\}$;
         \[
         \begin{matrix}
             m_1 & := & \inf \{a_2, a_3, a_4,a_5 \ldots,\},\\
             m_2 & := & \inf \{a_3, a_4,a_5 \ldots,\},\\
             m_3 & := & \inf \{ a_4,a_5 \ldots,\},\\
             \vdots && \vdots
         \end{matrix}
        \]
        не убывает, ограничена и имеет предел, который называется \textit{нижним пределом последовательности} и обозначается $\lim \inf a_n$.

        \item Для любой подпоследовательности $ (b_n ) \subseteq  (a_n )$, 
        \[
         \lim \inf a_n \le \lim_{n \to \infty } b_n \le \lim \sup a_n.
        \]

    \end{enumerate}
\end{theorem}

Прежде чем доказать эту теорему, докажем следующую лемму:

\begin{lemma}\label{A<B=sup(A)<sup(B)}
    Пусть $A \subseteq B \subseteq \mathbb{R}$ -- два ограниченных подмножества в $\mathbb{R}$, тогда $\sup(A) \le \sup(B)$ (если $B$ ограничено сверху) и $\inf(A) \ge \inf(B)$ (если $B$ ограничено снизу). 
\end{lemma}
\begin{proof}
    Во-первых, заметим, что существование точной грани в зависимости от характера ограниченности обеспечивается принципом полноты Вейерштрасса (Теорема \ref{W=complete}). 

    Далее, если $\sup(B) = b$, то для любых $x \in B$ имеем $x \le b$, так как $A \subseteq B$, то и $x \le b$ для любого $x \in A$, что и доказывает требуемое. Аналогично доказывается и для $\inf$, где нужно учесть, что $\inf(A) = -\sup (-A)$.
\end{proof}

\begin{proof}[Доказательство теоремы.] Имеем $M_n : = \sup\left\{a_{n+1}, \ldots \right\}$, $M_{n+1}:=\sup \left\{a_{n+2}, \ldots,\right\}$ и так как
\[
\left\{a_n, a_{n+1}, \ldots \right\} \supseteq \left\{a_{n+1}, a_{n+2} \ldots \right\},
\]
то по Лемме \ref{A<B=sup(A)<sup(B)} $M_n \ge M_{n+1}$, \ie последовательность $ (M_n )$ не возрастает и также $m_n \le m_{n+1}$, \ie последовательность $ (m_n )$ не убывает.

Далее, так как последовательность $ (a_n )$ ограничена, значит, и $ (M_n )$, $ (m_n )$ --- ограничены. Тогда по теореме Вейерштрасса у них есть пределы.

Теперь мы должны показать, что эти пределы являются частичными пределами для нашей последовательности. Другими словами, мы должны предъявить такие подпоследовательности $ (a_{n_k} ) \subseteq  (a_n )$, $ (a_{n_r} ) \subseteq  (a_n )$, что $\lim_{n_k \to \infty} a_{n_k} =M$, $\lim_{n_r \to \infty} a_{n_r} =m$.

Мы это проделаем только для первого случая, так как второй аналогичный.

Так как $M_1 = \sup \left\{a_2, a_3, a_4,a_5 \ldots\right\}$, то найдётся хотя бы один элемент $a_{n_1} \in \{a_2, a_3, a_4,a_5 \ldots\}$ такой, что 
\[
 M_1 -1< a_{n_1} \le M_1,
\]
такой элемент найдётся, потому что $M_1 -1 \ne \sup \left\{a_2, a_3, a_4,a_5 \ldots\right\}.$ Продолжим теперь следующим образом: рассмотрим $M_{n_1}$, по определению
\[
 M_{n_1}: = \sup \left\{a_{n_1 +1}, a_{n_1+2}, a_{n_1+3},\ldots\right\},
\]
найдётся хотя бы один $a_{n_2} \in \{a_{n_1 +1}, a_{n_1+2}, a_{n_1+3},\ldots\}$ такой, что 
\[
 M_{n_1}-\frac{1}{2} < a_{n_2} \le M_{n_1},
\]
при этом очевидно, что $n_2>n_1.$

Продолжая по индукции, мы получаем следующее: если для $n_k$ мы нашли $a_{n_k}$ такой, что
\[
 M_{n_{k-1}} - \frac{1}{k} < a_{n_{k}} \le M_{n_{k-1}},
\]
при этом $n_{k}>n_{k-1}> \cdots > n_1$, то мы получили подпоследовательность $ \left(a_{n_k} \right).$

Ясно, что $\left(M_{n_k} \right)$ есть подпоследовательность последовательности $\left(M_n\right)$, но мы уже знаем, что $\lim_{n\to \infty}M_n = M$, тогда по теореме \ref{lim(sub)=lim}, $\lim_{n_k\to \infty}M_{n_k} = M$. Наконец, по лемме о зажатой последовательности (Лемма \ref{sqeezy}) получаем, что $\lim_{n_k \to \infty}a_{n_k} = M.$

Таким образом, мы построили подпоследовательность $\left(a_{n_k}\right)$, у которой предел равен $M$, что и доказывает, что $M$ -- это частичный предел последовательности $\left(a_n\right).$

Наконец, покажем, что любой частичный предел лежит в отрезке $[m,\,M].$ Возьмём произвольную подпоследовательность $\left(a_{n_t}\right)$, и пусть $\lim_{n_t \to \infty}a_{n_{t}} = a$. По определению чисел $m_t, M_t$ имеем: $m_{n_{t}-1} \le a_{n_t} \le M_{n_t-1}$. Тогда учитывая, что $\lim_{n_t \to \infty} m_{n_t} = m$, $\lim_{n_t \to \infty}M_{n_t} = M$ и Лемму \ref{a<b}, получаем $m\le a \le M$. Это завершает доказательство теоремы.
\end{proof}