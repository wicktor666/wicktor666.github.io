
\newpage
\begin{definition}
    Рассмотрим функцию $f: [a, b] \to \mathbb{R}$, где $|a|, |b| < \infty$. Разобьём этот отрезок на $n$ отрезков точками
    \[
    a = x_0 < x_1 < x_2 < \cdots < x_{n-1} < x_n = b.
    \]
Выберем в каждом отрезке $[x_i, x_{i+1}]$, $i=0,1,\ldots, n-1$, произвольную точку $\theta_i$, \textit{т.е.} $x_i \le y_i \le x_{i+1}$ и составим \textit{интегральную сумму}
\[
 \sum_{i=0}^{n-1} f(\theta_i)(x_{i+1} - x_{i}).
\]

Пусть далее $\Delta: = \max_{0 \le i \le n-1} \{x_{i+1} -x_i\}$ -- длина наибольшего из полученных отрезков, тогда если существует предел интегральной суммы при $\Delta \to 0$, \textit{то говорят, что функция $f(x)$ интегрируема на отрезке $[a,b]$ и пишут}
\[
 I:=\int_a^b f(x) \mathrm{d}x : = \lim_{\Delta \to 0}  \sum_{i=0}^{n-1} f(\theta_i)(x_{i+1} - x_{i}).
 \]
Предел этой суммы называется \textit{определённым интегралом от $f(x)$ по отрезку $[a,b]$.}

Таким образом, это определение означает, что для любого $\varepsilon >0$ существует такое $\delta >0$, что при любом разбиении отрезка $[a,b]$, 
\[
 a = x_0 < x_1 < \cdots < x_{n-1} < x_n = b
\]
длины которых меньше $\delta$, \textit{т.е.} $|x_i - x_{i+1}|<\delta$ и при любом выборе точек $x_i \le \theta_i < x_{i+1}$, $i = 0,1,\ldots, n-1$ выполняется неравенство
\[
 \left| \sum_{i=0}^{n-1} f(\theta_i)(x_{i+1} - x_{i}) -I \right|< \varepsilon. 
\]

Функция $f(x)$ называется \textit{подынтегральной функцией}, а $a,b$ -- \textit{пределами интегрирования.}
\end{definition}

Теперь мы займёмся вопросом когда функция интегрируема.

\begin{lemma}
    Если функция интегрируема на отрезке $[a,b]$, то она ограничена на нём.
\end{lemma}
\begin{proof}
    Пусть имеем функцию $f(x)$ которая не ограничена на отрезке $[a,b]$.
\end{proof}





\begin{remark}
    Итак, мы получили, что интеграл Римана от ограниченной функции $f:I \to \mathbb{R}$ не зависит от разбиения промежутка $I$. Как следствие мы получаем известное, но не очень удобное для работы определение интеграла Римана; \textit{для разбиения $\lambda(I)$ положим 
    $$\Delta := \max_{A\in \lambda(I)}\{|A|\} $$
    тогда интегралом Римана называется число (если оно существует)
    \[
    \int_I f: = \lim_{\Delta\to 0} \sum_{A\in \lambda(I), A \ne \varnothing}f(a)|A|
    \]
    }
\end{remark}



 Прежде всего докажем простые формулы для функций $f,g$
\[
 \inf (f+g) \ge \inf f + \inf g, \qquad \sup(f+g) \le \sup f + \sup g.
\]
Действительно, для любого $x\in I$ имеем
\[
 \inf f \le f(x), \quad \inf g \le g(x), \quad f(x) \le \sup f, \quad g(x) \le \sup g,
\]
тогда
\[
 \inf f + \inf g \le f(x) + g(x), \qquad f(x) + g(x) \le \sup f + \sup g,
\]
тогда и получаем формулы.
\[
 \inf (f+g) \ge \inf f + \inf g, \qquad \sup(f+g) \le \sup f + \sup g.
\]

(1) Имеем