 We have 
\begin{eqnarray*}
 \D(x) &=& a\partial(x), \\
 \D^2(x) &=& \D(a) \partial(x) + a^2\partial^2(x),\\
 \D^3(x) &=& \D^2(a) \partial(x) + a\D(a)\partial^2(x) + \D(a^2)\partial^2(x) + a^3 \partial^3(x) \\
 &=& \D^2(a)\partial(x) + (a\D(a) + \D(a^2))\partial^2(x) + a^3\partial^3(x),\\
 \D^4(x) &=& \D^3(a)\partial(x) + a\D^2(a)\partial^2(x) + \D(a\D(a) + \D(a^2))\partial^2(x) + a(a\D(a) + \D(a^2))\partial^3(x)\\
 &&+ \D(a^3)\partial^3(x) + a^4 \partial^4(x) \\
 &=& \D^3(a)\partial(x) + ( a\D^2(a) + \D(a\D(a) + \D(a^2)) ) \partial^2(x) \\
 &&+ (a^2\D(a) + a\D(a^2) + \D(a^3))\partial^3(x) + a^4\partial^4(x).
\end{eqnarray*}


\documentclass[a4paper]{article}

%% Language and font encodings
\usepackage[english]{babel}
\usepackage[utf8x]{inputenc}
\usepackage[T1]{fontenc}
\usepackage{amsfonts}
\usepackage{mathrsfs}

\usepackage[sc]{mathpazo}
\linespread{1.05}


%% Sets page size and margins
\usepackage[a4paper,top=3cm,bottom=2cm,left=3cm,right=3cm,marginparwidth=1.75cm]{geometry}

%% Useful packages
\usepackage{amsmath}
\usepackage{amssymb}
\usepackage{amsthm}
\usepackage{graphicx}
\usepackage[colorinlistoftodos]{todonotes}
\usepackage[colorlinks=true, allcolors=blue]{hyperref}

\usepackage{BOONDOX-frak} % to make mathfrak font more evilly
%Theorem Environments
\newtheorem{theorem}{Theorem}[section]
\newtheorem{lemma}[theorem]{Lemma}
\newtheorem{proposition}[theorem]{Proposition}
\newtheorem{corollary}[theorem]{Corollary}


\theoremstyle{definition}
\newtheorem{definition}[theorem]{Definition}
\newtheorem{example}[theorem]{Example}
\newtheorem{remark}[theorem]{Remark}
\newtheorem{construction}[theorem]{Construction}

\newcommand{\D}{\mathscr{D}}
\newcommand{\n}{\mathbf{n}}


%% Table packages
\usepackage{array}
\newcolumntype{P}[1]{>{\centering\arraybackslash}p{#1}}
\usepackage{multirow}


\title{LND}
\author{Viktor Lopatkin}

\begin{document}



\maketitle

\textbf{Claim:} \textit{Let $A$ be a ring and $\partial$ its LND. For any $a\in A$ a linear map $\mathscr{D}:=a\partial$ is a LND if and only if $A$ is a domain and $a \in \mathrm{Ker}(\partial)$}.

\textbf{Proof.}
By the Leibniz product rule it immediately follows that if $\D$ is LND then $\partial$ is so.

Let $\D$ be LND. Take $x \in A$ such that $x \ne 0$ and $\partial(x) \ne 0$. Assume that $\D^n(x) = 0$ for some $n \in \mathbb{N}$.

(1) Let $n=1$ then $\D(x) = a\partial(x) =0$ then if $A$ is a domain then $\partial(x) = 0$ therefore $n$ must be assumed to be greater than $1$.

(2) Let $n=2$ then $\D^2(x) = \D(a\partial(x))= \D(a)\partial(x)+ a\D(\partial(x)) = a\partial(a)\partial(x) + a^2\partial^2(x)$. Thus $\D^2(x)=0$ implies that $a\partial(a)=0$ and either $a^2=0$ or $\partial^2(x)=0$. Since $A$ is assumed to be a domain we then get $\partial^2(x)=0$, and $a\partial(a)=0$. Hence $\partial(a) = 0$ and the statement follows.

(3) Let $n\ge 3.$ It is cleat that any power of a derivation is a linear map.

Next,
\begin{eqnarray*}
 \D^n(x) &=& \D^{n-1}(\D(x)) = \D^{n-1}(a\partial(x))\\
  &=& \D^{n-2}(\D(a\partial(x))) = \D^{n-2}(\D(a)\partial(x) + a^2\partial^2(x)) \\
  &=& \D^{n-2}(\D(a)\partial(x)) + \D^{n-2}(a^2\partial^2(x)).
\end{eqnarray*}

It is obviously that $\D^{n-2}(a^2\partial^2(x))$ does not contain a term of a form $u\partial(x)$, where $u$ is a word in $a,\partial(a),\partial^2(a),\ldots, \partial^{n-1}(a)$.

Further, we obtain
\begin{eqnarray*}
 \D^{n-2}(\D(a)\partial(x)) &=& \D^{n-3}(\D^2(a)\partial(x) + a\D(a)\partial^2(x)) \\
 &=& \D^{n-3}(\D^2(a)\partial(x)) + \D^{n-3}(a\D(a)\partial^2(x))
\end{eqnarray*}

Similarly, we see that $\D^{n-3}(a\D(a)\partial^2(x))$ does not contain a term of a form $u\partial(x)$, where $u$ is a word as above. Continuing this line of reasoning we obtain that, by $\D^n(x)=0$, $\D^{n-1}(a)\partial(x) =0$. So, if $A$ is a domain and $\partial(x) \ne 0$ then $\D^n(x) =0$ implies that $\D^{n-1}(a)=0.$ Similarly, arguing as above we see that from $\D^{n-1}(a)=0$ it follows that $\D^{n-2}(a)\partial(a)=0.$ Continuing this line of reasoning we see that $\D^n(x) = 0$ implies that $a \underbrace{\partial(a) \cdots \partial(a)}_{n-1}=0$, and the statement follows.

\end{document}



 
Имеем отображение $(\Phi_n): \mathscr{W} \to \mathbb{R}^m$,
\[
 \Phi_n(x_1,\ldots, x_n, x_{n+1}, \ldots, x_{n+m}): = \Phi(a_1,\ldots, a_n, x_{n+1}, \ldots, x_{n+m})
\]

Итак, мы имеем непрерывно дифференцируемое отображение $\Phi:\mathscr{W} \to \mathbb{R}^m$, $\Phi(\m{a}) = \m{0}_m$ и якобиан отображения $\Phi_n:\mathscr{W} \to \mathbb{R}^m$ отличен от нуля, тогда по теореме о нявном отображении \ref{implicit_theorem}, найдутся окрестности $\mathscr{U}$, $\mathscr{V}$ точкек $(a_1,\ldots, a_n)$, $(a_{n+1}, \ldots, a_{n+m})$ соотвественно, и нерперывно дифференцируемое отображение $F:\mathscr{U} \to \mathscr{V}$ такое, что $(x_{n+1}, \ldots, x_{n+m}) = F(x_1,\ldots, x_n)$, \textit{т.е.,} мы получаем 
\begin{equation}\label{extr_connection_2}
  \begin{matrix}
     x_{n+1} &=& \varphi_1(x_1, \ldots, x_n),\\
     \vdots & & \vdots \\
     x_{n+m} &=& \varphi_m(x_1,\ldots, x_n),
 \end{matrix}    
\end{equation}
где $\varphi_1,\ldots, \varphi_m: \mathscr{U} \to \mathbb{R}$ -- непрерывно дифференцируемые функции которые задают $F$;
\[
 F: \begin{pmatrix}
     x_1 \\
     \vdots \\
     x_n
 \end{pmatrix} \mapsto \begin{pmatrix}
     \varphi_1(x_1,\ldots, x_n) \\
     \vdots \\
     \varphi_m(x_1,\ldots, x_n)
 \end{pmatrix}.
\]

\begin{mydanger}{\bf{!}}
 Иными словами, требование, чтобы значения переменных $x_1,\ldots, x_{n+m}$ удовлетворяли уравнениям связи (\ref{extr_connections}), можно заменить предположением, что переменные $x_{n+1}, \ldots, x_{n+m}$ представляют собой функции (\ref{extr_connection_2}) от $x_1,\ldots, x_n$.    
\end{mydanger}

Таким образом, вопрос об условном экстремуме для фукнции $f$ от $n+m$ переменных в точке $\m{a} = (a_1,\ldots, a_n, \ldots, a_{n+m})$ сводится к вопросу об обыкновенном экстремуме для функции $\widetilde{f}:\mathscr{U} \to \mathbb{R}$ уже от $n$ переменных
\[
\widetilde{f}(x_1,\ldots, x_n):= f(x_1,\ldots, x_n, \varphi_1(x_1,\ldots, x_n), \ldots, \varphi_m(x_1,\ldots, x_n))
\]
в точке $\m{a}_0 = (a_1, \ldots, a_n)$.

Другими словами функция $\widetilde{f}$ есть наклонная стрелка в коммутативной диаграмме
\[
 \xymatrix{
 \mathbb{R}^n \supseteq \mathscr{U} \ar@{->}[r]^{\overline{F}} \ar@{->}[rd]_{\widetilde{f}} & \mathscr{U} \times \mathscr{V} \ar@{->}[d]^f \ar@{^{(}->}[r] & \mathbb{R}^n \times \mathbb{R}^m  \\
 & \mathbb{R} &
 }
\]
где
\[
 \overline{F}: = (\mathrm{id}_\mathscr{U},F): \begin{pmatrix}
     x_1 \\
     \vdots \\
     x_n
 \end{pmatrix} \mapsto \begin{pmatrix}
     x_1 \\
     \vdots \\
     x_n \\
     \varphi_1(x_1,\ldots, x_n) \\
     \vdots \\
     \varphi_m(x_1,\ldots, x_n)
 \end{pmatrix},
\]
а $\mathscr{U} \times \mathscr{V} \hookrightarrow \mathbb{R}^n \times \mathbb{R}^m$ обозначает вложение.



Предположим теперь, что $f$ дифференцируема на некотором промежутке\footnote{промежуток это либо отрезок, либо интервал, либо полуинтервал.} который мы обозначим через $I.$ 

Таким образом, мы получаем функцию
\[
 \mathrm{d}f: I \to \mathbb{R}, \qquad I \ni x \mapsto (\mathrm{d}f)_{x},
\]
для которой
\[
 (\mathrm{d}f)_{x_0}(h) = f'(x_0) \cdot h = f'(x_0) \cdot  (\mathrm{d}x)_{x_0}(h), \qquad x_0 \in I.
\]



раз
\[
 \int \frac{[x]}{x^{\lambda +1}}\mathrm{d}x, \qquad \lambda \in \mathbb{R}, x \ge 1
\]
два
\[
 \int \left[ \frac{1}{\sqrt{x}} \right] \mathrm{d}x, \qquad  x \in (0,1].
\]

Здесь $[x]$ -- нижняя целая часть числа. 





Для каждого $A \in \lambda_f(I)$ найдём (если сможем) все такие $B_{i_1}, \ldots, B_{i_p}$, что $B_{i_1}, \ldots, B_{i_p} \subseteq A$. Тогда если $f(A)  = \alpha,$ $g(B_{i_t}) = \beta_t$, $1 \le t \le p$, то
\begin{eqnarray*}
    (f \pm g)(x_t) &=& \alpha + \beta_t, \\
    \max \{f,g\}(x_t) &=& \max \{ \alpha, \beta_t \}, \\
    (f\cdot g) (x_t) &=& \alpha \cdot \beta_t,\\
    \left( \frac{f}{g} \right)(x_t) &=& \frac{\alpha}{\beta_t}, \qquad \beta_t \ne 0.
\end{eqnarray*}

Если для $A$ таких $B_{i_t}$ не существует, то значит существует хотя бы одно $B \in \lambda_g(I)$, что $A \subseteq B$ и тогда мы найдём все такие $A_{j_1}, \ldots, A_{j_q}$ чтобы $\{A_{j_1}, \ldots, A_{j_q} \subseteq B\}$ и рассуждая аналогично, мы завершаем доказательство. 
















\begin{lemma}
    Пусть $I_1, I_2$ -- два интервала, тогда если $I_1 \cap I_2 \ne \varnothing$, то $I_1 \cup I_2$ -- тоже интервал.
\end{lemma}

\begin{proof}
    Пусть $I_1 =(a_1, b_1)$, $I_2 = (a_2, b_2)$
\end{proof}

\begin{theorem}
    Любое открытое множество $\mathscr{U} \subseteq \mathbb{R}$ есть объединение дизъюнктных интервалов\footnote{\textit{т.е.,} попарно непересекающихся интервалов}.  
\end{theorem}





\begin{theorem}
 Пусть $\mathscr{U},F \subseteq \mathbb{R}$ -- два непустых множества, при этом $\mathscr{U}$ -- открыто в $\mathbb{R}$, а $F$ -- замкнуто в $\mathbb{R}$. Тогда если $\mathscr{V} \subseteq \mathscr{U}$ -- открыто в $\mathscr{U}$, то оно открыто и в $\mathbb{R}$. Если $S\subseteq F$ -- замкнуто в $F$, то оно замкнуто и в $\mathbb{R}$.
\end{theorem}

\begin{proof}~\\
(1) Если $\mathscr{V}$
    
\end{proof}


\begin{proof}
    Так как $F$ -- дифференцируемо в $\mathscr{U}$, то для любых точек $\m{a}, \m{a}+\m{h} \in \mathscr{U}$ имеет место равенство
\[
F(\m{a} + \m{h}) - F(\m{a}) = (\mathrm{d}F_\m{a})\m{h} + \alpha(\m{h})\| \m{h}\|, \qquad \|\m{h}\| \to 0.
 \]

(1) Так как $\mathscr{U}$ -- открыто, то для любой точки $\m{c} \in \mathscr{U}$ найдётся открытый шар $B(\m{c},r) \subseteq \mathscr{U}$. Согласно Аксиоме Выбора мы можем выбрать любые две разные точки $\m{x},\m{y} \in B(\m{c},r)$. Рассмотрим отображение
\[
 \chi_{\m{x},\m{y}}: [0,1] \to \mathbb{R}^n, \qquad t \mapsto  \m{x}t + (1-t)\m{y},
\]
покажем, что $\chi_{\m{x},\m{y}}(t) \in B(\m{c},r)$ для любого $t \in [0,1]$. Действительно, имеем $\|\m{x} - \m{c}\|, \|\m{y} - \m{c}\| < r$, тогда для любого $t \in [0,1]$ получаем

\begin{eqnarray*}
  \| (t\m{x}+ (1-t)\m{y}) - \m{c}\| &=& \|  t\m{x}+ (1-t)\m{y}- t\m{c} - (1-t)\m{c}\|  \\
  &= & \| t(\m{x} - \m{c}) + (1-t)(\m{y} - \m{c})  \| \\
  &\le & t \| \m{x} - \m{c} \| + (1-t) \| \m{y}  -\m{c} \| \\
  &<& tr + (1-t)r = r,
\end{eqnarray*}
\textit{т.е.,} $\m{x}t + (1-t)\m{y} \in B(\m{a},r)$ для любого $t \in [0,1].$


(2) Пусть отображение $F$ определено следующим образом
\[
 F: \begin{pmatrix}
     x_1 \\ \vdots \\ x_n
 \end{pmatrix} \mapsto 
 \begin{pmatrix}
     f_1(x_1,\ldots, x_n) \\
     \vdots \\
     f_n(x_1,\ldots, x_n)
 \end{pmatrix}.
\]

Определим для каждого $1 \le i \le n$ функцию $\varphi_i(t):[0,1] \to B(\m{a}, r)$ следующим образом
\[
 \varphi_i(t): = f_i(t\m{x} + (1-t)\m{y}), \qquad 1\le i \le n,
\]
другими словами, $\varphi_i: = f_i \circ \chi_{\m{x},\m{y}}$,
\[
 \xymatrix{
  [0,1] \ar@{->}[rd]_{\varphi_i} \ar@{->}[r]^{\chi_{\m{x},\m{y}}} & B(\m{a},r) \ar@{->}[d]^{f_i} \\
  & \mathbb{R}
 }
\]

Пусть $\m{z}_t:=t\m{x} + (1-t)\m{y}$, тогда по теореме о дифференциале композиции \ref{d(FG)} получаем
\begin{eqnarray*}
 \varphi'_i(t) &=& (\mathrm{d}f_i)_{\m{z}_t} \cdot \chi'_{\m{x},\m{y}}(t) \\
 &=& (\mathrm{d}f_i)_{\m{z}_t} \cdot (\m{x} - \m{y}) \\
 &=& \begin{pmatrix}
     \dfrac{\partial f_i}{\partial x_1}({\m{z}_t}) & \ldots & \dfrac{\partial f_i}{\partial x_n} ({\m{z}_t})
 \end{pmatrix} \begin{pmatrix}
     x_1 - y_1 \\
     \vdots \\
     x_n -y_n
 \end{pmatrix} \\
 &=& \sum_{k=1}^n \dfrac{\partial f_i}{\partial x_k}({\m{z}_t}) (x_k - y_k) .
\end{eqnarray*}

(3) По теореме Лагранжа \ref{Langrange} существует такой $\vartheta_i \in (0,1)$, что 
\[
 \varphi_i(1) - \varphi_i(0) = \varphi_i'(\vartheta_i),
\]
тогда, положим, $\m{z}_i: = \vartheta_i \m{x} + (1-\vartheta_i)\m{y}$, и, принимая во внимание, что $\varphi_i(1) - \varphi_i(0) = f_i(\m{x}) - f_i(\m{y})$, мы получаем
\begin{eqnarray*}
 F(\m{x}) - F(\m{y}) &=& \begin{pmatrix}
   f_1(\m{x}) - f_1(\m{y})\\
   \vdots \\
    f_n(\m{x}) - f_n(\m{y})
 \end{pmatrix} = \begin{pmatrix}
     \dfrac{\partial f_1}{\partial x_1}(\m{z}_1) & \ldots & \dfrac{\partial f_1}{\partial x_n}(\m{z}_1) \\
     \vdots & \ddots & \vdots \\
     \dfrac{\partial f_n}{\partial x_1}(\m{z}_n) & \ldots & \dfrac{\partial f_n}{\partial x_n}(\m{z}_n)
 \end{pmatrix}\begin{pmatrix}
   x_1-y_1\\
   \vdots \\
    x_n - y_n
 \end{pmatrix} = J(\m{z}_1,\ldots, \m{z}_n)(\m{x} - \m{y}).
\end{eqnarray*}

В связи с этим рассмотрим функцию 
\[
 \mathrm{det}(J): \underbrace{\mathbb{R}^n \times \cdots \times \mathbb{R}^n}_n \to \mathbb{R}, \qquad (\m{x}_1,\ldots, \m{x}_n) \mapsto \mathrm{det} \begin{pmatrix}
     \dfrac{\partial f_1}{\partial x_1}(\m{x}_1) & \ldots & \dfrac{\partial f_1}{\partial x_n}(\m{x}_1) \\
     \vdots & \ddots & \vdots \\
     \dfrac{\partial f_n}{\partial x_1}(\m{x}_n) & \ldots & \dfrac{\partial f_n}{\partial x_n}(\m{x}_n)
 \end{pmatrix}
\]

Так функция $\mathrm{det}$ -- непрерывная, то $\mathrm{det}(J)$ непрерывная, с другой стороны, $\mathrm{det}J(\m{a},\ldots, \m{a}) = \mathrm{det} (\mathrm{d}F)_a$. По условию, $(\mathrm{d}F)_\m{a}$ -- обратима, \textit{т.е.} $\mathrm{det} (\mathrm{d}F)_a \ne 0$, \textit{т.е.} $\mathrm{det}(J(\m{a},\ldots,\m{a})) \ne 0$. Но тогда в силу непрерывности $\mathrm{det}(J)$ можно найти такой открытый шар $B(\m{a}, \varepsilon)$, что для любых $\m{z}_1,\ldots, \m{z}_n \in B(\m{a},\varepsilon)$, $\mathrm{det}(J)(\m{z}_1,\ldots, \m{z}_n) \ne 0.$

(4) Итак, пусть $\m{x},\m{y} \in B(\m{a},\varepsilon)$, $\m{x} \ne \m{y}$ тогда из предыдущих рассуждений мы получаем
\begin{equation}\label{F(x)-F(y)}
  F(\m{x}) - F(\m{y}) = J(\m{z}_1,\ldots, \m{z}_n)(\m{x}- \m{y})    
\end{equation}

и так как $\mathrm{det}(J)(\m{z}_1,\ldots, \m{z}_n) \ne 0$, то $J(\m{z}_1,\ldots, \m{z}_n) \ne 0$, и тогда $F(\m{x}) \ne F(\m{y}).$~\\

\boxed{
\boxed{
\textbf{Вывод: мы нашли такой шар $B(\m{a}, \varepsilon)$, что отображение $F: B(\m{a}, \varepsilon) \to F(B(\m{a}, \varepsilon))$ -- биекция!}
}}~\\

(5) Покажем, что $F(B(\m{a}, \varepsilon))$ -- открыто \textit{т.е.} для любой точки $\m{x} \in F(B(\m{a}, \varepsilon))$ найдётся такой шар малого радиуса с центром в $F(\m{x})$, который будет содержаться в $ F(B(\m{a}, \varepsilon)).$

Для таких целей мы рассмотрим функцию $\psi_\m{v}(\m{y}): = \| F(\m{y}) - \m{v}\|^2$ на замкнутом в $\mathscr{U}$ шаре $\bar B(\m{a}, \varepsilon)$, которую можно определить как наклонную стрелку в диаграмме
\[
 \xymatrix{
  \bar B(\m{a}, \varepsilon) \ar@{->}[r]^F \ar@{->}[rd]_{\psi_\m{v}} & F(\bar B(\m{a}, \varepsilon)) \ar@{->}[d]^{g_\m{v}}\\
  & \mathbb{R}
 }
\]
где $g_\m{v}(\m{b}): = \| \m{b} - \m{v}\|^2 = \left( \sqrt{ (b_1-v_1) + \cdots + (b_n - v_n)^2 } \right)^2 = (b_1-v_1) + \cdots + (b_n - v_n)^2$, где $\m{b} = (b_1,\ldots, b_n)^\top$ и $\m{v} = (v_1,\ldots, v_n)^\top.$

Тогда по теореме о дифференциале композиции \ref{d(FG)}
\begin{eqnarray*}
    (\mathrm{d} \psi_\m{v})_\m{y} &=&(\mathrm{d}g_\m{v})_{F(\m{y})} \cdot (\mathrm{d}F)_\m{y} \\
    &=& \begin{pmatrix}
        2(f_1(\m{y}) - v_1) & \ldots & 2(f_n(\m{y}) - v_n) 
    \end{pmatrix} \begin{pmatrix}
     \dfrac{\partial f_1}{\partial x_1}(\m{y}) & \ldots & \dfrac{\partial f_1}{\partial x_n}(\m{y}) \\
     \vdots & \ddots & \vdots \\
     \dfrac{\partial f_n}{\partial x_1}(\m{y}) & \ldots & \dfrac{\partial f_n}{\partial x_n}(\m{y})
     \end{pmatrix} 
\end{eqnarray*}
тогда для каждого $1 \le i \le n$
\begin{equation}\label{d(psi)}
  \dfrac{\partial \psi}{\partial x_i}(\m{y}) = 2 \sum_{k=1}^n \dfrac{\partial f_i}{\partial x_k}(\m{y}) (f_i(\m{y}) - v_i).    
\end{equation}
~\\


 \textbf{Мы хотим показать, что точка минимума этой функции находится в шаре $B(\m{a},\varepsilon)$.}

~\\

(6) Рассмотрим функцию $\rho_\m{x}: \bar B(\m{a},\varepsilon) \to \mathbb{R}$, $\rho_\m{x}(\m{y}): = \| F(\m{y}) - F(\m{x})\|$, которая, очевидно, непрерывна. Пусть $\m{s} \in S(\m{a},\varepsilon): = \{\m{y}\in \mathbb{R}^n\, :\, \| \m{y} - \m{a}\| = \varepsilon\}$, так как $S$ -- компакт (Следствие \ref{sphere_is_compact}), то согласно Теореме \ref{general_Weistrass} функция $\rho_\m{x}$, рассматриваемая только на $S$ достигает минимума, но так как $\rho_\m{x}(\m{s}) = \| F(\m{s}) - F(\m{x})\| = 0$, если и только если $\m{s} = \m{x}$ (потому что $F$ биективно на $B(\m{a},\varepsilon)$), то минимум этой функции -- положителен.

Тогда, можно выбрать положительное число $\delta>0$ так, чтобы $\rho_\m{x}(\m{s}) = \| F(\m{s}) - F(\m{x})\| > 2 \delta$ для всех $\m{s} \in S(\m{a}, \varepsilon).$

(7) Покажем теперь, что открытый шар $B(F(\m{x}), \delta) \subseteq F(B(\m{a}, \varepsilon))$. Пусть $\m{v} \in B(F(\m{x}), \delta)$, \textit{т.е.,} $\| F(\m{x}) - \m{v}\| < \delta$. Для любого $\m{s} \in S(\m{a}, \varepsilon)$ согласно Лемме \ref{x-y<|x-y|} получаем
\begin{eqnarray*}
    \| F(\m{s}) - \m{v}\| &=& \| ( F(\m{s}) - F(\m{x} )) - (\m{v} - F(\m{x}))\| \\
    &\ge & \Bigl|  \| F(\m{s}) - F(\m{x})  \| - \| \m{v} - F(\m{x}) \| \Bigr| \\
    &>& |2\delta - \delta| = \delta.
\end{eqnarray*}

Вернёмся к функции $\psi_\m{v}(\m{y}): = \| F(\m{y}) - \m{v}\|^2$, определённой на $\bar B(\m{a}, \varepsilon)$, тогда получаем 
\begin{align*}
    & \psi_\m{v}(\m{s}) = \| F(\m{s}) - \m{v}\|^2 > \delta^2, \qquad \m{s} \in S(\m{a},\varepsilon),\\
    &\psi_\m{v}(\m{x}) = \| F(\m{x}) - \m{v}\|^2 <\delta^2.
\end{align*}

С другой стороны, $\bar B(\m{a}, \varepsilon)$ -- компакт (так как он, очевидно, ограничен и замкнут согласно Лемме \ref{closed_ball=closed}), тогда функция (Теорема \ref{general_Weistrass}) $\psi_\m{v}$ принимает на нём минимальное значение, но тогда из неравенств выше вытекает, что точка минимума $\m{w} \in B(\m{a},\varepsilon)$.

(8) В таком случае, если $\m{w}$ -- точка минимума, то из необходимого признака \ref{nessary_condition_for_extr}, $\frac{\partial \psi}{\partial x_i}(\m{w})=0$, но так как $J \ne 0$ в шаре $B(\m{a},\varepsilon)$, то из (\ref{d(psi)}) вытекает, что $f_i(\m{w}) = v_i$ для каждого $1\le i \le n$, \textit{т.е.} $\m{v} = F(\m{w})$.

Итак, мы взяли произвольную точку $\m{v} \in B(F(\m{x}, \delta)$ и только что показали, что $\m{v} = F(\m{w})$, где $\m{w} \in B(\m{a},\varepsilon)$, но это и означает, что $B(F(\m{x}, \delta) \subseteq F(B(\m{a}, \varepsilon))$, \textit{т.е.} мы показали, что $F(B(\m{a}, \varepsilon))$ -- открыто.


(9) Нам осталось показать, что $F^{-1}$ -- дифференцируемо в $F(B(\m{a},\varepsilon))$, \textit{т.е.} для любой точки $\m{a}' \in F(B(\m{a},\varepsilon))$ существует такой линейный оператор $\Phi_{\m{a}'}$, что для любых двух точек $\m{a}', \m{a}' + \m{h}' \in F(B(\m{a},\varepsilon))$ имеет место равенство
\[
 F^{-1}(\m{a}' + \m{h}') - F^{-1}(\m{a}') = \Phi_{\m{a}'}(\m{h}') + \beta(\m{h}') \| \m{h}'\|, \qquad \m{h}' \to \m{0}_n.
\]

Пусть $\m{a}: = F^{-1}(\m{a}')$ и $\m{a} + \m{h} := F^{-1}(\m{a}' + \m{h}')$, \textit{т.е.} $F(\m{a}) = \m{a}'$ и $F(\m{a} + \m{h}) = \m{a}' + \m{h}'$. 

Тогда
\begin{eqnarray*}
    \Phi_{\m{a}'}(\m{h}') &=& \Phi_{\m{a}'}\left( F(\m{a} + \m{h}) - \m{a}'  \right) \\
    &=& \Phi_{\m{a}'}\left( F(\m{a} + \m{h}) - F(\m{a}) \right)
\end{eqnarray*}
так как $F$ -- дифференцируемо в $\mathscr{U}$, то согласно определению 
\[
 F(\m{a} + \m{h}) - F(\m{a}) = (\mathrm{d}F_\m{a})\m{h} + \alpha(\m{h})\| \m{h}\|, \qquad \|\m{h}\| \to 0
\]
тогда, требуя, чтобы $\Phi_{\m{a}'}$ была обратной к матрице $(\mathrm{d}F)_\m{a}$, получаем
\begin{eqnarray*}
    \Phi_{\m{a}'}(\m{h}')&=& \Phi_{\m{a}'}\left( (\mathrm{d}F_\m{a})\m{h} + \alpha(\m{h})\| \m{h}\| \right) \\
    &=& \m{h} + \Phi_{\m{a}'}(\alpha(\m{h}))\cdot \| \m{h}\|
\end{eqnarray*}

Рассмотрим теперь 
\[
 \beta(\m{h}'): = \frac{\|\m{h}\|}{\| \m{h}'\|}\Phi_{\m{a}'}(\alpha(\m{h})) ,
\]
и покажем, что $\lim_{\m{h}' \to \m{0}_n} \beta(\m{h}') = 0$.

Вернёмся к равенству (\ref{F(x)-F(y)}) и рассмотрим теперь функцию $J(\m{z}_1,\ldots, \m{z}_n)$ на замкнутом шаре $\bar B(\m{a},\varepsilon)$, тогда по теореме \ref{W=complete} эта функция принимает минимальное значение $\mu$, \textit{т.е.} мы получаем
\[
 \| F(\m{x}) - F(\m{y})\| = \| J(\m{x}-\m{y})\| \ge \mu \|\m{x} - \m{y}\|.
\]

Так как $\m{h}'= F(\m{a} + \m{h}) - F(\m{a})$, то из неравенства выше, получаем
\[
  \| \m{h}'\| =\| F(\m{a} + \m{h}) - F(\m{a}) \| \ge  \mu \| \m{h} \|.
\]

Тогда
\[
 \beta(\m{h}'): = \frac{\|\m{h}\|}{\| \m{h}'\|}\Phi_{\m{a}'}(\alpha(\m{h})) \le \mu \Phi_{\m{a}'}(\alpha(\m{h})),
\]
в силу непрерывности $\Phi_{\m{a}'}$ (так как это линейный оператор, Лемма \ref{linear_is_contious}), мы получаем, что 
\[
 \beta(\m{h}'): = \Phi_{\m{a}'}(\alpha(\m{h})) \to 0, \qquad \m{h}' \to \m{0}_n,
\]
что и показывает $\lim_{\m{h}' \to \m{0}_n} \beta(\m{h}') = 0$, а тогда имеет место равенство 
\[
 F^{-1}(\m{a}' + \m{h}') - F^{-1}(\m{a}') = \Phi_{\m{a}'}(\m{h}') + \beta(\m{h}') \| \m{h}'\|, \qquad \m{h}' \to \m{0}_n.
\]

Тем самым, теорема об обратной функции полностью доказана.
\end{proof}
